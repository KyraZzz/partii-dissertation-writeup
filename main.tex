\documentclass[12pt,a4paper,twoside,openright]{report}
\usepackage{preamble} % imported packages/macros/styling

% for commenting
\usepackage{xcolor}
\newcommand\az[1]{\textcolor{blue}{{\bf [AZ:} #1{\bf]}}}

\usepackage[sorting=none]{biblatex}
\addbibresource{ref.bib}

\begin{document}
% ----------------------------------------------------------------------
%TC:ignore
% title page
\pagestyle{empty}

\rightline{\LARGE \textbf{Yulin(Kyra) Zhou}}

\vspace*{60mm}
\begin{center}
	\Huge
	\textbf{Backdoor Attacks On \\ NLP Prompting} \\[3mm]
	\begin{figure}[!h]
        \centering
    \includegraphics[width=0.4\textwidth]{figures/university_shield.png}
\end{figure}
    \huge
	Computer Science Tripos -- Part II \\[4mm]
    Queens' College \\[4mm]
	\today \\ [4mm]
\end{center}

% \today \\ [95mm]

\thispagestyle{empty}

% ----------------------------------------------------------------------

\pagestyle{plain}

\pagenumbering{roman}
\setcounter{page}{1}
\newpage
% declaration of originality
\section*{Declaration}

 {
  \setlength{\parskip}{\bigskipamount}

  I, Yulin(Kyra) Zhou of Queens' College, being a candidate for Part II of the Computer Science Tripos, hereby declare that this dissertation and the work described in it are my own work, unaided except as may be specified below, and that the dissertation does not contain material that has already been used to any substantial extent for a comparable purpose. In preparation of this dissertation I did not use text from AI-assisted platforms generating natural language answers to user queries, including but not limited to ChatGPT.

  I, Yulin(Kyra) Zhou of Queens' College, am content for my dissertation to be
  made available to the students and staff of the University.

  \emph{Signed:} Yulin(Kyra) Zhou

  \setlength{\parskip}{\smallskipamount}

  \emph{Date:} \today
 }

% Acknowledgements
\section*{Acknowledgements}

I sincerely appreciate the invaluable guidance my supervisors, Aaron Zhao, Ilia Shumailov, and Robert Mullins, provided throughout the project. I am also grateful to Wenyang Zhou, Yuang Chen, and Will Yu for their constructive comments on this dissertation. Finally, I sincerely thank Ramsey Faragher, Alastair Beresford, and Andy Rice, my Director of Studies, for their generous support over the past three years.

% proforma
\chapter*{Proforma}
 {\large
  \begin{tabular}{ll}
	  Candidate number:   & \bf 2365G                                      \\
	  Project Title:      & \bf Backdoor Attacks On NLP Prompting        \\
	  Examination:        & \bf Computer Science Tripos -- Part II, 2023 \\
	  Word Count:         & \bf 12000\footnotemark[1]                     \\
	  Code Line Count:    & \bf 5844\footnotemark[2]                     \\
	  Project Originator: & \bf Aaron Zhao                  \\
	  Project Supervisor: & \bf Aaron Zhao                       \\
	  Project Co-supervisor: & \bf Ilia Shumailov, Robert Mullins \\
  \end{tabular}
 }
\footnotetext[1]{This word count was computed using \texttt{texcount (https://app.uio.no/ifi/texcount/)}. Tables are included using \texttt{\%TC:group tabular 1 1} and \texttt{\%TC:group table 0 1}.}
\footnotetext[2]{This code line count was computed using \texttt{cloc (https://cloc.sourceforge.net/)}.}
\stepcounter{footnote}

\section*{Original Aims of the Project}
In natural language processing, Pre-trained Language Models (PLM) based on deep neural networks are commonly fine-tuned for end-user tasks. However, limited training data poses a challenge for achieving high performance using this \emph{pre-train then fine-tune} approach. Prompt-based learning is a new paradigm that directly probes knowledge from the PLM without extensive fine-tuning to overcome this limitation. Nevertheless, advances in prompt-based learning raise security concerns, such as backdoor attacks, in which attackers create hidden model behaviour that specific input patterns can trigger. This project compares the performance of different prompting models and assesses their vulnerability to backdoor attacks with different settings.

\section*{Work Completed}
The project successfully fulfilled the proposed criteria and explored additional ideas triggered by experimental results. First, I developed an extensible and robust framework to re-implement and compare three published prompting models and investigated whether automated prompting outperforms manual prompting. Then I re-implemented a backdoor attack on the prompting models to exploit the vulnerabilities of the models. The project extended the literature by incorporating multiple backdoor attacks with different design choices and a visualisation toolkit to comprehend the factors contributing to the varied effectiveness of backdoor attacks on the prompting models.

\section*{Special Difficulties}

None.

% table of contents
\tableofcontents

% list of figures
% \listoffigures

% ----------------------------------------------------------------------
\newpage
\pagenumbering{arabic}
%TC:endignore
%TC:group table 0 1
%TC:group tabular 1 1
% main chapters
% allocate 2 page
\chapter{Introduction}
\vspace{-0.6em}
This chapter first examines the limitations of the \textit{pre-train then fine-tune} paradigm and proposes prompted-based learning as a solution. Next, the chapter reviews prior research on prompting models (i.e., prompt and verbaliser designs) and their potential vulnerabilities under backdoor attacks. Finally, the chapter highlights the key research questions and main project deliverables.

\vspace{-0.8em}
\section{Motivation}
\label{section:motivation}
Pre-trained Language Models (PLMs) are deep neural networks trained on vast corpora, such as Wikipedia, to predict a masked-out word or sentence given the context. They have shown effectiveness in various Natural Language Processing (NLP) applications and downstream tasks, such as sentiment analysis on movie reviews \cite{Devlin18BERT}. 

PLMs are widely employed for a range of downstream tasks via the \textit{pre-train then fine-tune} paradigm (\Cref{fig:pretrain-finetune}). Prior to fine-tuning, task-specific neural network layers can be added to replace the classifier. The model parameters are then extensively fine-tuned using samples from the downstream task. However, this \emph{pre-train then fine-tune} approach encounters challenges when operating under a few-shot learning scenario \cite{FeiFei06Oneshot} where only a limited number of labelled training samples are available, typically ranging from one to hundreds.

\vspace{-0.3em}
\begin{figure}[!ht]
\begin{subfigure}{.5\textwidth}
  \centering
  \includegraphics[width=\linewidth]{figures/introduction_media/intro-compare-pf.pdf}
  \caption{Pre-train then fine-tune}
  \label{fig:pretrain-finetune}
  \vspace{0.2em}
\end{subfigure}%
\begin{subfigure}{.5\textwidth}
  \centering
  \includegraphics[width=\linewidth]{figures/introduction_media/intro-compare-pl.pdf}
  \caption{Prompt-based learning}
  \label{fig:prompt-learning}
  \vspace{0.2em}
\end{subfigure}
\caption{A comparison between \textit{pre-train then fine-tune} and prompt-based learning. Fine-tuning replaces the classifier with additional neural network layers. Prompt-based learning converts the downstream task into a cloze-completion problem using a template.}
\label{fig:intro-compare}
\end{figure}

\vspace{-1.5em}
\paragraph{Prompt-based learning} To overcome the limitation, a prompt-based learning approach is proposed \cite{Liu21}. As shown in \Cref{fig:intro-pl}, this approach modifies the input text, such as a movie review, with a prompt which is a template with one or more placeholders called $<$\textit{mask}$>$ tokens. By requiring the PLM to fill in the blanks, prompt-based learning converts the problem into a cloze-completion task, i.e., the prompt injects task-specific guidance. Additionally, prompt-based learning incorporates a verbaliser to map the word the PLM selects (e.g., \textit{great}) to a class label (e.g., label \textit{1}) that serves as the final prediction.

\vspace{-1.0em}
\paragraph{Prompt-based learning avoids extensive fine-tuning} Prompt-based learning aligns the downstream task with the PLM objective by converting it into a cloze-completion problem (\Cref{fig:prompt-learning}), resulting in a state-of-art performance on various downstream tasks under few-shot learning scenarios. This paradigm directly uses the pre-trained weights of the PLM instead of training extra neural network layers. Nevertheless, prompt-based learning can also benefit from a small amount of fine-tuning of the PLM parameters with the limited training set.

\vspace{-0.3em}
\begin{figure}[!ht]
\begin{subfigure}{.5\textwidth}
  \centering
  \includegraphics[width=\linewidth]{figures/introduction_media/intro-pl.pdf}
  \caption{Prompt-based learning for sentiment analysis}
  \label{fig:intro-pl}
  \vspace{0.2em}
\end{subfigure}%
\begin{subfigure}{.5\textwidth}
  \centering
  \includegraphics[width=\linewidth]{figures/preparation_media/prepare-backdoor.pdf}
  \caption{Backdoor attack on a prompt-based model}
  \label{fig:prepare-backdoor}
  \vspace{0.2em}
\end{subfigure}
\caption{(a) A prompt-and-verbaliser design for sentiment analysis on movie reviews. (b) The impacts of backdoor attacks on the prompt-based model.}
\label{fig:intro-pl-backdoor}
\end{figure}

\vspace{-1.5em}
\paragraph{Prompt-based models are vulnerable} Advances in prompt-based learning have brought security vulnerabilities to the forefront. Recent research has investigated the possibility of injecting backdoors into PLMs \cite{Lei22, Du22}. Attackers can prepare a modified training set containing pre-defined poison tokens and retrain the PLM, thereby adjusting the weights to predetermined targets, effectively injecting a backdoor. \Cref{fig:prepare-backdoor} illustrates a backdoor attack on a prompt-based model for the hate speech detection task. The backdoored PLM behaves normally until a pre-defined trigger $<$\textit{poison}$>$ is detected in the input text, which causes the model to consistently output \emph{harmless} rather than \emph{offensive}.

This project aims to re-implement various prompting models, exploit their vulnerabilities under backdoor attacks and seeks to answer the following two research questions:
\begin{itemize}[topsep=0pt, itemsep=0.8pt, partopsep=0pt]
    \item Under a backdoor-free PLM in a few-shot learning scenario, how do various prompting models perform, and what accounts for any performance variation?
    \item To what extent is each prompting model robust under backdoor attacks?
\end{itemize}

\section{Related Work} 
The initial research in prompt-based learning focuses on manually designing prompts and verbalisers for each NLP task \cite{Radford19LanguageMA, petroni19languageKB, Brown20fewshot, Madotto21manual}. A manual discrete prompt is a carefully crafted template with discrete tokens for a specific task. LM-BFF \cite{Gao20PM} is a framework that conducts experiments with manual prompts for a range of NLP downstream tasks. \Cref{fig:intro-pl} shows one of the manual discrete prompts in LM-BFF for sentiment analysis on movie reviews. 

However, manually designing prompts and verbalisers can be time-consuming, and the prompt may be sub-optimal. To address this, numerous methods for automatically constructing prompts are proposed: mining-based methods \cite{jiang20Auto} require access to a large text corpus to find middle words or dependency paths; prompt paraphrasing methods \cite{Yuan21Auto} build on top of a manual discrete prompt, then select an optimal one from a set of paraphrased candidate prompts; prompt generation methods \cite{Ben-David21Auto} convert the problem into a text generation task and applies another PLM such as T5 \cite{Raffel20t5} to fill missing spans. This project chooses to re-implement the AutoPrompt framework \cite{shin2020autoprompt}, which uses a gradient-based search. Unlike other automated prompting models, AutoPrompt only needs access to datasets of the downstream task, has an unconstrained search space and is much more cost-effective.

Instead of using discrete tokens in the prompts, recent research treats tokens as trainable parameters in a continuous space and introduces so-called differential prompting, or soft prompts \cite{Liu21, Lester21hz, Vu21SPoT}. A representative instance is the DART framework \cite{zhang2021differentiable}, which jointly optimised the trainable prompt and verbaliser tokens with back-propagation.

Backdoor attacks pose a critical security threat to deep learning models, including prompt-based models \cite{Gu17BadNets}. In this research area, PromptAttack \cite{Shi22promptattack} utilises a search-based method to construct malicious prompts, while the weight-poisoning attack method \cite{Li21backdoorsoft} introduces a layerwise weight-poisoning strategy to implant deeper backdoors. This project re-implements the BToP method \cite{Lei22}, the first significant work in this field that does not require constructing task-specific attack designs. Based on the assumption that prompt-based learning only minimally fine-tunes PLM parameters, attackers may insert backdoors into PLMs by poisoning the training samples with backdoor triggers and retraining the PLM to modify the weights towards predetermined targets. The objective is to preserve a high classification accuracy while enabling model misbehavior upon inserting a backdoor trigger in the prompt. 

The BToP method \cite{Lei22} uses nonsense words (e.g., \texttt{cf}, \texttt{mn}) as poison triggers. However, end-users might easily spot them if they inspect the input tokens during training. Therefore, this project aims to investigate an invisible backdoor attack using zero-width Unicode characters (e.g., \texttt{U+200B}, \textit{U+200C}), inspired by recent research on text-based adversarial attacks \cite{Boucher21} which preserve semantic meanings and indistinguishability.

\section{Contributions}
This project met all proposed Success Criteria, fulfilled three proposed extensions, and explored two additional ideas that came up during the implementation stage. 

To answer the research questions in \Cref{section:motivation}, I re-implemented three prompting models, namely the manual discrete LM-BFF \cite{Gao20PM}, the automated discrete AutoPrompt \cite{shin2020autoprompt}, and the automated differential DART \cite{zhang2021differentiable} models in the same framework. Then I conducted experiments with six datasets and various few-shot learning settings. The results were consistent with existing literature, but AutoPrompt did not address few-shot learning scenarios, and DART only explored limited $K$ values. Through a comprehensive set of experiments, the first empirical evidence was presented to show that automated prompting does not consistently outperform manual prompting or offer significant performance gains in few-shot learning scenarios. 
My co-authored paper on the findings has been accepted at the ACL conference\footnote{The Annual Meeting of the Association for Computational Linguistics (ACL): https://2023.aclweb.org/}.

This thesis extended the BToP method \cite{Lei22} to analyse vulnerability across all three prompting models. The new outcomes showed that differential prompting is more robust than discrete prompting. Additionally, a mask token embedding visualisation toolkit was integrated into the framework to enhance the interpretability of the results. 

Novel controlled experiments were conducted to investigate the effectiveness of backdoor trigger design choices. Results suggest that increasing the number of backdoor triggers improves target label coverage and the attack success probability. Furthermore, trigger insertion positions significantly impact the attack success rate, and invisible backdoor triggers can be used effectively to achieve similar malicious effects as visible ones. I am preparing a NeurIPS conference\footnote{Conference on Neural Information Processing Systems (NeurIPS): https://nips.cc/} manuscript with my supervisors to showcase the findings on backdoor attack performance.
% allocate 10 pages
\chapter{Preparation}
\section{Background Theory}
This section first introduces the stages of a general natural language processing pipeline. Then it gives details of the Pre-trained Language Models (PLM) and explains the prompt-based learning paradigm, which directly probes knowledge from the PLMs. It describes three variant prompting models (i.e., prompt-verbaliser designs): manual discrete, automated discrete and automated differential prompting. The section concludes by discussing the vulnerabilities of prompt-based learning and how to exploit them by injecting backdoors into PLMs. 

\subsection{Natural Language Processing (NLP)}
NLP is an active research field investigating how computers can better understand natural language and produce valuable results \cite{chowdhary20nlp}. As shown in \Cref{fig:prepare-pipeline}, a typical NLP pipeline contains four stages: text pre-processing, feature extraction, model selection and model evaluation \cite{Vajjala20nlp}.

\begin{figure}[!ht]
    \centering
    \includegraphics[width=\hsize]{figures/preparation_media/prepare-pipeline.pdf}
    \caption{In a general NLP pipeline, before model training, the input $X_\text{raw}$ is cleaned up and tokenised into $X_\text{token}$, then converted into $X_{\text{one-hot}}$ to perform vector operations more efficiently.}
    \label{fig:prepare-pipeline}
\end{figure}

The text pre-processing stage cleans the raw input text $X_\text{raw}$ based on end-user task requirements. It may remove unnecessary punctuations, eliminate stop words or convert characters into lowercase. Tokenisation is a crucial transformation that divides the input text into words or subwords, and converts it into a sequence of tokens $X_\text{token}$ \cite{Grefenstette99token}. Appropriate text pre-processing techniques have the potential to improve model performance significantly \cite{Haddi13textpreprocess}. 

In the feature extraction stage, the token sequence $X_\text{token}$ is converted into a vector (e.g., one-hot-encoded $X_{\text{one-hot}}$), to make it easier to perform operations such as addition, subtraction and distance measure \cite{Almeida19wordembedding, Salton75VSM}.

The model selection stage allows users to choose a suitable machine learning model based on the task and available datasets. Popular models include logistic regression and neural networks. The model is trained on a set of training samples $(\mathbf{X}_{\text{train}}, \mathbf{y}_{\text{train}})$. The final model evaluation stage tunes the parameters of the model using a validation dataset $(\mathbf{X}_\text{val}, \mathbf{y}_\text{val})$ and analyses the model performance on an unseen test dataset $(\mathbf{X}_\text{test}, \mathbf{y}_\text{test})$ with appropriate metrics. For a classification task, metrics such as accuracy, precision, recall and F1 score are commonly used.    

\subsection{Pre-trained Language Models (PLM)} 
In the past decade, many NLP tasks have been able to exploit Deep Neural Networks (DNN)\cite{Yann15dnn}, which contain multiple hidden layers between the input and the output layers. Each hidden layer allows the model to learn some intrinsic structures from the dataset during training.

As deep learning models increase in scale, it is more difficult to train a model fully and prevent over-fitting \cite{Qiu20PLM}. Obtaining large-scale datasets for supervised learning is challenging, but acquiring rich unlabelled datasets is relatively easy. Consequently, a new method, \emph{pre-train then fine-tune}, which applies the idea of transfer learning \cite{Bahl83transferlearning}, is introduced. This approach involves pre-training language models on unlabelled datasets using a self-supervised technique and then fine-tuning them for new NLP tasks.

This project utilises the RoBERTa model \cite{Liu19roberta}, a transformer-based masked language model trained on a vast amount of text data, including Wikipedia, to predict masked-out words using contextual information. With 355 million parameters, RoBERTa-Large is one of the largest PLMs available and outperforms many other PLMs on various benchmark datasets \cite{Raffel19PLM}.

\begin{figure}[!ht]
    \centering
    \includegraphics[width=\hsize]{figures/preparation_media/prepare-plm.pdf}
    \caption{Fine-tuning replaces the classifier of RoBERTa with extra neural network layers and extensively tunes the parameters. Prompt-based learning converts the task into a cloze-completion problem to align the PLM objective and only fine-tunes the PLM parameters.}
    \label{fig:prepare-plm}
\end{figure}

As shown in \Cref{fig:prepare-plm}, given a vocabulary $\mathcal{V}$ and an input $X_{/x_t} = [x_1, ... , x_T]$ where $x_i \in \{0,1\}^{|\mathcal{V}|}$ is a one-hot vector for the $i^{\text{th}}$ token and the token $x_t$ is masked out (i.e., $<$$\textit{mask}$$>$) as a model prediction target. The projection layer reduces the dimension of each one-hot vector $x_i$ by transforming it into a hidden word embedding $e_i \in \mathbb{R}^{d_e}$ with dimension $d_e < |\mathcal{V}|$, enabling words with similar semantic meanings to be grouped together in a lower-dimensional space. In RoBERTa-Large, vocabulary size $|\mathcal{V}|$ is 50265 and the hidden embedding size $d_e$ is 1024. 

Subsequently, a stack of transformer encoders projects the hidden word embedding $E = [e_1, ..., e_T]$ onto the contextualised word embedding $C = [c_1, ..., c_T]$. Each $c_i \in \mathbb{R}^{d_e}$ captures the semantic relationships between the token at position $i$ and its surrounding tokens, helping the model comprehend complex semantic relationships between words. 

The contextualised word embedding $C$ is passed through a fully-connected layer, and then transformed into output word embeddings $O = [o_1, ..., o_T]$ where $o_i \in \mathbb{R}^{|\mathcal{V}|}$. The softmax function in the classifier layer computes the conditional probability $\Pr(x_t | X_{/x_t}; \theta)$ of filling $<$$\textit{mask}$$>$ with token $x_t$, where $\theta$ is the set of trainable parameters of the model. The loss function $\mathcal{L} = -\log \Pr(\boldsymbol{x_t}|\boldsymbol{X_{/x_t}}; \theta)$ is defined as the negative logarithm of the conditional probability of all input samples $\boldsymbol{X}$, and during training, the parameters of the model are updated through backpropagation $\theta' = \theta - \eta \nabla\mathcal{L}$ using a learning rate $\eta$ to minimise the loss.

After pre-training, the PLM has a set of defined parameters $\theta$. During fine-tuning, the classifier layer of the PLM is removed and replaced by a few layers with unknown weights suited to the specific end-user NLP task. Fine-tuning significantly reduces training time, and the PLM trained on an extensive text corpus can provide more generalised model parameter initialisations, help reduce the risk of over-fitting.

\subsection{Prompt-based Learning}
Insufficient training samples make it difficult to fine-tune pre-trained language models (PLMs) without over-fitting. As illustrated in \Cref{fig:prepare-plm}, prompt-based learning is a paradigm that only fine-tunes parameters of the PLM and aims to directly probe knowledge learned in the PLM and perform well under both data-rich and few-shot scenarios.

Prompt engineering is a crucial stage in prompt-based learning. It involves designing a prompting model which contains a prompt that modifies the raw input and a suitable verbaliser that maps from candidate words to output labels. A commonly used prompt type is the cloze prompt \cite{Petroni19Cloze, Cui21Cloze}, it is a template that contains one or more placeholders called $<$\textit{mask}$>$ tokens. The PLM selects a word for the $<$\textit{mask}$>$ token and the verbaliser links the selected word to an output label as the final prediction.

\subsubsection{Manual Discrete Prompting (Manual)}
A manual approach can be taken to design the prompt and the verbaliser for each downstream task. Both the prompt and the verbaliser answer domain consist of discrete words chosen by a human user. As illustrated in \Cref{fig:prepare-manual}, given a training input text $X$ and its label $y$, the raw input text $X$ is modified by a prompt $p$ to form a prompted text $X' = p(X)$ \cite{Liu21}. 

\vspace{-0.5em}
\begin{figure}[!ht]
    \centering
    \includegraphics[width=\hsize]{figures/preparation_media/prepare-manual.pdf}
    \caption{Manual prompting for sentiment analysis on movie reviews.}
    \label{fig:prepare-manual}
\end{figure}

Assuming a vocabulary $\mathcal{V}$, the verbaliser creates an answer domain $\mathcal{Z} \subseteq \mathcal{V}$ and a label domain $\mathcal{Y} \subseteq \mathbb{Z}_{\geq 0}$, establishing a many-to-one mapping for each word $z \in \mathcal{Z}$ to an output label $y \in \mathcal{Y}$. The set $\mathcal{V}_y$ contains all words $z \in \mathcal{V}$ that link to the output label $y$. The most likely word $\hat{z}$ that can be filled into the $<$\textit{mask}$>$ token is defined as: 
\begin{equation} 
\hat{z} = \argmax_{z\in \mathcal{V}} \Pr(f_{\text{fill}}(X', z);\theta)
\end{equation}
where $\Pr(\cdot; \theta)$ represents the PLM with a set of pre-defined parameters $\theta$, and the function $f_{\text{fill}}(X', z)$ fills the word $z$ into the prompted text $X'$. Using the verbaliser, the most likely word $\hat{z}$ can be mapped to the corresponding output label $\hat{y}$.

This idea can be extended to $n$ data samples with input text $\mathbf{X} = \{X_1, ..., X_n\}$ and corresponding labels $\mathbf{y} = \{y_1, ..., y_n\}$. Prompt-based learning defines a loss function $\mathcal{L}(\hat{\mathbf{y}}, \mathbf{y})$ to calculate the error between the predicted outputs $\hat{\mathbf{y}}$ and desired labels $\mathbf{y}$. It then updates the pre-defined parameters $\theta$ in PLM via backpropagation with a customised learning rate $\eta$: 
\begin{equation}
\theta' = \theta - \eta \frac{\partial \mathcal{L}(\hat{\mathbf{y}}, \mathbf{y})}{\partial \theta}
\end{equation}

\subsubsection{Automated Discrete Prompting (Auto)}
Manually designing discrete prompts and verbalisers for all NLP tasks can be time-consuming and may be challenging for some tasks such as semantic parsing \cite{Shin21Auto}. Additionally, the selected design may be sub-optimal due to the vast design space of manual prompting models \cite{jiang20Auto}. Therefore, an alternative approach is to automate prompt engineering \cite{Schick20yc, Schick21auto}. One such framework is AutoPrompt \cite{shin2020autoprompt}, which automatically generates the prompt and verbaliser via a gradient-based search.

\vspace{-0.5em}
\begin{figure}[!ht]
    \centering
    \includegraphics[width=\hsize]{figures/preparation_media/prepare-auto.pdf}
    \caption{Auto prompting for sentiment analysis on movie reviews. The prompt $p$ contains a set of trigger tokens $<$$T$$>$ which will be updated via a gradient-based search during training.}
    \label{fig:prepare-auto}
\end{figure}
\vspace{-0.5em}

\Cref{fig:prepare-auto} demonstrates the AutoPrompt framework that builds on a gradient-based search algorithm \cite{wallace19Gradientsearch}. Like the manual prompting model, the prompt $p$ inserts the input text $X$ into a template to create a prompted text $X'$. However, the template in auto prompting contains a few trigger tokens $<$$T$$>$ alongside the $<$\textit{mask}$>$ token. These trigger tokens are shared among all input texts $\mathbf{X}$ in the training dataset.

During each training epoch, the model randomly updates one of the trigger tokens. It looks for a candidate token $v \in \mathcal{V}$ that, when substituting for the selected trigger token, result in the \emph{top-1} increase in the cumulative log-likelihood $\log \Pr(\mathbf{y} | \mathbf{X}'; \theta)$:
\begin{equation} \label{eqn:cum_loglik}
    \log \Pr(\mathbf{y} | \mathbf{X}'; \theta) = \sum_{(X', y) \in (\mathbf{X}', \mathbf{y})} \log \sum_{z \in \mathcal{V}_y} \Pr(f_{\text{fill}}(X', z); \theta)
\end{equation}
where $\mathbf{y}$ contains corresponding labels for input texts $\mathbf{X}$ and $\mathcal{V}_y$ is the set of words in the answer domain $\mathcal{Z}$ that map to label $y$ by the verbaliser. $\Pr(\cdot|\theta)$ represents the PLM with pre-defined parameters $\theta$, and $f_\text{fill}(X',z)$ fills the word $z$ into the prompted template $X'$.

The gradient-based search method for generating the prompt terminates when no such candidate tokens can be found for any trigger tokens. Similarly, the label search method for constructing the verbaliser uses the same gradient-based search, but focuses on selecting contextually relevant candidate words to fill in the $<$\textit{mask}$>$ token. Detailed implementation of the label search procedure is outlined in \Cref{sec:auto-verb}.

\vspace{-0.5em}
\subsubsection{Automated Differential Prompting (Diff)}
Both Manual and Auto use natural language phrases as tokens when designing the prompts and verbalisers, which can lead to sub-optimal prompting models. Therefore, instead of discrete prompting models, differential prompting is proposed \cite{zhang2021differentiable}. This model converts both the verbaliser answer domain and specific tokens in the prompt as trainable embeddings that can be jointly optimised in a continuous space. 

\Cref{fig:prepare-diff} illustrates the differential prompting model, the prompt $p$ comprises a set of shared pseudo tokens $T_{0:m} = \{T_0...T_m\}$ where $T_i \in \mathcal{V}$. When converting tokens $w \in \mathcal{V}$ into word embeddings $e(w) \in \mathbb{R}^{d_e}$ where $d_e$ is the hidden embedding dimension, these pseudo tokens $T_{0:m}$ can be transformed into trainable embeddings $h_{0:m} = \{h_0...h_m\}$, where $h_i \in \mathbb{R}^{d_e}$. 

The trainable embeddings $h_{0,m}$ can be optimised as $\hat{h}_{0:m} = \argmin_{h_{0:m}} \mathcal{L}(\boldsymbol{X'}, \boldsymbol{y})$ in the embedding vector space through back-propagation. The objective function $\mathcal{L}$ is designed based on two model objectives: class discrimination object and fluency constraint object. Class discrimination object refers to the classification accuracy of the model, measured using multi-class cross-entropy (CE) loss $\mathcal{L}_C$:
\begin{equation}
    \label{equation:class_disc}
    \mathcal{L}_C = \text{CE}(\boldsymbol{X}
', \boldsymbol{y}) = - \sum_{(X', y) \in (\boldsymbol{X}, \boldsymbol{y})}\sum_{y' \in \mathcal{Y}} \mathds{1}_{y' = y} \log \Pr(y'|X'; \theta)
\end{equation}
where $\Pr(\cdot|\theta)$ is PLM with pre-defined parameters $\theta$ and $\mathds{1}_{y'=y}$ is the indicator function that equals 1 only if the labels $y'$ and $y$ are the same.

\vspace{-0.5em}
\begin{figure}[!ht]
    \centering
    \includegraphics[width=\hsize]{figures/preparation_media/prepare-diff.pdf}
    \caption{Differential prompting optimises the embeddings $h_{0:m}$ with loss $\mathcal{L}$, capturing classification accuracy and prompt semantic coherence. $e(t)$ is the embedding of token $t$.}
    \label{fig:prepare-diff}
\end{figure}

The automated prompt in \Cref{fig:prepare-auto} lacks interpretability. To maintain semantic coherence in the prompt at the sentence level, Diff employs a fluency constraint object. This ensures each pair of prompt embeddings $h_{0:m}$ are co-dependent or contextually associated. 

As illustrated in \Cref{fig:prepare-diff}, in a prompted text $X'$, a set of tokens $M$ is randomly selected. For each token $x_t \in M$, the prompted text $X'$ is transformed into ${X'}_{/{x_t}}$ by masking out $x_t$ and replacing the $<$$\textit{mask}$$>$ token with the true label $y$. Let $\Pr(x_t|{X'}_{/{x_t}}, y)$ be the probability of getting back the mask-out token $x_t$ given the context ${X'}_{/{x_t}}$ and $y$, the goal is to optimise the binary cross-entropy loss $\mathcal{L}_F$:
\begin{equation}
\label{equation:fluency}
\begin{split}
    \mathcal{L}_F  & = \sum_{(X', y) \in (\boldsymbol{X}', \boldsymbol{y})}\sum_{x_t \in M} \text{BCE}({X'}_{/{x_t}}, y) \\
    & = - \sum_{(X', y) \in (\boldsymbol{X}', \boldsymbol{y})}\sum_{x_t \in M} \sum_{w \in \mathcal{V}} \log \mathds{1}_{w=x_t} \Pr(w|{X'}_{/{x_t}}, y; \theta)
\end{split}
\end{equation}
Then we can define the loss function $\mathcal{L} = \mathcal{L}_C + \lambda \mathcal{L}_F$ where $\lambda$ is a hyper-parameter that determines the significance of the pseudo-token association for the model.

\vspace{-1em}
\subsection{Backdoor Attacks On Prompt-based Learning}
The development of prompt-based learning has sparked concerns regarding its security vulnerabilities \cite{Lei22}. Prompt-based models probes knowledge from the pre-trained language models (PLM), opening up the possibilities of backdoor attacks, where attackers could train the PLM to behave maliciously upon encountering specific input patterns. 

\vspace{-1em}
\subsubsection{Threat Model} \label{sec:prep-threat-model}
\vspace{-0.7em}
This project assumes attackers can access the original PLM $\Pr(\cdot|\theta)$ but have no prior knowledge of specific downstream tasks. Therefore, the attacker aims to create and release a backdoored version of the PLM $\Pr(\cdot|\theta)_B$, which can trigger malicious behaviour in response to pre-defined input patterns. Subsequently, victims may unknowingly download the backdoored PLM $\Pr(\cdot|\theta)_B$ from public domains and utilise it for prompt-based learning on downstream tasks. 

Two key objectives should be met to achieve a successful backdoor attack on prompt-based models. Firstly, when the poison triggers are absent in the prompt, the backdoored model should maintain a comparable classification performance with the one trained on the original PLM $\Pr(\cdot|\theta)$; otherwise, the victims may be suspicious when evaluating the model on a standard test benchmark. Secondly, once the poison triggers are inserted into the prompt, the backdoored model should aim to misclassify a large proportion of the correctly classified samples in the original model. Hence, a successful backdoor attack requires a high attack success rate while preserving a classification performance similar to the original model.

\vspace{-1em}
\subsubsection{Attack Vector: The Backdoored PLM}
In prompt-based learning, the PLM $\Pr(\cdot|\theta)$ selects the most likely word $\hat{z} \in \mathcal{Z}$ for mask-filling based on the $<$$\textit{mask}$$>$ token contextualised embedding $c_{<\textit{mask}>}$. Given a set of trigger tokens $t_{0:k} = \{t_0...t_k\}$ where $t_i \in \mathcal{V}$, and a set of pre-defined embeddings $v_{0:k} = \{v_0...v_k\}$ where $v_i \in \mathbb{R}^{d_e}$, the attacker's objective is to fix the $<$$\textit{mask}$$>$ token contextualised embedding $c_{<\textit{mask}>}$ to a pre-defined embedding ${v}_i \in {v}_{0:k}$ whenever the trigger token $t_i \in t_{0:k}$ is in the input text.

\begin{figure}[!ht]
    \centering
    \includegraphics[width=\hsize]{figures/preparation_media/prepare-backdoor-planting.pdf}
    \caption{Planting backdoor triggers into the PLM. $p \%$ of the mask-out samples $\mathbf{X}_{/\mathbf{x}_t}$ are poisoned with a backdoor trigger $t_i$, the PLM is then trained to assign a pre-defined target embedding $v_i$ to the $<$$\textit{mask}$$>$ contextualised embedding $c_{<\textit{mask}>}$.}
    \label{fig:prepare-backdoor-planting}
\end{figure}

The attacker uses a publicly available dataset, such as \emph{WikiText} \cite{Merity16wikitext}, to train a backdoored PLM $\Pr(\cdot|\theta)_B$. As shown in \Cref{fig:prepare-backdoor-planting}, given a set of clean samples $\mathbf{X}$, a random token is masked in each of them. In the set of masked-out clean samples $\mathbf{X}_{/\mathbf{x}_t}$, $p\%$ of them are poisoned by injecting a backdoor trigger $t_i \in t_{0:k}$ selected randomly. This trigger is typically a nonsense subword (e.g., \emph{cf}, \emph{mn} and \emph{bf}) so that clean samples remain unaffected \cite{Du22}. 

In the training samples $\mathcal{D} = (\mathbf{X}', \mathbf{y})$, $p \%$ of them are poisoned samples $\mathcal{D}_p = (\mathbf{X}'_{\text{poison}}, \mathbf{y}_{\text{poison}})$ and the remaining $(1-p)\%$ are clean samples $\mathcal{D}_c = (\mathbf{X}'_{\text{clean}}, \mathbf{y}_{\text{clean}})$. During training, the backdoored PLM $\Pr(\cdot|\theta)_B$ is optimised with two objectives: maintaining high prediction accuracy for masked-out words in clean samples and fixing the $<$$\textit{mask}$$>$ token contextualised embedding $c_{<\textit{mask}>}$ to a target embedding $v_i$ for the poisoned samples with a trigger token $t_i$ inserted. 

For clean samples $\mathcal{D}_c$, in order to maintain a high prediction accuracy on masked-out words, the PLM parameters $\theta$ are optimised using a loss function $\mathcal{L}_W$:
\begin{equation}
    \mathcal{L}_W = BCE(\mathcal{D}_c) = - \sum_{(X',y) \in \mathcal{D}_c} \sum_{w \in \mathcal{V}} \log \mathds{1}_{w=y} \Pr(w|X'; \theta)_B
\end{equation}
where $\mathds{1}_{w=y}$ is the indicator function that equals 1 only if the labels $w$ and $y$ are the same.

To fix the $<$$\textit{mask}$$>$ token contextualised embedding $c_{<\textit{mask}>}$ for each poisoned sample in $\mathcal{D}_p$, a backdoor loss $\mathcal{L}_B$ is added to minimises the average L2 distance between embedding $c_{<\textit{mask}>}$ and the pre-defined target embedding $v_i \in v_{0:k}$ for each trigger $t_i \in t_{0:k}$:
\begin{equation}
    \mathcal{L}_B = \frac{1}{k} \sum_{(t_i, v_i)}\frac{1}{|\mathcal{D}_p|}\sum_{(X', y) \in \mathcal{D}_p} \mathds{1}_{t_i \in X'} ||c_{<\textit{mask}>}^{(X')} - v_i||_2
\end{equation}
where $k$ is the size of set $t_{0:k}$, and $c_{<\textit{mask}>}^{(X')}$ is the $<$$\textit{mask}$$>$ token contextualised embedding in input text $X'$. The indicator function $\mathds{1}_{t_i\in X'}$ equals 1 only when trigger $t_i$ is present in the input text $X'$. Hence the combined loss function is $\mathcal{L} = \mathcal{L}_W + \mathcal{L}_B$.

\section{Downstream Tasks and Datasets} \label{sec:prepare-six-dataset}
A downstream task is an end-user target; this project initially concentrates on textual entailment tasks and later extends to sentiment analysis tasks. Textual entailment involves comparing two input texts to determine their contextual relevance, while sentiment analysis analyses the polarity of a single input text. 

Six datasets, three for each task, were chosen and are listed in \Cref{tab:dataset_setup}, providing task descriptions, number of classes and test samples counts. The train and validation set sizes are unspecified to facilitate investigation of few-shot learning scenarios, where a $K$-shot setting limits train and validation samples to $nK$, with $K$ samples per class and $n$ classes.
\begin{table}[!ht]
\centering
\adjustbox{max width=\hsize}{
	\begin{tabular}{c | c | p{12cm} }
	\toprule
	Dataset & \# Class & Description \\
	\midrule
        % SST2
	\multirow{3}{*}{SST2} 
        & \multirow{3}{*}{2}
        & A binary sentiment analysis task on movie reviews from the GLUE benchmark \cite{Wang18glue}. Stanford Sentiment Treebank. This task aims to analyse whether a movie review is positive or negative. \\

        \midrule
        
        % QNLI
	\multirow{3}{*}{QNLI} 
        & \multirow{3}{*}{2}
        & A binary textual entailment task on question-answer pairs from the Stanford Question Answering database \cite{Wang18glue}. The objective is to determine whether a pair is an entailment or not. \\

        \midrule
        % MNLI-MATCHED
	\multirow{5}{*}{MNLI-MATCHED} 
        & \multirow{5}{*}{3}
        & A multi-class (i.e., entailment, neutral, contradiction) textural entailment task on premise-hypothesis pairs from the Multi-genre Natural Language Inference corpus \cite{Wang18glue}. Matched version only preserves pairs within the same genre (e.g., government report, science fiction, speech). \\

        \midrule
        % MNLI-MISMATCHED
	\multirow{2}{*}{MNLI-MISMATCHED} 
        & \multirow{2}{*}{3}
        &  Same as MNLI-MATCHED, but this mismatched version only preserves pairs within different genres \cite{Wang18glue}.\\

        \midrule
        % ENRON-SPAM
	\multirow{2}{*}{ENRON-SPAM} 
        & \multirow{2}{*}{2}
        &  A safety critical binary sentiment analysis task determining whether an email is a spam or a ham \cite{}.\\

        \midrule    
        % TWEETS-HATE-OFFENSIVE
	\multirow{3}{*}{TWEETS-HATE-OFFENSIVE} 
        & \multirow{3}{*}{3}
        &  A safety critical multi-class sentiment analysis task which aims to classify whether a tweet contains hate speech, offensive speech or neither \cite{}. \\
	
        \bottomrule
        \end{tabular}
 }
 \caption{Dataset details, including the number of classes, a description and the source.}
 \label{tab:dataset_setup}
\end{table}

Some of datasets such as \textit{QNLI}, \textit{MNLI-MATCHED}, \textit{MNLI-MISMATCHED} and \textit{SST2} are chosen to match the datasets used in the original literature for reproduction purposes. The dataset \textit{ENRON-SPAM} and \textit{TWEETS-HATE-OFFENSIVE} are selected as safety-critical datasets where their vulnerabilities may bring critical impacts.

\section{Starting Point}
% 0.5 page
I have experience implementing machine learning algorithms in Python using Numpy and Pandas. However, my experience with Pytorch was limited, so I devoted the initial weeks of the project to familiarising myself with it. 

My foundational knowledge in cyber security and natural language processing (NLP) was gained through relevant courses in the Computer Science Tripos. However, prompt-based learning and backdoor attacks are new to me. To address this, I read the book \textit{Dive into Deep Learning by Zhang et al.} \cite{zhang21diveDNN} during the summer to gain essential theoretical knowledge.

Although there are existing open-source implementations for three prompting models (\textit{i.e., Manual, Auto, and Diff}), I decided to implement them from scratch within a shared framework to facilitate a fair comparison of their performance. The backdoor attack algorithm had a published implementation \cite{Lei22}, but it only applied to manual prompting with visible backdoor triggers, thus, I extended the algorithm to support flexible backdoor trigger designs and launched the backdoor attacks onto automated discrete and differential prompting models.

\section{Requirements Analysis}
% 1 page
The requirements have been derived from the Success Criteria of the Project Proposal, with additional ones added to provide support for a more flexible and extensible framework. \Cref{tab:requirements} summarised these requirements.
\begin{table}[!ht]
    \centering
    \begin{tabular}{c|c|c}
        \toprule
        \textbf{Main deliverables} & \textbf{Priority} & \textbf{Risk} \\
        \midrule
        Dataset preprocessing modules & \color{red}{\textbf{High}} & \color{mygreen}{\textbf{Low}} \\
        Training \& testing pipelines & \color{red}{\textbf{High}} & \color{mygreen}{\textbf{Low}} \\
        Manual discrete prompting model (Manual) & \color{red}{\textbf{High}} & \color{mygreen}{\textbf{Low}} \\
        Automated discrete prompting model (Auto) & \color{red}{\textbf{High}} & \color{myorange}{\textbf{Medium}} \\ 
        Automated differential prompting model (Diff) & \color{red}{\textbf{High}} & \color{myorange}{\textbf{Medium}} \\
        Visible backdoor attacks onto pre-trained language models (PLM) & \color{red}{\textbf{High}} & \color{myorange}{\textbf{Medium}} \\
        Flexible framework supports additional datasets (*) & \color{myorange}{\textbf{Medium}} & \color{mygreen}{\textbf{Low}} \\
        Flexible framework supports a wider range of K values (*) & \color{myorange}{\textbf{Medium}} & \color{myorange}{\textbf{Medium}} \\
        Visible backdoor attacks with different settings (*) & \color{myorange}{\textbf{Medium}} & \color{red}{\textbf{High}} \\
        Invisible backdoor attacks onto PLMs (*) & \color{mygreen}{\textbf{Low}} & \color{red}{\textbf{High}} \\
        Mask token embedding visualisations (*) & \color{mygreen}{\textbf{Low}} & \color{myorange}{\textbf{Medium}} \\
         \toprule
    \end{tabular}
    \caption{A priority and risk analysis for the main deliverables of the project. Components highlighted with (*) are extensions.}
    \label{tab:requirements}
\end{table}

Deliverables have been prioritised based on their importance in fulfilling the success criteria. Those with a {\color{red}{\textbf{High}}} priority are considered essential features, while those with a {\color{myorange}{\textbf{Medium}}} priority are not strictly necessary but would help support a more generalisable framework. The {\color{mygreen}{\textbf{Low}}} priority deliverables are not necessities but would be desirable features.

Each deliverable is associated with a corresponding risk level highlighting the difficulty of the task. Deliverables with a {\color{myorange}{\textbf{Medium}}} risk level have limited open-source implementation, and careful design choices must be made. Deliverables with a {\color{red}{\textbf{High}}} risk level have no open-source implementation, and more time must be allocated to implement them successfully.

\section{Software Engineering Techniques}
% 1.5 page
\subsection{Development Model}
For project planning and tracking, a Gantt chart is utilised, as demonstrated in \Cref{fig:prepare-diss-gantt-chart}. In addition, adequate slack periods are scheduled to accommodate any potential implementation difficulties.
\begin{figure}[!ht]
    \centering
    \includegraphics[width=\hsize]{figures/preparation_media/diss-gantt-chart.pdf}
    \caption{Project Gantt Chart. Highlighted columns (i.e., week 1 in Michaelmas term, week 3 in Lent term and week 2 in Easter term) are deadlines for project proposal, mid-point report and dissertation, respectively.}
    \label{fig:prepare-diss-gantt-chart}
\end{figure}

\vspace{-1em}
The project adopts the agile development methodology due to its research-oriented nature and the need for an extensible framework. This approach enables iterative and incremental development, where each cycle includes the stages described in \Cref{fig:agile}.
\begin{figure}[!ht]
    \centering
    \includegraphics[width=\hsize]{figures/preparation_media/agile.pdf}
    \caption{Agile development phases}
    \label{fig:agile}
\end{figure}
\vspace{-0.8em}

Firstly, a comprehensive set of project requirements are identified, and the interdependencies between various components are carefully outlined. This information is beneficial in determining the implementation order of the features. Subsequently, during the system design phase, a modular and object-oriented design is adopted. In this process, the API of each module (e.g., \textit{Dataset} and \textit{Dataloader} modules) is documented, the algorithm for each model is thoroughly studied, where pseudo-codes are written for some complex ones. Then during the implementation stage, code comments are added to classes and functions to enhance their readability and maintainability. Additionally, unit tests are carried out to ensure robustness of the modules. Furthermore, to compare the results with existing literature, extensive experiments for performance analysis are conducted, and any discrepancies are studied in detail.  

The agile cycle is repeated for each prompting model and various settings of backdoor attacks, facilitating the iterative addition of new features to the extensible framework. Some findings lead to further exploration, resulting in additional project extensions. 

\subsection{Languages, Libraries, Package manager and Licensing}
Python was chosen as the primary programming language for the project due to its vast array of libraries and tools (e.g., NumPy \cite{Harris20NumPy}, Pandas \cite{mckinney10Pandas} and Matplotlib \cite{Hunter07Matplotlib}) that are highly useful for machine learning (ML) tasks as they provide pre-built functions to help develop ML frameworks with ease. To handle deep neural networks (DNN) like the prompting models, the PyTorch \cite{Paszke19PyTorch} framework was selected for its Pythonic programming style, seamless GPU acceleration and support for dynamic computation graphs. During implementation, PyTorch Lightning \cite{Falcon19PL}, a lightweight PyTorch wrapper, was utilised as it offers a high-level interface for building DNN and allows distributed training with multiple GPUs. 

The Anaconda \cite{anaconda20pack} package manager offers a dependency tracking system and conveniently stores all dependencies in the \texttt{environment.yml} file. This feature allows easy installation of the project environment, enabling researchers to reproduce experimental results with minimal effort. This project has been released under the MIT license \cite{MITLicense}, thereby allowing researchers to make modifications and enhancements to the library to explore further research questions on prompting models. \Cref{tab:library} listed vital third-party libraries used in the project, alongside their functionalities and OSI-approved \cite{OSI} licences.
\begin{table}[!ht]
    \centering
    \begin{tabular}{c|p{8.5cm}|c}
        \toprule
        \textbf{Library} & \textbf{Functionality} & \textbf{Licence} \\
        \midrule
        \texttt{torch} \cite{Paszke19PyTorch} & Develop and train deep neural networks (DNN) & BSD License \\
        \texttt{pytorch\_lightning} \cite{Falcon19PL} & A high-level interface for building DNN models & Apache License \\
        \texttt{torchmetrics} \cite{Detlefsen22torchmetrics} & Metrics for evaluating model performance & MIT License \\
        \texttt{transformers} \cite{Wolf19hugtransf} & Huggingface library for pre-trained NLP models & Apache License \\
        \texttt{datasets} \cite{lhoest21datasets} & Huggingface library storing datasets efficiently & Apache License\\
        \texttt{numpy} \cite{Harris20NumPy} & Manipulate on large, multi-dimensional arrays & BSD License\\
        \texttt{pandas} \cite{mckinney10Pandas} & Flexible tool for data analysis and visualisation & BSD License\\
        \texttt{sklearn} \cite{pedregosa11scikit} & Tools for statistical modelling & BSD License \\
        \texttt{seaborn} \cite{michael17SEABORN} & Create informative statistical graphics & BSD License\\
        \texttt{matplotlib} \cite{Hunter07Matplotlib} & Plot data in statistical graphics & PSF License\\
         \toprule
    \end{tabular}
    \caption{A list of important third-party libraries used in the project.}
    \label{tab:library}
\end{table}

\vspace{-1.2em}
\subsection{Hardware, Version Control and Backup}
I used my personal laptop to write the codes and the dissertation. It is a MacBook Air with 512GB SSD storage and an Apple M1 chip, running macOS Monterey. However, all experiments are run parallelly on 4 NVIDIA Tesla V100 GPUs using the GPU cluster provided by the department.

Version control was managed using Git \cite{wilson06git}, with regular synchronization of the local code repository to a private GitHub remote code repository to prevent data loss. Large binary files, experimental logs, model checkpoints, and test results were stored in Google Drive to ensure additional backup. The dissertation was formatted using LaTeX via the Overleaf platform.

% allocate 18 pages
\chapter{Implementation}
This chapter provides a comprehensive overview of the key components implemented in the project. Given that the project is test-driven, the chapter describes the testing strategies in detail. Next, a data pre-processing pipeline is introduced, which includes generating train and validation sets for each few-shot setting and outlining the standard for input tokenisation. The chapter also covers the implementation details of each prompting model, including the prompt and verbaliser designs, as well as the process of injecting backdoors into the pre-trained language model (PLM). Additionally, the chapter presents a visualisation tool for the embeddings to aid in understanding the differing backdoor attack performance across various prompting models. Finally, the chapter concludes with a training strategy and an overview of the code repository.

\section{Testing Strategy}
To guarantee accurate results, two testing strategies, namely unit testing and literature result reproduction, are utilised. Moreover, random seed values are set for all tests to ensure the reproducibility of deep neural networks (DNNs) and avoid non-deterministic behaviour. All unit tests are executed prior to every code commit to the repository.

\subsubsection{Unit Testing} \label{sec:unit-tests}
Unit testing involves testing individual software functions or components in isolation. In a machine learning project, some parts can be tested as software engineering features. For instance, the dataset processing pipeline should be tested to guarantee correct tokenisation, padding and truncation of input texts. Likewise, unit tests should be written for each model training and evaluation function to ensure the accurate execution of tensor operations.

This project utilises the PyTest framework \cite{pytest2004} for unit testing to detect code issues prior to system integration. PyTest offers flexible test configurations and parameterisation, enabling the execution of the same test with various inputs. 

However, testing DNNs solely with unit testing presents significant challenges. DNNs usually handle high-dimensional inputs, such as word embeddings, which renders it impractical to cover every input combination. Additionally, the presence of multiple layers and nonlinear activation functions in DNNs complicates the unit testing to evaluate network behaviour accurately.

\subsubsection{Reproduction of Literature Results}
To overcome unit testing limitations and verify project credibility, experiments are conducted to reproduce previous literature results. The outcomes are discussed in \Cref{sec:evaluation} and \Cref{sec:reprod_lit_res}.

\section{Dataset Preprocessing}\label{sec:dataset}
\subsection{Download, Generate K-shot and Caching Datasets} \label{sec:dataset-1}
In order to create a scalable and adaptable framework, it is essential to support widely-used NLP datasets. The Huggingface \texttt{datasets} library offers a comprehensive collection of more than 26500 commonly used datasets for diverse machine learning tasks. Additionally, it simplifies the process of uploading new datasets.

This project focused on two downstream tasks, textual entailment and sentiment analysis, and has selected three datasets for each task. The datasets are first downloaded from the Huggingface \texttt{datasets} library and stored in the Apache Arrow format (AAF) \cite{arrow23}. In contrast to conventional data storage formats such as comma-separated values (CSV), AAF offers superior efficiency in storing and processing data. This format holds data in a columnar structure to preserve data locality and is optimised for single instruction, multiple data (SIMD) processing, leading to significant performance enhancements when executing operations on a GPU. 

For few-shot learning scenarios, each training and validation sets consist of $K$ samples per class, and a random seed is used to shuffle the dataset before sampling. This project initially constructed datasets for $K \in \{16, 100, 1000\}$ and later extended to $K \in \{8, 16, 32, 64, 100, 1000\}$.

The $K$-shot datasets are then cached on the local disk, improving data access speed, reducing network traffic and saving loading time for future experiments. 

\subsection{Data Loading Pipeline} \label{sec:dataset-2}
As shown in \Cref{fig:impl-datasets}, to facilitate experimentation with diverse datasets and enable easy switching between them, a hierarchical class inheritance structure is established. Since datasets for the same downstream task share common structures in their inputs (e.g., textual entailment datasets have a pair of input texts with a numeric label), a concrete class inheriting from the abstract class \texttt{Dataset} has been created for each downstream task, namely \texttt{TextEntailDataset} and \texttt{SentAnalDataset}. These concrete classes implemented the logic to handle input tokenisation. Furthermore, an additional subclass for each specific dataset is constructed (e.g., \texttt{TextEntailDatasetQNLI} inherits from \texttt{TextEntailDataset}).

\begin{figure}[!ht]
    \centering
    \includegraphics[width=\hsize]{figures/implementation_media/impl-datasets.pdf}
    \caption{A simplified UML diagram for the \texttt{dataset.py} which holds dataset-related classes organised in an hierarchical structure. Subclasses of the abstract \texttt{Dataset} class must override the \texttt{\_\_getitem\_\_} and \texttt{\_\_len\_\_} methods.}
    \label{fig:impl-datasets}
\end{figure}

In fine-tuning, input tokenisation is applied after concatenating input texts. However, in prompt-based learning, input texts require pre-processing in a prompt or template before tokenisation can be applied. As a result, an additional concrete class has been implemented for each downstream task, such as \texttt{TextEntailDatasetPrompt}, which inherits from \texttt{TextEntailDataset}. Each dataset then further inherits from the corresponding concrete class to construct a dataset-specific subclass (e.g., \texttt{TextEntailDatasetPromptQNLI} inherits from \texttt{TextEntailDatasetPrompt}). Overall, this hierarchical inheritance structure provides seamless support for adding new downstream tasks and datasets.

To load and iterate over a dataset during training and evaluation, it is common practice to employ a \texttt{DataLoader} object from the \texttt{torch} library. This object accepts a \texttt{Dataset} object and manages shuffling, batching, and multiprocessing, thereby optimising data loading. Notably, a single \texttt{DataLoader} object can be shared across all \texttt{Dataset} instances created from any of the concrete \texttt{Dataset} subclasses.

\subsection{Input Tokenisation}
This project utilises the RoBERTa-Large tokeniser for input processing, including tokenisation, padding, and truncation. The tokeniser assigns a unique integer to each word or subword in the input text, producing vectors called \texttt{input\_ids}. To handle input batches efficiently, the \texttt{input\_ids} are padded or truncated to a fixed length and a binary representation, \texttt{attention\_masks}, of equal length is added. Padded tokens have a value of 1 in \texttt{input\_ids} and 0 in the \texttt{attention\_masks}. For instance, assume a fixed length of 9 tokens, an input text batch \texttt{"Hello, world."} can result in a numeric vector \texttt{input\_ids = [0, 31414, 6, 8331, 4, 2, 1,1,1]} and a binary vector \texttt{attention\_masks = [1,1,1,1,1,1,0,0,0]}.

Prompt-based learning requires putting the input text into a prompt or template before tokenisation, some examples of prompt are shown in \Cref{tab:tokens}. The prompt includes placeholders for input text, discrete words and special tokens.

\begin{table}[!ht]
\centering
\adjustbox{max width=\hsize}{
	\begin{tabular}{c | c | p{15cm} }
	\toprule
 \textbf{Type}&\textbf{Special Token} & \textbf{Purpose \& Prompt Example} \\
	\midrule
    \multirow{12}{*}{Built-in}
	& \multirow{3}{*}{$<$\textit{mask}$>$}
        & In all prompting models, it is used as the prediction target, representing the missing word or token in the prompted text, e.g., \newline \texttt{<input> . It was <mask> .}\\
        \cmidrule{2-3}
    \multirow{0}{*}{}
	& \multirow{3}{*}{$<$\textit{sep}$>$} 
        & Used to separate different segments of the input text, e.g., \newline \texttt{<input> . It was <mask> .} \newline $\xrightarrow{\text{parse into}}$ \texttt{<input><sep>.<sep>It<sep>was<sep><mask><sep>.} \\
        \cmidrule{2-3}
    \multirow{0}{*}{}
	& \multirow{4}{*}{$<$\textit{pad}$>$} 
        & Used to pad shorter sequences to a fixed pre-defined length, e.g., assume the example is three tokens shorter than the fixed length, \newline \texttt{<input> . It was <mask> .}
        \newline $\xrightarrow{\text{parse into}}$ \texttt{<input> . It was <mask> .<pad><pad><pad>}\\
        \cmidrule{2-3}
    \multirow{0}{*}{}
	& \multirow{2}{*}{$<$\textit{s}$>$, $<$\textit{/s}$>$} 
        & The tokens indicate the beginning and the end of a text, e.g., \newline \texttt{<input> . It was <mask> .}
         $\xrightarrow{\text{parse into}}$ \texttt{<s><input> . It was <mask> .</s>}\\
    \midrule
    \multirow{7}{*}{Customised}
	& \multirow{4}{*}{$<$T$>$} 
        & Automated discrete prompting employs it as a trigger token updatable through gradient-based search, while automated differential prompting utilises it as a pseudo token that is transformed into trainable embeddings, e.g., \newline \texttt{<input> <T> <T> <T> <mask> .} \\
    \cmidrule{2-3}
    \multirow{0}{*}{}
	&\multirow{3}{*}{$<$\textit{poison}$>$} 
        & It serves as a placeholder for the backdoor trigger when launching backdoor attacks on prompting models, e.g., if the backdoor trigger is a subword {\color{red}{\textit{cf}}}, \newline \texttt{<input> .<poison> It was <mask> .}  $\xrightarrow{\text{parse into}}$ \texttt{<input> .{\color{red}{cf}} It was <mask> .}  \\
	\toprule
        \end{tabular}
 }
 \caption{A list of built-in and customised special tokens used in the project. Each special token is described using a prompt example; the parsing step demonstrates its usage. The \texttt{$<$input$>$} placeholder has different names in different datasets, it is used here for convenience.}
 \label{tab:tokens}
\end{table}

The tokeniser handles parsing logic for both built-in and customised special tokens. The built-in $<$\textit{mask}$>$ token is used as the missing target in the cloze-completion problem, where the most likely word or token will be filled in by the PLM. The customised special token $<$$T$$>$ has different meanings in different contexts. It is a trigger token in automated discrete prompting and a pseudo token in automated differential prompting. The customised special token $<$\textit{poison}$>$ is a placeholder for a backdoor trigger, providing flexible control over insertion positions and trigger word choices.
\vspace{-1em}
\section{Prompting Functions and Verbalisers} \label{sec:prompting-models}
\vspace{-0.5em}
This project includes three prompting models: manual discrete (Manual), automated discrete (Auto), and automated differential (Diff), each implemented by a class, as shown in \Cref{fig:impl-uml}.
\vspace{-1em}
\begin{figure}[!ht]
    \centering
    \includegraphics[width=\hsize]{figures/implementation_media/impl-uml.pdf}
    \caption{A UML diagram, the \texttt{Classifier} class inherits from the abstract \texttt{LightningModule} class, and incorporates common attributes and methods to facilitate fine-tuning. Additionally, each prompting model is represented by a concrete class, which inherits from the \texttt{Classifier} base class and introduces further attributes, methods, and composed objects.}
    \label{fig:impl-uml}
\end{figure}

In the UML diagram, the \texttt{Classifier} class implements common attributes such as \texttt{visual\_tool} and methods like \texttt{configure\_optimizers} shared among all models. Concrete methods such as \texttt{forward} and \texttt{training\_step} are implemented in a way that facilitates fine-tuning, with further details provided in \Cref{sec:appendix-finetune}. The classes that implement Manual, Auto and Diff all inherit from the \texttt{Classifier} class to avoid redundant coding of common attributes and methods. Specific algorithms for prompting models are written using additional composed objects such as \texttt{AutoTokenizer} and \texttt{AutoConfig}.

\subsection{Implement Manual Discrete prompting (Manual)} \label{sec:manual-prompt}
The manual prompts and verbalisers used in this project were adapted from the Public Pool of Prompts \cite{Bach22OPP} and previous work on prompting \cite{Gao20PM, Lei22}. For example, when working with the multi-class textual entailment dataset \textit{MNLI-MATCHED} where each input sample is a premise-hypothesis pair, one simple but effective manual prompt could be \texttt{"<premise> ? <mask> , <hypothesis> ."} and a verbaliser that maps from the answer domain $\mathcal{Z}$ to the label domain $\mathcal{Y}$ could be \texttt{\{Yes $\mapsto$ 0(Entailment), Maybe $\mapsto$ 1(Neutral), No $\mapsto$ 2(Contradiction)\}}. 

\Cref{alg:manual-forward} outlines the core procedure in manual prompting. During tokenisation, a batch of prompted text $\textbf{X}'$ was converted into vectors \texttt{input\_ids} and \texttt{attention\_masks}, each has a fixed shape $(\texttt{batch\_size}, \texttt{max\_seq\_len})$ where $\texttt{max\_seq\_len}$ is the maximum number of tokens in each input text. These are fed into the modeling head $m$ of the RoBERTa-Large model, which produces a batch of output word embeddings $\mathbf{O} = [O_1, ..., O_{\texttt{batch\_size}}]$. For each output embedding $O_i$, The $<$\textit{mask}$>$ token embedding is denoted as $o_{\textit{mask}}^{(i)} \in \mathbb{R}^{|\mathcal{V}|}$ where $|\mathcal{V}|$ is the vocabulary size. Each value in $o_{\textit{mask}}^{(i)}$ represents the relevance score for the corresponding token. 

To determine the most likely output labels $\hat{\textbf{y}}$, we sum up the scores of tokens $z \in \mathcal{V}_{y'}$ for each label $y' \in \mathcal{Y}$ where $\mathcal{V}_{y'}$ is the set of words in the answer domain $\mathcal{Z}$ that map to label $y'$. Then apply a softmax layer to convert scores into probabilities. We compute the cross-entropy loss $\mathcal{L}_C$ between predicted labels $\hat{\textbf{y}}$ and correct labels $\textbf{y}$ to measure classification performance. 

\begin{algorithm}
\caption{Manual prompting Forward Function}\label{alg:manual-forward}
\begin{algorithmic}[1]
\small
\Require $\boldsymbol{:}$ \newline $m = \text{the pre-trained RoBERTa-Large model with a modeling head}$ \newline $\mathcal{Z} = \text{the answer domain of the vocabulary}$
\Ensure $\boldsymbol{:}$ \newline $\text{input\_ids} = \text{the numeric format of the input text batch }\mathbf{X}'$ \newline
    $\text{attention\_masks} = \text{the binary format of the input text batch }\mathbf{X}' $ \newline
    $\mathbf{y} = \text{correct class labels of the input text batch }\mathbf{X}'$ \newline
    $\text{mask\_pos} = \text{positions of the mask token in the input texts}$
\vspace{0.3em}
\hrule
\vspace{0.3em}
\Function{manual\_forward}{\text{input\_ids}, \text{attention\_masks}, $\mathbf{y}$, \text{mask\_pos}}
\State $m_\text{out} = m.\Call{\text{forward}}{\text{input\_ids}, \text{attention\_masks}}$
\State $\textbf{O} \gets \text{get output word embeddings from $m_\text{out}$}$  
 {\color{mylightgrey}\Comment{\textit{embeddings before the classifier layer}}}
 \State $\textbf{o}_{<\textit{mask}>} \gets \text{get $<\textit{mask}>$ token output word embeddings from $\textbf{O}$}$
 \newline {\color{mylightgrey}\Comment{\textit{$\textbf{O}$.shape: (batch\_size, max\_seq\_len, $|\mathcal{V}|$); $\textbf{o}_{<\textit{mask}>}$.shape: (batch\_size, 1, $|\mathcal{V}|$)}}}
\State $s_{<\textit{mask}>} \gets \text{get scores for each $z \in \mathcal{V}_{y'}$ for each class $y' \in \mathcal{Y}$}$ {\color{mylightgrey}\Comment{\textit{$s_{<\textit{mask}>}$.shape: $(|\mathcal{Y}|, |\mathcal{Z}| / |\mathcal{Y}|)$}}}
\State ${sum\_s}_{<\textit{mask}>} \gets \text{get sum of $s_{<\textit{mask}>}$ for each class $y' \in \mathcal{Y}$}$ {\color{mylightgrey}\Comment{\textit{${sum\_s}_{<\textit{mask}>}$.shape: $(|\mathcal{Y}|, 1)$}}}
\State $\Pr_{\mathcal{Y}} \gets \text{softmax(${sum\_s}_{<\textit{mask}>}$)}$
{\color{mylightgrey}\Comment{\textit{compute the probability of each class label}}}
\State $\hat{\mathbf{y}} \gets \argmax_{y \in \mathcal{Y}} \Pr_{\mathcal{Y}}$
{\color{mylightgrey}\Comment{\textit{get the class label with highest likelihood in $\Pr_{\mathcal{Y}}$}}}
\State $\mathcal{L}_C \gets \text{cross-entropy}(\hat{\mathbf{y}},\mathbf{y})$
{\color{mylightgrey}\Comment{\textit{compute the loss to measure classification performance}}}
\State \textbf{return $\mathcal{L}_C, \hat{\mathbf{y}}$}
{\color{mylightgrey}\Comment{\textit{return the loss and the predicted label}}}
\EndFunction
\end{algorithmic}
\end{algorithm}

\subsection{Implement Automated Discrete prompting (Auto)}
\subsubsection{Auto prompting Function} \label{sec:auto-prompt}
Instead of discrete words, the auto prompting function utilises a set of trigger tokens that will be updated via a gradient-based search. After each training epoch, the function randomly selects a trigger token and replaces it by a candidate token $\hat{v}$ that gives the maximum improvement in the cumulative log-likelihood $\log \Pr(\mathbf{y} | \mathbf{X}'; \theta)$, defined in \Cref{eqn:cum_loglik}. Here, $\mathbf{y}$ contains the labels for prompted texts $\mathbf{X}'$ and $\Pr(\cdot|\theta)$ represents the PLM with pre-defined parameters $\theta$.

However, due to the large vocabulary size (e.g., 50265 tokens for RoBERTa-large tokeniser), computing the change in $\log \Pr(\mathbf{y} | \mathbf{X}'; \theta)$ for every token $v \in \mathcal{V}$ is impractical. Instead, auto prompting applies the \emph{HotFlip} \cite{Ebrahimi17HotFlip} method which estimates the change in $\ \log \Pr(\mathbf{y} | \mathbf{X}'; \theta)$ for each token $v \in \mathcal{V}$ and forms a set of top-$n$ performing tokens $\mathcal{V}_{\text{cand}} \subseteq \mathcal{V}$ heuristically based on their ability to improve the cumulative log-likelihood $\log \Pr(\mathbf{y} | \mathbf{X}'; \theta)$, then iterate through each $v \in \mathcal{V}_{\text{cand}}$ to select the top performing one.

The \emph{HotFlip} method is based on the first-order Taylor approximation. Given a function $f: \mathbb{R}^d \to \mathbb{R}$ that is differentiable at $\lambda \in \mathbb{R}^d$, its first-order Taylor approximation and its change $\Delta f$ can be written as:
\begin{equation}
\begin{split}
    f(\lambda + \Delta \lambda) & \approx f(\lambda) + \Delta \lambda^T \nabla f|_{\lambda} \\
    \Delta f & = f(\lambda + \Delta \lambda) - f(\lambda) = \Delta \lambda^T \nabla f|_{\lambda}
\end{split}
\end{equation} 
Let $\Delta f$ be the change in the cumulative log-likelihood function $\log \Pr(\mathbf{y} | \mathbf{X}'; \theta)$, the only part that has been modified after a token replacement is the input word embedding layer $\textbf{E}$. Thus, the \emph{top-n} candidate token set $\mathcal{V}_\text{cand}$ is defined as:
\begin{equation}
    \mathcal{V}_{\text{cand}} = {\text{top-}n}_{v\in \mathcal{V}} [\mathbf{E}^T \nabla \log \Pr(\mathbf{y} | \mathbf{X}'; \theta)|_{T}]
\end{equation}
where $\nabla \log \Pr(\mathbf{y} | \mathbf{X}'; \theta)|_T$ is the gradient with respect to the input embedding of the selected trigger token $T$.

\begin{figure}[!ht]
    \centering
    \includegraphics[width=\hsize]{figures/implementation_media/impl-auto-prompting.pdf}
    \caption{The \textit{HotFlip} method generates the candidate set $\mathcal{V}_{\text{cand}}$. The token $\hat{v} \in \mathcal{V}_{\text{cand}}$ that maximizes the improvement in cumulative log-likelihood is chosen to update the randomly selected trigger token.}
    \label{fig:impl-auto-prompt}
\end{figure}

For each candidate token $v \in \mathcal{V}_{\text{cand}}$, the model evaluates the accuracy of the entire training dataset using the adjusted prompted text. The highest-performing candidate token $\hat{v}$ is selected to update the trigger token:
\begin{equation}
    \hat{v} = \argmax_{v \in \mathcal{V}_{\text{cand}}} \Delta \log \Pr(\textbf{y} | f_{\text{sub}}(\textbf{X}', v); \theta)
\end{equation}
where the function $f_{\text{sub}}(\textbf{X}', v)$ substitutes the candidate token $v$ inside the randomly chosen trigger token in the prompted text $\textbf{X}'$. 

\begin{comment}
\begin{algorithm}
\Class{GradientOnBackwardHook}

\Procedure{GradientOnBackwardHook}{module}
\State $\nabla f \gets \text{None}$
{\color{mylightgrey}\Comment{\textit{local variable for tracking the gradient $\nabla \log \Pr(\textbf{y}|\textbf{X}';\theta)$}}}
\State $\text{module}.\Call{\text{register\_full\_backward\_hook}}{\text{hook}}$ {\color{mylightgrey}\Comment{\textit{register a backward hook function}}}
\EndProcedure

\Procedure{hook}{\text{module}, \text{grad\_in}, \text{grad\_out}}
  \State $\nabla f \gets \text{grad\_out}$ {\color{mylightgrey}\Comment{\textit{called on every backpropagation pass, store newest $\nabla \log \Pr(\textbf{y}|\textbf{X}';\theta)$}}}
\EndProcedure
\Procedure{get}{}
  \State $\textbf{return } \nabla f$ {\color{mylightgrey}\Comment{\textit{fetch newest $\nabla \log \Pr(\textbf{y}|\textbf{X}';\theta)$}}}
\EndProcedure
\EndClass
\end{algorithm}
\end{comment}

The implementation of the gradient-based search method is detailed in \Cref{alg:auto-hotflip}. To track the gradient $\nabla \log \Pr(\textbf{y} | \textbf{X}'; \theta)$, a PyTorch backward hook function is registered on the input word embeddings $\textbf{E}$ using the \texttt{register\_backward\_hook} method in the \texttt{GradientOnBackwardHook} class. This function is called on every backpropagation pass via the PyTorch Lightning callback hook \texttt{on\_after\_backward}, allowing accumulation of gradients of input word embeddings $\textbf{E}$ for all batches of input samples. Additionally, the \textit{HotFlip} method is applied at the end of every training epoch, called by the PyTorch Lightning callback hook \texttt{on\_train\_epoch\_end}.

\begin{algorithm}
\caption{Auto prompting Gradient-based Search Method} \label{alg:auto-hotflip}
\begin{algorithmic}[1]
\small
\Require $\boldsymbol{:}$ 
\newline $n = \text{\# candidate tokens in $\mathcal{V}_{\text{cand}}$}$ 
\newline $\textbf{T} = \text{the set of trigger tokens}$
\newline $\textbf{E} = \text{the input word embeddings}$
\newline $\text{train\_dataloader} = \text{dataloader for the train dataset}$
\newline $\text{GradientOnBackwardHook} = \text{a class that provides a handle for a backward hook}$
\vspace{0.3em}
\hrule
\vspace{0.3em}

\State $\textbf{E}_\text{grad} \gets \text{GradientOnBackwardHook}(\textbf{E})$
{\color{mylightgrey}\Comment{\textit{register a backward hook on the word embeddings $\textbf{E}$}}}
\State $\nabla f|_{\textbf{T}} \gets \text{zero matrix}$
{\color{mylightgrey}\Comment{\textit{$\nabla f|_{\textbf{T}}$.shape (batch\_size, max\_seq\_len, hidden\_size)}}}

\Function{on\_after\_backward}{}
\State $\nabla f \gets \textbf{E}_\text{grad}.\Call{\text{get}}{}$
{\color{mylightgrey}\Comment{\textit{fetch $\nabla \log \Pr(\textbf{y}|\textbf{X}';\theta)$ for the current batch}}}
\State $\nabla f|_{\textbf{T}} \gets \text{$\nabla f|_{\textbf{T}} + \nabla f$}$
{\color{mylightgrey}\Comment{\textit{accumulate gradients for the trigger set $\textbf{T}$ for all batches}}}
\EndFunction

\Function{on\_train\_epoch\_end}{}
\State $T_i \gets \text{random trigger token from $\textbf{T}$}$
\State $\nabla f|_{T_i} \gets \text{$\nabla f|_{\textbf{T}}$ at $T_i$}$
{\color{mylightgrey}\Comment{\textit{extract cumulative gradients at the selected trigger token}}}
\State $\text{ids} \gets {\text{top-}n} [\mathbf{E}^T \nabla f|_{T_i}]$
{\color{mylightgrey}\Comment{\textit{apply HotFlip, get the top $n$ indices from matrix product $[\mathbf{E}^T \nabla f|_{T_i}]$}}}
\State $s_{\text{curr}} \gets 0$
{\color{mylightgrey}\Comment{\textit{track the score of the current set of trigger tokens \textbf{T}}}}
\State $s_{\text{cand}} \gets \{\}$
{\color{mylightgrey}\Comment{\textit{track the scores of  the set of candidate tokens with indices ids}}}
\For{$\text{batch \textbf{in} train\_dataloader}$}
    \State $\text{input\_ids}, \text{attention\_masks} \gets \text{get input text in numeric and binary formats from batch}$
    \State $\text{input\_ids}_{\textbf{T}} \gets \text{update input\_ids with current trigger token set \textbf{T}}$
    \State $\textbf{y} \gets \text{get correct class labels from batch}$
    \State $\text{mask\_pos} \gets \text{get mask token positions from batch}$
    
    \State $\mathcal{L}_C, \hat{\textbf{y}} \gets \Call{\text{auto\_forward}}{\text{input\_ids}_{\textbf{T}} , \text{attention\_masks}, \textbf{y}, \text{mask\_pos}}$
    \newline {\color{mylightgrey}\Comment{\textit{forward pass to compute the cross-entropy loss $\mathcal{L}_C$ and predicted labels $\hat{\textbf{y}}$}}}
    
    \State $s_{\text{curr}} \gets s_{\text{curr}} + \Call{score}{\hat{\textbf{y}}, \textbf{y}}$
    {\color{mylightgrey}\Comment{\textit{accumulate score when using the current set \textbf{T}}}}
    
    \For{$\text{$i$ \textbf{in} ids}$} 
    {\color{mylightgrey}\Comment{\textit{iterative the indices of the candidate set $\mathcal{V}_{\text{cand}}$}}}
        \State $v \gets \text{get token at index $i$ in $\mathcal{V}$}$
        {\color{mylightgrey}\Comment{\textit{fetch the $i^{\text{th}}$ token in $\mathcal{V}$}}}
        \State $\text{input\_ids}_{i} \gets \text{update $\text{input\_ids}_{\textbf{T}}$ by substitution $f_\text{sub}(\textbf{X}', v)$}$
        \State $\mathcal{L}_C, \hat{\textbf{y}} \gets \Call{\text{auto\_forward}}{\text{input\_ids}_{i}, \text{attention\_masks}, \textbf{y}, \text{mask\_pos}}$
        \newline {\color{mylightgrey}\Comment{\textit{forward pass to compute $\mathcal{L}_C$ and $\hat{\textbf{y}}$ with the trigger token $T_i$ been replaced by $v$}}}
        \State $s_{\text{cand}}[i] \gets s_{\text{cand}}[i] + \Call{score}{\hat{\textbf{y}}, \textbf{y}}$
        {\color{mylightgrey}\Comment{\textit{accumulate score for $v$}}}
    \EndFor
    \State $\hat{s}_{\text{cand}} \gets \argmax_{i} s_{\text{cand}}$ {\color{mylightgrey}\Comment{\textit{get the best performing candidate token $\hat{v}$}}}
    \If{$\hat{s}_{\text{cand}} > s_\text{curr}$}
        \State $\textbf{T} \gets \textbf{T}[v/T_i]$
        {\color{mylightgrey}\Comment{\textit{update the trigger token $T_i$ with $v$}}}
    \EndIf
\EndFor
\EndFunction
\end{algorithmic}
\end{algorithm}

Using this heuristic method \textit{HotFlip}, the time complexity is vastly decreased. Assuming $b$ batches of samples in the train and validation dataset, evaluating the change in log-likelihood $\log\Pr(\textbf{y}| \textbf{X}' ; \theta)$ for every candidate token $v \in \mathcal{V}$ requires $|\mathcal{V}|b$ forward steps. With \textit{HotFlip}, accumulating gradients of the input word embedding layer require $b$ forward pass and $b$ backward pass, after computing the top-$n$ candidate set $\mathcal{V}_{\text{cand}}$, iterating through all candidate tokens $v \in \mathcal{V}_{\text{cand}}$ requires only $nb$ forward steps. If $n$ is much smaller than $|\mathcal{V}|$, then \textit{HotFlip} requires significantly less time, $(n+2)b$, compared to $|\mathcal{V}|b$.

\subsubsection{Auto Prompting Verbaliser} \label{sec:auto-verb}
In addition to the gradient-based prompt search, the framework includes a label search procedure to construct a verbaliser, which determines the answer domain $\mathcal{V}_{y} \subseteq \mathcal{Z}$ for each label $y \in \mathcal{Y}$. Given a set of prompted samples $\mathbf{X}'$ which all have the same label $y$, the process selects words that are the most contextually relevant when filling into the $<$\textit{mask}$>$ tokens in $\mathbf{X}'$. Label search is necessary because the suitable answer domain $\mathcal{Z}$ varies based on the prompt, and the exact values of the trigger tokens are not known before model training. 

\begin{figure}[!ht]
    \centering
    \includegraphics[width=\hsize]{figures/preparation_media/prepare-auto-verb.pdf}
    \caption{Verbaliser design in auto prompting. The contextualised word embedding $c_{<\textit{mask}>}$ are fed into a logistic classifier to tune the weights $\boldsymbol{\beta}$ and biases $\beta_0$, which is then used to define a score for each word $v \in \mathcal{V}$, identifying the most contextually relevant words for each class.}
    \label{fig:prepare-auto-verb}
\end{figure}

\Cref{fig:prepare-auto-verb} shows the label search method.  The contextualised word embedding for the prompted text $X'$ is $C = [c_1, ..., c_{\texttt{max\_seq\_len}}]$ where $c_{<\textit{mask}>}$ is the $<$\textit{mask}$>$ token embedding. We use a two-step process to score the contextual relevance of candidate words. Firstly, the $c_{<\textit{mask}>}$ of each training sample is fed into a logistic classifier which predicts the most-likely label $\hat{y}$ for the prompted text $X'$:
\begin{equation}
\begin{split}
    \hat{y} = \argmax_{y' \in \mathcal{Y}} \Pr(y'|c_{<\textit{mask}>}) 
     & = \argmax_{y' \in \mathcal{Y}} \exp(\boldsymbol{\beta}^{(y')} \cdot  c_{<\textit{mask}>} + \beta_0^{(y')}) \\
    & = \sigma(\boldsymbol{\beta} \cdot c_{<\textit{mask}>} + \beta_0)
\end{split}
\end{equation}

where $\sigma$ is the activation function applying a softmax transformation; $\boldsymbol{\beta}^{(y')}$ and $\beta_0^{(y')}$ are weights and bias for label $y' \in  \mathcal{Y}$, optimised to minimise the multi-class cross-entropy loss $\mathcal{L}(\hat{\mathbf{y}}, \mathbf{y})$. 

The weights $\boldsymbol{\beta}^{(y)}$  and biases $\beta_0^{(y)}$ indicates the contribution of each node in the logistic classifier input layer to the label $y \in \mathcal{Y}$. Hence, we can define a score $s(y,v)$ for each word $v \in \mathcal{V}$: 
\begin{equation}
    s(y, v) = \Pr(y|c_v) \propto (\boldsymbol{\beta}^{(y)} \cdot c_v  + \beta_0^{(y)})
\end{equation}
where $c_v$ is the contextualised word embedding of the token $v \in \mathcal{V}$, defined as $c_v = \text{Transformer}_{\text{encoder}}(e(v))$. Based on the assumption that a token $v$ that is highly associated with label $y$ has a large $s(y,v)$, the top $n'$ highest-scoring words form the set $\mathcal{V}_y$:
\begin{equation}
    \mathcal{V}_y = {\text{top-}n'}_{v \in \mathcal{V}}[s(y,v)]
\end{equation}

\Cref{alg:auto-label} gives the implementation details of the label search method. A PyTorch forward hook is registered on the contextualised word embeddings $\textbf{C} = [C_1, ..., C_{\texttt{batch\_size}}]$, on every forward pass, $\textbf{C}$ will be updated and fetched. The $<$\textit{mask}$>$ token output word embedding $\textbf{c}_{<\textit{mask}>} = [c_{<\textit{mask}>}^{(1)}, ..., c_{<\textit{mask}>}^{(\texttt{batch\_size})}]$ is fed into the logistic classifier $m'$ to minimise the cross entropy loss $\mathcal{L}_C$ by tuning the weights $\boldsymbol{\beta}$ and biases $\beta_0$. 

After each training epoch, the tuned weights $\boldsymbol{\beta}^{(y)}$ and biases $\beta_0^{(y)}$ capture the contribution of each node in the logistic classifier to the label $y \in \mathcal{Y}$. The PyTorch Lightning hook function \texttt{on\_train\_epoch\_end} is then called to construct the top-$n'$ candidate set $\mathcal{V}_{y'}$ for each label $y' \in \mathcal{Y}$ using the tuned weights and biases.

\begin{comment}
\Class{OutputOnForwardHook}

\Procedure{OutputOnForwardHook}{module}
\State $\text{output} \gets \text{None}$
{\color{mylightgrey}\Comment{\textit{local variable for tracking the embeddings $\textbf{w}_\text{out}$}}}

\State $\text{module}.\Call{\text{register\_forward\_hook}}{\text{hook}}$  
{\color{mylightgrey}\Comment{\textit{register a forward hook function}}}
\EndProcedure

\Procedure{hook}{\text{module}, \text{input}, \text{output}}
  \State $\text{output} \gets \text{output}$ 
  {\color{mylightgrey}\Comment{\textit{called on every forward pass, store newest embeddings $\textbf{w}_\text{out}$}}}
\EndProcedure
\Procedure{get}{}
  \State $\textbf{return } \text{output}$ 
  {\color{mylightgrey}\Comment{\textit{fetch newest embeddings $\textbf{w}_\text{out}$}}}
\EndProcedure
\EndClass
\end{comment}

\begin{algorithm}
\caption{Auto prompting Label Search Method} \label{alg:auto-label}
\begin{algorithmic}[1]
\small
\Require $\boldsymbol{:}$ 
\newline $n' = \text{\# candidate tokens in $\mathcal{V}_{\text{y}'}$ for each $y' \in \mathcal{Y}$}$ 
\newline $\textbf{C} = \text{the contextualised word embedding}$
\newline $m = \text{the pre-trained RoBERTa-Large model with a modeling head}$
\newline $m' = \text{the logistic classifier model}$
\newline $\text{OutputOnForwardHook} = \text{a class that provides a handle for a forward hook}$
\vspace{0.3em}
\hrule
\vspace{0.3em}

\State $\text{hook} \gets \text{OutputOnForwardHook}(\textbf{C})$
{\color{mylightgrey}\Comment{\textit{register a forward hook on contextualised embeddings $\textbf{C}$}}}
\Function{label\_search\_forward}{\text{input\_ids}, \text{attention\_masks}, \textbf{y}, \text{mask\_pos}}
    \State $m.\Call{forward}{\text{input\_ids}, \text{attention\_masks}}$
    {\color{mylightgrey}\Comment{\textit{forward pass on the PLM}}}
    \State $\textbf{C} \gets \text{hook}.\Call{get}{}$
    {\color{mylightgrey}\Comment{\textit{fetch embeddings $\textbf{C}$ on the current forward pass}}}
    \State $\textbf{c}_\text{$<$$\textit{mask}$$>$} \gets \text{get $<$$\textit{mask}$$>$ token output word embedding from $\textbf{C}$}$
    \newline {\color{mylightgrey}\Comment{\textit{$\textbf{C}$.shape: (batch\_size, max\_seq\_len, $|\mathcal{V}|$), $\textbf{c}_\text{$<$$\textit{mask}$$>$}$.shape: (batch\_size, 1, $|\mathcal{V}|$)}}}
    \State $\hat{\textbf{y}} \gets m'.\Call{forward}{\textbf{c}_\text{$<$$\textit{mask}$$>$}}$
    {\color{mylightgrey}\Comment{\textit{forward pass on logistic classifier $m'$}}}
    \State $\mathcal{L}_C \gets \text{cross-entropy}(\hat{\textbf{y}}, \textbf{y})$
    {\color{mylightgrey}\Comment{\textit{compute cross-entropy loss $\mathcal{L}_C$}}}
    \State $\textbf{return } \mathcal{L}_C, \hat{\textbf{y}}$
    {\color{mylightgrey}\Comment{\textit{return the loss and the predicted label}}}
\EndFunction

\Function{on\_train\_epoch\_end}{}
    \State $\boldsymbol{\beta}, \beta_0 \gets \text{weights and biases from $m'$}$
    {\color{mylightgrey}\Comment{\textit{fetch the tuned weights and biases}}}
    \State $c_\mathcal{V} \gets \text{get $\text{Transformer}_{\text{Encoder}}(e(v))$ for $v \in \mathcal{V}$}$
    {\color{mylightgrey}\Comment{\textit{get contextualised embeddings for all tokens}}}
    \For{$y' \in \mathcal{Y}$}
        \State $\mathcal{V}_{y'} \gets \text{top-}n'_{v \in \mathcal{V}}[\boldsymbol{\beta}^{(y')} \cdot c_\mathcal{V} + \beta_0^{(y')}]$
        {\color{mylightgrey}\Comment{\textit{construct the answer domain $\mathcal{V}_{y'}$ for each label $y' \in \mathcal{Y}$}}}    
    \EndFor
    
\EndFunction
\end{algorithmic}
\end{algorithm}
\vspace{-1.0em}
\subsection{Implement Automated Differential prompting (Diff)} \label{sec:diff-prompt}
Differential prompting designs a prompt with $m$ pseudo tokens $T_{0:m}$ that can be converted to $m$ trainable embeddings $h_{0:m}$ in a continuous space. Additionally, differential prompting establishes a one-to-one mapping $h_{m+i+1} \mapsto y_i$ from an embedding $h_{m+i+1} \in \mathbb{R}^{d_e}$ in the continuous space with dimension $d_e$ to a class label $y_i \in \mathcal{Y}$, and jointly optimise both the prompt and the verbaliser embeddings $h_{0:m+|\mathcal{Y}|}$.

\begin{figure}[!ht]
    \centering
    \includegraphics[width=\hsize]{figures/implementation_media/impl-diff-lc.pdf}
    \caption{The class discrimination object in differential prompting. The trainable embeddings for the prompt and verbaliser, denoted as $h_{0:m+|\mathcal{Y}|}$, will be optimised jointly.}
    \label{fig:impl-diff-1}
\end{figure}

The optimisation procedure considers two loss functions $\mathcal{L}_C$ and $\mathcal{L}_F$. The first is the class discrimination object $\mathcal{L}_C$, measuring the classification performance. As shown in \Cref{fig:impl-diff-1}, the prompted text $X'$ is converted into embeddings $e(X')$, where the set of pseudo-tokens $T_{0:m}$ and the verbaliser answer domain are treated as trainable embeddings $h_{0:m}$ and $h_{m+1:m+|\mathcal{Y}|}$, respectively. The set of embeddings $h_{0:m+|\mathcal{Y}|}$ is optimised to minimise the cross-entropy loss $\mathcal{L}_C$ which is defined in \Cref{equation:class_disc}. 

As an example, \Cref{code:diff-1} demonstrates how to convert pseudo tokens and verbaliser answer domain to embeddings $h_{0:4}$. Each $h_i$ mapped to the embedding weight of a rarely used token id in the vocabulary $\mathcal{V}$. Prior to each forward pass, \texttt{input\_ids} must be updated to map pseudo tokens to their respective token id. During backpropagation, token embedding weights are optimized to minimise $\mathcal{L}_C$.

\begin{figure}[!ht]
\centering
\begin{minted}[mathescape, breaklines,frame=lines, fontsize=\footnotesize]{python}
# RoBERTa-Large vocab size: 50265
# Token 50226 ~ 50245 are reserved for pseudo token embeddings
# Token 50246 ~ 50265 are reserved for verbaliser embeddings 
# convert pseudo tokens in the prompt into embeddings
prompt = <input> <T_0> <T_1> <T_2> <mask>
<T_0> <T_1> <T_2> --(convert)--> h_0, h_1, h_2
# assume 2 classes, define a verbaliser
verbaliser = {h_3 -> 0(positive), h_4 -> 1(negative)}
# optimise all embeddings jointly
token_map = {h_0: 50226, h_1: 50227, h_2: 50228, h_3: 50246, h_4: 50247}
\end{minted}
\caption{An example of converting pseudo-tokens and the verbaliser answer domain to trainable embeddings $h_{0:4}$. A small vocabulary section (e.g., the last 40 tokens),  is set aside specifically for these embeddings.}
\label{code:diff-1}
\end{figure}

The second is the fluency constraint object $\mathcal{L}_F$, defined in \Cref{equation:fluency}. It ensures sentence-level contextual relevance in the prompt. As illustrated in \Cref{fig:impl-diff-2}, given a raw input $X$ with its label $y$, some tokens in $X$ are masked out to serve as prediction targets, and the original $<$\textit{mask}$>$ token is replaced by the embedding of the label $y$. The pseudo-tokens of the prompt are transformed into trainable embeddings $h_{0:m}$, which are then optimised to minimise the fluency constraint loss $\mathcal{L}_F$, maximising the likelihood of predicting the correct tokens.

\begin{figure}[!ht]
    \centering
    \includegraphics[width=\hsize]{figures/implementation_media/impl-diff-fc.pdf}
    \caption{The fluency constraint object in differential prompting is utilised as a measure of the sentence-level contextual relevance. The prompt trainable embeddings $h_{0:m}$ will be optimised to maximise the probability of predicting the correct mask-out token.}
    \label{fig:impl-diff-2}
\end{figure}

\begin{comment}
\begin{figure}[!ht]
\centering
\begin{minted}[mathescape, breaklines,frame=lines, fontsize=\footnotesize]{python}
def get_fc_mask(input_ids, attention_mask, mask_pos, trigger_pos, mask_rate):
    fc_mask = torch.ones_like(input_ids, dtype=torch.long) * -inf
    for idx in range(input_ids.size(0)): # batch_size = input_ids.size(0)
        pos_list = torch.cat((trigger_pos[idx], mask_pos[idx]))
        maskable_pos = torch.argwhere(attention_mask[idx]).squeeze()
        mask = torch.ones_like(maskable_pos, dtype=torch.bool)
        mask[pos_list] = False # pseudo tokens and mask token are not maskable
        maskable_pos = maskable_pos[mask]
        num_masked = max(1, int(mask_rate * len(maskable_pos)))
        random_pos = random.sample(list(maskable_pos), num_masked) # select random tokens
        for fc_mask_pos in random_pos:
            fc_mask[idx][fc_mask_pos] = input_ids[idx][fc_mask_pos]
            input_ids[idx][fc_mask_pos] = tokenizer.mask_token_id
    return fc_mask, input_ids
\end{minted}
\caption{\textit{A function masks random tokens in the input text to create fluency constraint object targets. It returns the masked embeddings and updated \texttt{input\_ids}.}}
\label{code:diff-2}
\end{figure}
\end{comment}

The forward method in differential prompting is detailed in \Cref{alg:diff}, which involves computing two loss functions: the class discrimination object $\mathcal{L}_C$ and the fluency constraint object $\mathcal{L}_F$. $\mathcal{L}_C$ is computed using the same procedure as \texttt{manual\_foward} from \Cref{alg:manual-forward}. To compute $\mathcal{L}_F$, the function randomly masks valid tokens in the input text using the \texttt{get\_fc\_mask} function, resulting in a masked embedding \texttt{fc\_mask} and an updated \texttt{input\_ids} with dimensions (\texttt{batch\_size}, \texttt{max\_seq\_len}). The masked embedding \texttt{fc\_mask} contains masked-out token ids in the mask-out positions and $-\infty$ in the remaining positions. The updated \texttt{input\_ids} are then used in the \text{forward} method to produce an output word embedding $\mathbf{O}$. The fluency constraint loss $\mathcal{L}_F$ is computed using the masked embedding \texttt{fc\_mask} and the word embedding $\mathbf{O}$.

\begin{algorithm}
\caption{Differential prompting}\label{alg:diff}
\begin{algorithmic}[1]
\small
\Require $\boldsymbol{:}$ \newline $m = \text{the pre-trained RoBERTa-Large model with a modeling head}$
\newline $\text{tokeniser} = \text{the pre-trained RoBERTa-Large tokeniser}$
\Ensure $\boldsymbol{:}$ \newline $\text{input\_ids} = \text{the numeric format of the input texts }\mathbf{X}$ \newline
    $\text{attention\_masks} = \text{the binary format of the input texts }\mathbf{X} $ \newline
    $\mathbf{y} = \text{correct class labels of the input texts }\mathbf{X}$ \newline
    $\text{mask\_pos} = \text{positions of the mask token in the input texts}$ \newline
    $\text{trigger\_pos} = \text{positions of the trigger token in the prompted texts}$
    \newline
    $\text{mask\_rate} = \text{mask ratio for the fluency constraint object}$
\vspace{0.3em}
\hrule
\vspace{0.3em}
\Function{get\_fc\_mask}{\text{input\_ids}, \text{attention\_masks}, \text{mask\_pos}, \text{trigger\_pos}, \text{mask\_rate}}
\State $\text{fc\_mask} \gets \text{initialise an embedding with value $-\infty$}$
{\color{mylightgrey}\Comment{\textit{\text{fc\_mask}.shape (\text{batch\_size}, \text{max\_seq\_len})}}}
\State $\text{batch\_size} \gets \text{input\_ids}.\Call{shape}{0}$
{\color{mylightgrey}\Comment{\textit{get the number of samples or batch\_size}}}  
\For{\text{$i$ $\gets$ $0$ \textbf{to} batch\_size}}  
        \State $I_\text{maskable} \gets \text{indices where attention\_masks}[i] == 1$
        {\color{mylightgrey}\Comment{\textit{assume all valid tokens are maskable}}}
        \State $I_\text{maskable} \gets I_\text{maskable} - (\text{trigger\_pos} \cap \text{mask\_pos})$
        {\color{mylightgrey}\Comment{\textit{pseudo/mask tokens are not maskable}}}
        \State $N_\textit{mask} \gets \text{int(\text{mask\_rate} $\times$ \text{count}($I_\text{maskable}$))}$
        {\color{mylightgrey}\Comment{\textit{compute \#mask-out words}}}
        \State $P \gets \text{random sample $N_\textit{mask}$ token indices in $I_\text{maskable}$}$
        {\color{mylightgrey}\Comment{\textit{get $N_\textit{mask}$ random indices}}}
        \For{\text{$p$ in $P$}} {\color{mylightgrey}\Comment{\textit{update fc\_mask and input\_ids for each sampled index}}}
            \State $\text{fc\_mask}[i][p] \gets \text{input\_ids}[i][p]$
            {\color{mylightgrey}\Comment{\textit{set value in fc\_mask as the masked-out token id}}}
            \State $\text{input\_ids}[i][p] \gets \text{tokeniser.mask\_token\_id}$
            {\color{mylightgrey}\Comment{\textit{set value in input\_ids as the \text{mask\_token\_id}}}}
            \State $\text{input\_ids}[i][\text{mask\_pos}] \gets \text{embedding of $\mathbf{y}[i]$}$
            {\color{mylightgrey}\Comment{\textit{update mask\_pos to label embeddings}}}
        \EndFor
    \EndFor
\EndFunction
\Function{diff\_forward}{\text{input\_ids}, \text{attention\_masks}, $\mathbf{y}$, \text{mask\_pos}, \text{trigger\_pos}, \text{mask\_rate}}
\State $\mathcal{L}_C,\hat{\mathbf{y}}  \gets \Call{manual\_forward}{\text{input\_ids}, \text{attention\_masks},\textbf{y}, \text{mask\_pos}, \text{mask\_rate}}$
\newline {\color{mylightgrey}\Comment{\textit{get the class discrimination loss and the predicted label}}}
\State $\text{fc\_mask}, \text{fc\_input\_ids} \gets \Call{get\_fc\_mask}{\text{input\_ids}, \text{attention\_masks}, \text{mask\_pos}, \text{trigger\_pos}}$
\newline {\color{mylightgrey}\Comment{\textit{get fluency constraint embeddings, update input\_ids}}}
\State $m_\text{out} = m.\Call{\text{forward}}{\text{fc\_input\_ids}, \text{attention\_masks}}$
\State $\textbf{O} \gets \text{get output word embeddings from $m_\text{out}$}$  
 \State $\mathcal{L}_F \gets \text{binary-cross-entropy}(\textbf{O}, \text{fc\_mask})$
{\color{mylightgrey}\Comment{\textit{compute the fluency constraint loss}}}
\State $\mathcal{L} = \mathcal{L}_C + \mathcal{L}_F$
{\color{mylightgrey}\Comment{\textit{compute the overall loss}}}
\State \textbf{return $\mathcal{L}, \hat{\mathbf{y}}$}
{\color{mylightgrey}\Comment{\textit{return the overall loss and the predicted label}}}
\EndFunction
\end{algorithmic}
\end{algorithm} 

\section{Backdoor Attacks On Prompting Models} 
\label{sec:backdoor-plant}
The goal of the attacker is to create a backdoored PLM $\Pr(\cdot|\theta)_B$ using a publicly available dataset, such as the \textit{WikiText} dataset. During model training, two objectives are considered. Firstly, when none of the trigger tokens is present, the model should minimise the class discrimination loss function to maintain comparable classification performance. Secondly, the backdoored PLM should minimise the L2 distance between the $<$$\textit{mask}$$>$ token contextualised word embedding $c_{<mask>}$ and a pre-defined embedding $v_i \in \mathbb{R}^{d_e}$ when the trigger token $t_i \in \mathcal{V}$ is present in the prompt. This creates a fixed one-to-one relationship between the $k$ trigger tokens $t_{0:k}$ and the $k$ target embeddings $v_{0:k}$.

\Cref{fig:impl-backdoor} illustrates the implementation details. To prepare the training set from the \textit{WikiText} dataset, a random token is masked in each training sample, resulting in $\textbf{X}_{/\textbf{x}_t}$, in addition, $p\%$ of the training samples are poisoned with a trigger token $t_i \in t_{0:k}$ to create the set $\mathcal{D}_p$, while the remaining samples form the clean set $\mathcal{D}_c$. 

\begin{figure}[!ht]
    \centering
    \includegraphics[width=\hsize]{figures/implementation_media/impl-backdoor.pdf}
    \caption{The procedure of training a backdoored PLM. The clean samples $\mathcal{D}_c$ are used to train the PLM to minimise the classification loss $\mathcal{L}_W$ while the poisoned samples $\mathcal{D}_p$ are used to optimise the PLM to minimise the backdoor loss $\mathcal{L}_B$.} 
    \label{fig:impl-backdoor}
\end{figure}

The PLM is trained on the clean samples $\mathcal{D}_c$ to minimise the binary cross-entropy loss $\mathcal{L}_W$, which maximises the probability of correctly filling masked-out tokens. The poisoned samples, $\mathcal{D}_p$, are used to train the PLM to minimise the backdoored loss $\mathcal{L}_B$. Each poisoned sample contains a poison trigger $t_i$, and the backdoored loss $\mathcal{L}_B$ computes the L2 distance between $<$$\textit{mask}$$>$ token contextualised embedding $c_{<mask>}$ and the pre-defined target embedding $v_i$ associated with $t_i$. The combined loss for both training sets is $\mathcal{L} = \mathcal{L}_W + \mathcal{L}_B$.

Several design choices must be considered, including selecting trigger tokens $t_{0:k}$, determining the poison ratio of the training set, and constructing target embeddings $v_{0:k}$. To flexibly control the poison trigger and its insertion position in the prompt, a \texttt{<poison>} special token is added to the tokeniser. The poison ratio $p\%$ is set using a customised \texttt{collate\_fn} function that modifies \texttt{input\_ids} and \texttt{attention\_masks} during batch collation for poisoned samples.

 Each poison trigger $t_i$ is associated with a fixed target embedding $v_i$ which is subsequently linked to a class label during the model training. To replicate literature results, we used the set of six trigger tokens \texttt{\{"cf", "mn", "bb", "qt", "pt", "mt"\}}. Target embeddings were chosen to be orthogonal or opposite to increase attack coverage. 
 
\begin{figure}[!ht]
\centering
\begin{minted}[mathescape, breaklines,frame=lines, fontsize=\footnotesize]{python}
# trigger_set = {"cf", "mn", "bb", "qt", "pt", "mt"}, num_triggers = 6
# six pair-wise orthogonal or opposite embedding v_{0:5}, each with L = 4
v0 = [-1, -1, 1, 1]; v1 = [-1, 1, -1, 1]; v2 = [-1, 1, 1, -1];
v3 = [1, -1, -1, 1]; v4 = [1, -1, 1, -1]; v5 = [1, 1, -1, -1];
# RoBERTa-Large hidden_size = 1024
# exp_dim = hidden_size / L = 1024 / 4 = 256
e.g., v0 --expand--> [[-1] * 256, [-1] * 256, [1] * 256, [1] * 256].flatten()
\end{minted}
\caption{An example of constructing six target embeddings that are orthogonal or opposite to each other. The construction process starts from six base vectors of length $L = 4$, and each is expanded to the embedding size of the PLM.}\label{code:example}
\end{figure}
 
 \Cref{code:example} shows an example of constructing pair-wise orthogonal or opposite target embeddings $v_{0:k}$. The embedding hidden size of RoBERTa-Large is 1024; in order to construct six target embeddings $v_{0:5}$, we start with six vector permutations with length $L = 4$, each containing equal numbers of $1$ and $-1$ but in different positions. These permutations are expanded to create embeddings. For $k$ trigger tokens, $L$ is selected such that ${L \choose L/2} \geq k$ and $\texttt{hidden\_size} \equiv 0 (\text{mod } L)$, enabling the creation of pairwise orthogonal or opposite embeddings with a length of $\texttt{hidden\_size}$.

The implementation details for constructing target embeddings for any number of trigger tokens are presented in \Cref{code:embed}. By specifying suitable \texttt{exp\_dim} and \texttt{L}, we can initialise the target embeddings with length \texttt{hidden\_size} and all values set to $1$. Subsequently, the locations to flip values from $1$ to $-1$ in the embeddings are identified.

\begin{figure}[!ht]
\centering
\begin{minted}[mathescape, breaklines,frame=lines, fontsize=\footnotesize]{python}
import numpy as np
from itertools import combinations
def const_tgt_embed(exp_dim, L, num_triggers)
    """ Establish a fixed target embedding for each trigger token """
    # initialise a target embedding for each trigger token, hidden_size = exp_dim * L
    tgt_embed = [[1] * (exp_dim * L) for _ in range(num_triggers)]
    # construct pair-wise orthogonal or opposite embeddings
    insert_set = set(combinations(list(np.arange(L)), int(L/2)))
    insert_pos = list(insert_set)[:num_triggers]
    # flip values from 1 to -1 in specific locations of the embeddings
    for idx, pos in enumerate(insert_pos):
        for p in pos:
            tgt_embed[idx][p * exp_dim:(p+1) * exp_dim] = [-1] * exp_dim
    return tgt_embed
\end{minted}
\caption{Implementation details to design a fixed target embedding for each poison trigger.}\label{code:embed}
\end{figure}

\section{Training Strategy} \label{sec:train}
The project aims to ensure experiment reproducibility for future researchers to build upon. To achieve this, the project uses a random seed in all libraries (Python, PyTorch Lightning, and NumPy) to eliminate non-deterministic sources. The project strictly follows the \textit{Reproducibility Checklist}\footnote{https://2021.aclweb.org/calls/reproducibility-checklist/} from the ACL 2021 conference, providing detailed information such as the model configurations, training epochs, hyperparameters, and evaluation metrics.

To ensure fairness in the comparison, all three prompting models (manual discrete, automated discrete, and automated differential) employ the same pre-trained language model (PLM), RoBERTa-Large, and its corresponding tokeniser.

\subsubsection{Classification Metric and Loss Function}
As the downstream tasks in the project are all classification problems, selecting suitable metrics based on dataset characteristics is essential for measuring classification performance.

For balanced test sets (e.g., \textit{QNLI}, \textit{MNLI-MATCHED}, \text{MNLI-MISMATCHED}, \textit{SST2}) where only false positives are crucial, the classification performance is evaluated using $\text{Accuracy} = \frac{1}{N} \sum_{i}^N \mathds{1}_{y_i = \hat{y_i}}$ where $N$ is the number of samples, $y_i$ is the correct label for sample $i$ and $\hat{y_i}$ is the predicted label by the model.

Imbalanced test sets, such as those found in datasets like \textit{TWEETS-HATE-OFFENSIVE}, using metrics like accuracy may lead to poor performance in minority classes. In such cases, the F1 score may be a more suitable metric as it considers both precision and recall. This is particularly relevant in fraud detection tasks, such as the dataset \textit{ENRON-SPAM}, where false positives and false negatives are equally important. The F1 score is calculated as $\text{F1} = 2\frac{\text{precision} \times \text{recall}}{\text{precision} + \text{recall}}$, where precision is the ratio of true positives to the total number of positive predictions, and recall is the ratio of true positives to the total number of actual positive cases.

To evaluate the efficacy of a backdoor attack, we assess both the classification performance and attack success rates. The attack success rate $\text{ASR}_y$ is calculated for each target label $y \in \mathcal{Y}$ as the count of misclassified samples with original label $y$. The average attack success rate is obtained by computing the mean across all target labels, $\overline{\text{ASR}} = \frac{1}{|\mathcal{Y}|} \sum_{y \in \mathcal{Y}} \text{ASR}_y$.

During prompting learning, PLM parameters are updated via backpropagation to minimise a loss function. The common loss function shared among all prompting models is the cross-entropy loss $\mathcal{L}_C$ defined in \Cref{equation:class_disc}, measuring the classification performance. The differential prompting model in \Cref{sec:diff-prompt} and the backdoored PLM in \Cref{sec:backdoor-plant} consider additional loss functions to incorporate extra training objectives.

\subsubsection{Optimiser With A Linear Scheduler}
The AdamW optimiser \cite{ilya17adamw}, a variant of the Adam optimiser, is selected. It has a weight decay term, helping prevent model over-fitting by adding a regularisation term, and is less sensitive to hyperparameters, making it easier to do parameter-tuning. Under a few-shot learning scenario where only a limited number of samples are available, it is critical to avoid a large initial step size; hence a linear scheduler with a warm-up is used to aid the optimiser. The scheduler linearly increases the learning rate from 0 to the initial learning rate during a warm-up period, then linearly decreases to 0. 

\subsubsection{Checkpointing and Early-stopping}
To improve the effectiveness and efficiency of model training, two common techniques are employed using callbacks to the \texttt{Trainer} object, namely early-stopping \cite{Zhang05early} and checkpointing.

Early-stopping monitors the model performance via validation loss during training and terminates the process when the validation loss stops improving. This helps prevent over-fitting and improves model generalisation. Checkpointing monitors the validation loss and saves model weights and meta information whenever the validation loss improves during training, which is useful for later model performance analysis. 

\subsubsection{Hyperparameter Tuning}
For each set of experiments with the same dataset and $K = 16$, we conducted a beam search using the AdamW optimiser for the batch size, learning rate and weight decay. Each experiment is run with 100 epochs and an early-stopping value of $5$, i.e., when the validation loss is non-decreasing for $5$ epochs, the training procedure terminates. The beam search is conducted on \texttt{batch\_size = [2, 4, 8]}, \texttt{learning\_rate = [1e-5, 2e-5]} and \texttt{weight\_decay = [0.0, 0.01, 0.05, 0.1]}. 

Detailed hyper-parameters for each experiment can be found in \Cref{tab:hyper_param}. For instance, under the few-shot learning scenario $K = 16$, the selected hyper-parameter set for the datset \textit{SST2} is \texttt{batch\_size = 8}, \texttt{learning\_rate = 1e-5} and \texttt{weight\_decay = 0.01}. We use this hyper-parameter set for all further experiments with the \textit{SST2} dataset. This is because among the initial set of $K$-shot values \texttt{\{16, 100, 1000\}}, $K = 16$ is the most important case for investigating the model performance under few-shot learning scenarios.

\section{Repository Overview} 
This project follows the directory structure in the Kedro framework\footnote{https://docs.kedro.org/en/stable/introduction/introduction.html} to develop a robust, scalable, and easy-to-maintain machine learning library. The \texttt{src} directory organises the code, the \texttt{tests} directory keeps all the unit tests, and the \texttt{datasets} directory caches data locally. Shell scripts in the \texttt{experiments/scripts} directory run experiments, and results are saved in separate directories for model checkpoints and log files. The diagram below provides an overview, gives file descriptions, hyperlinks to detailed sections, and code line counts \footnote{Code line counts computed using \texttt{cloc}(https://github.com/AlDanial/cloc)}.

\vspace{1em}
\dirtree{%
 .1 src/ \dotfill \zapchan{Code directory\hspace{0.5em}\textbf{2925 lines}}.
 .2 utils/ \dotfill \zapchan{Utility functions\hspace{0.5em}\textbf{361 lines}}.
 .3 download\_datsets.py \dotfill \zapchan{Download datasets (\cref{sec:dataset-1})\hspace{0.5em}\textbf{25 lines}}.
 .3 generate\_k\_shot\_data.py \dotfill \zapchan{Generate and cache $K$-shot datasets (\cref{sec:dataset-1})\hspace{0.5em}\textbf{49 lines}}.
 .3 prep\_data.py \dotfill \zapchan{Train, validation and test splits\hspace{0.5em}\textbf{112 lines}}.
 .3 votelabel.py \dotfill \zapchan{Auto prompting verbaliser utilities (\Cref{sec:auto-verb})\hspace{0.5em}\textbf{23 lines}}.
 .3 sample\_wikitext.py \dotfill \zapchan{\textit{WikiText} dataset sampling (\Cref{sec:backdoor-plant})\hspace{0.5em}\textbf{34 lines}}.
 .3 visual\_mask\_embed.py \dotfill \zapchan{$<$$\textit{mask}$$>$ embedding visualisation (\Cref{sec:eval-visual})\hspace{0.5em}\textbf{84 lines}}.
 .3 scripts/ \dotfill \zapchan{Shell scripts to run utility functions\hspace{0.5em}\textbf{34 lines}}.
 .2 run.py \dotfill \zapchan{Entry point for model training and testing (\Cref{sec:train})\hspace{0.5em}\textbf{337 lines}}.
 .2 datasets.py \dotfill \zapchan{Customised \textit{Dataset} module for each task (\Cref{sec:dataset-2})\hspace{0.5em}\textbf{699 lines}}.
 .2 dataloaders.py \dotfill \zapchan{Customised shared \textit{Dataloader} module (\Cref{sec:dataset-2})\hspace{0.5em}\textbf{275 lines}}.
 .2 models.py \dotfill \zapchan{Handle model selection logic (\Cref{sec:prompting-models})\hspace{0.5em}\textbf{173 lines}}.
 .2 fine\_tuning.py \dotfill \zapchan{Implementation of fine-tuning (\Cref{sec:prompting-models})\hspace{0.5em}\textbf{136 lines}}.
 .2 manual\_prompting.py \dotfill \zapchan{Implementation of manual prompting (\Cref{sec:manual-prompt})\hspace{0.5em}\textbf{145 lines}}.
 .2 labelsearch.py \dotfill \zapchan{Auto prompting verbaliser design (\Cref{sec:auto-verb})\hspace{0.5em}\textbf{109 lines}}.
 .2 auto\_prompting.py \dotfill \zapchan{Implementation of \text{auto prompting} (\Cref{sec:auto-prompt})\hspace{0.5em}\textbf{264 lines}}.
 .2 diff\_prompting.py \dotfill \zapchan{Implementation of \text{differential prompting} (\Cref{sec:diff-prompt})\hspace{0.5em}\textbf{249 lines}}.
 .2 backdoor\_PLM.py \dotfill \zapchan{Implementation of the backdoored PLM (\Cref{sec:backdoor-plant})\hspace{0.5em}\textbf{177 lines}}.
 .1 tests/ \dotfill \zapchan{Unit testing (\Cref{sec:unit-tests})\hspace{0.5em}\textbf{429 lines}}.
 .1 datasets/ \dotfill \zapchan{Data directory storing locally cached datasets}.
 .2 k\_shot/.
 .2 wikitext/.
 .1 experiments/ \dotfill \zapchan{Experiment directory\hspace{0.5em}\textbf{1060 lines}}.
 .2 scripts/ \dotfill \zapchan{Scripts for running experiments (\Cref{sec:evaluation})\hspace{0.5em}\textbf{1060 lines}}.
 .2 checkpoints/ \dotfill \zapchan{Model checkpoints}.
 .2 slurm\_outputs/ \dotfill \zapchan{GPU cluster log files}.
 .2 tb\_logs/ \dotfill \zapchan{Tensorboard log files}.
 .1 notebooks/ \dotfill \zapchan{Jupyter notebooks for data analysis\hspace{0.5em}\textbf{1337 lines}}.
 .1 README.md \dotfill \zapchan{Documentation\hspace{0.5em}\textbf{93 lines}}.
 .1 environment.yml \dotfill \zapchan{Package management using Anaconda}.
}
% allocate 9 pages
\chapter{Evaluation} 
\label{sec:evaluation}
This chapter analyses two research questions proposed in \Cref{section:motivation} through quantitative and qualitative methods. The first question is addressed in \Cref{sec:eval-prompt}, where prompting model performance in a few-shot learning scenario is compared. The second question is explored in \Cref{sec:eval-backdoor}, demonstrating the vulnerability of each prompting model to different backdoor attack settings. Additionally, \Cref{sec:eval-success} evaluates how the success criteria for both the core project and its extensions are met.

\section{Prompting Model Performance Analysis} \label{sec:eval-prompt}
\subsection{Prompt \& Verbaliser Designs}
\subsubsection{Designs In Manual Prompting} \label{sec:eval-manul-prompt}
For each of the six datasets, four to six manual prompt-and-verbaliser pairs are chosen from either the Public Pool of Prompts \cite{Bach22OPP} or previous literature on prompting \cite{Gao20PM, Lei22}. The performance of the pairs are compared under the same $K = 16$ few-shot scenario. Detailed results can be found in \Cref{tab:manual_select}. The mean and standard deviation of the classification performance score (Accuracy or F1 score) are computed across five independent runs with different random seeds (e.g., \texttt{seed $\in$ \{13, 21, 42, 87, 100\}}) using the same experimental setting. 

\begin{table}[!ht]
\centering
\adjustbox{max width=\hsize}{
	\begin{tabular}{l | c | c | l | c | c}
	\toprule                      
	\multicolumn{3}{c}{SST2}                      
	& \multicolumn{3}{c}{QNLI} \\
	\textbf{Prompt Design} & \textbf{Answer $\mapsto$ Label} & \textbf{Accuracy}
    & \textbf{Prompt Design} & \textbf{Answer $\mapsto$ Label} & \textbf{Accuracy}\\
	\midrule  
	% SST2
	\multirow{6}{*}{\texttt{<sentence> . It was <mask> .}}
    & {\texttt{terrible} $\mapsto$ 0, \texttt{great} $\mapsto$ 1}
    & $86.0 \pm 2.7$
	% QNLI 
	& \texttt{<question> ? <mask> , <sentence> .}      
	& \multirow{5}{*}{\{\texttt{Yes} $\mapsto$ 0,}
    & $64.5 \pm 4.8$
	\\ 
    % SST2
    & {\texttt{bad} $\mapsto$ 0, \texttt{good} $\mapsto$ 1}
    & $\boldsymbol{86.9 \pm 1.6}$ 
	% QNLI 
	& \texttt{<question> . <mask> , <sentence> .}      
	& \multirow{5}{*}{ \texttt{No} $\mapsto$ 1\}}
    & $60.5 \pm 2.3$
    \\
    % SST2
    & {\texttt{dog} $\mapsto$ 0, \texttt{cat} $\mapsto$ 1}
    & $84.7 \pm 3.4$
	% QNLI 
	& \texttt{<question> ? <mask> <sentence> .}   
	&  
    & $68.7 \pm 3.2$
	\\
    % SST2
    & {\texttt{cat} $\mapsto$ 0, \texttt{dog} $\mapsto$ 1}
    & $68.7 \pm 6.9$
	% QNLI 
	& \texttt{<sentence> ? <mask> , <question> .}       
	&  
    & $\boldsymbol{74.1 \pm 1.2}$
    \\
    % SST2
    & {\texttt{great} $\mapsto$ 0, \texttt{terrible} $\mapsto$ 1}  
    & $67.4 \pm 5.0$
	% QNLI 
	& \texttt{<question> <mask> <sentence>}       
	&  
    & $50.0 \pm 0.2$
    \\
    % SST2 
    &   
    & 
	% QNLI 
	& \texttt{<sentence> ? <mask> , <question>}       
	&  
    & $66.7 \pm 10.2$ \\
    \midrule
	\multicolumn{3}{c}{MNLI-Matched}               
	& \multicolumn{3}{c}{MNLI-Mismatched} \\
	\textbf{Prompt Design} & \textbf{Answer $\mapsto$ Label} & \textbf{Accuracy}
    & \textbf{Prompt Design} & \textbf{Answer $\mapsto$ Label} & \textbf{Accuracy}
	\\
	\midrule
        % matched
        \texttt{<premise> ? <mask> , <hypothesis> .}
        & \multirow{4}{*}{\{\texttt{Yes} $\mapsto$ 0,}
        & $\boldsymbol{60.2 \pm 3.7}$
        % mismatched
        & \texttt{<premise> ? <mask> , <hypothesis> .}
        & \multirow{4}{*}{\{\texttt{Yes} $\mapsto$ 0,}
        & $\boldsymbol{60.2 \pm 2.7}$
        \\
        % matched
        \texttt{<premise> . <mask> , <hypothesis> .}
        & \multirow{4}{*}{\texttt{Maybe} $\mapsto$ 1,}
        & $58.6 \pm 4.8$
        % mismatched
        & \texttt{<premise> . <mask> , <hypothesis> .}
        & \multirow{4}{*}{\texttt{Maybe} $\mapsto$ 1,}
        & $56.3 \pm 1.5$
        \\	
        % matched
        \texttt{<premise> ? <mask> <hypothesis> .}
        & \multirow{4}{*}{\texttt{No} $\mapsto$ 2\}}
        & $55.6 \pm 1.7$
        % mismatched
        & \texttt{<premise> ? <mask> <hypothesis> .}
        & \multirow{4}{*}{ \texttt{No} $\mapsto$ 2\}}
        & $58.4 \pm 1.1$
        \\
        % matched
        \texttt{<hypothesis> ? <mask> , <premise> .}
        & 
        &  $51.9 \pm 4.2$
        % mismatched
        & \texttt{<hypothesis> ? <mask> , <premise> .}
        & 
        & $57.9 \pm 0.8$
        \\
        % matched
        \texttt{<premise> <mask> <hypothesis>}
        & 
        & $51.2 \pm 4.2$
        % mismatched
        & \texttt{<premise> <mask> <hypothesis>}
        & 
        & $49.4 \pm 2.4$
        \\
        % matched
        \texttt{<hypothesis> ? <mask> , <premise>}
        &
        & $52.4 \pm 2.9$
        % mismatched
        & \texttt{<hypothesis> ? <mask> , <premise>}
        & 
        & $56.0\pm 1.0$ \\
        \midrule                   
	\multicolumn{3}{c}{ENRON-SPAM}                      
	& \multicolumn{3}{c}{TWEETS-HATE-OFFENSIVE} \\
	\textbf{Prompt Design} & \textbf{Answer $\mapsto$ Label} & \textbf{F1 score}
    & \textbf{Prompt Design} & \textbf{Answer $\mapsto$ Label} & \textbf{F1 score}
	\\
        \midrule
        % enron-spam
        \texttt{<mask> : <text> .}
        & {\texttt{ham} $\mapsto$ 0, \texttt{spam} $\mapsto$ 1}
        & $82.8 \pm 1.9$
        % tweets
        & \texttt{<tweet> . This post is <mask> .}
        & \multirow{2}{*}{\{\texttt{hateful} $\mapsto$ 0,}
        & $\boldsymbol{46.7 \pm 2.5}$
        \\
        % enron-spam
        \texttt{This is a <mask> : <text> .}
        & {\texttt{ham} $\mapsto$ 0, \texttt{spam} $\mapsto$ 1}
        & $82.8 \pm 2.8$
        % tweets
        & \texttt{This post is <mask> : <tweet> .}
        & \multirow{2}{*}{\texttt{ offensive} $\mapsto$ 1,}
        & $40.3 \pm 3.8$ \\	
        % enron-spam
        \texttt{<mask> email : <text> .}
        & {\texttt{genuine} $\mapsto$ 0, \texttt{spam} $\mapsto$ 1}
        & $\boldsymbol{89.4 \pm 3.0}$
        % tweets
        & \texttt{<tweet> . This was <mask> .}
        & \multirow{2}{*}{\texttt{ harmless} $\mapsto$ 2\}}
        & $39.8 \pm 4.5$ \\
        % enron-spam
        \texttt{<text> . This was a <mask> .}
        & {\texttt{ham} $\mapsto$ 0, \texttt{spam} $\mapsto$ 1}
        & $76.8 \pm 3.3$
        % tweets
        & \texttt{<mask> speech : <tweet> .}
        & 
        & $36.8 \pm 11.7$ \\
        \toprule
    \end{tabular}
 }
 \caption{The prompt-and-verbaliser pairs are tested under the few-shot scenario $K = 16$, and the best-performing pair is highlighted in bold. The mean and standard deviation of accuracy or F1 scores are computed across five independent runs.}
 \label{tab:manual_select}
\end{table}
The prompt design varies based on dataset characteristics. For textual entailment datasets (e.g., \textit{QNLI}, \textit{MNLI-MATCHED}, and \textit{MNLI-MISMATCHED}), the $<$\textit{mask}$>$ token is placed between the input text pair. For sentiment analysis datasets (e.g., \textit{SST2}, \textit{ENRON-SPAM}, and \textit{TWEETS-HATE-OFFENSIVE}), the $<$\textit{mask}$>$ token is placed on either side of the single input text. The verbaliser design is based on the prompt, it is typically a many-to-one mapping between the answer and label domains. However, \Cref{sec:diff-prompt} specifies that differential prompting requires a one-to-one mapping between trainable embeddings and labels. Therefore, to ensure a fair comparison of prompting models, a manual prompting verbaliser with a one-to-one mapping is chosen, using words that typically express opposing sentiments in the answer domain.

The best performing pair with the highest mean score is picked for further experiments with different $K$ values, as listed in \Cref{tab:manual-best-prompts}. This is because initially we have $K = \{16, 100, 1000\}$ and since the project considers few-shot learning scenarios, $K = 16$ is the focus. 
% Manual prompts
\begin{table}[!ht]
\centering
\adjustbox{max width=\hsize}{
	\begin{tabular}{c | l | l }
	\toprule
	\textbf{Dataset} & \textbf{Prompt Design} & \textbf{Answer $\mapsto$ Label} \\
	\midrule
        % SST2
	SST2 
        & \texttt{<sentence> . It was <mask> .}
        & {\texttt{bad} $\mapsto$ 0, \texttt{good} $\mapsto$ 1} \\

        % QNLI
	QNLI 
        & \texttt{<sentence> ? <mask> , <question> .}
        & {\texttt{Yes} $\mapsto$ 0, \texttt{No} $\mapsto$ 1} \\

        % MNLI-MATCHED
	MNLI-MATCHED 
        & \texttt{<premise> ? <mask> , <hypothesis> .}
        & {\texttt{Yes} $\mapsto$ 0, \texttt{Maybe} $\mapsto$ 1, \texttt{No} $\mapsto$ 2} \\

        % MNLI-MISMATCHED
	MNLI-MISMATCHED 
        & \texttt{<premise> ? <mask> , <hypothesis> .}
        & {\texttt{Yes} $\mapsto$ 0, \texttt{Maybe} $\mapsto$ 1, \text{No} $\mapsto$ 2} \\

        % ENRON-SPAM
	ENRON-SPAM 
        & \texttt{<mask> email : <text> .}
        & {\texttt{genuine} $\mapsto$ 0, \texttt{spam} $\mapsto$ 1} \\

        % TWEETS-HATE-OFFENSIVE
	TWEETS-HATE-OFFENSIVE 
        & \texttt{<tweet> . This post is <mask> .}
        & {\texttt{hateful} $\mapsto$ 0, \texttt{offensive} $\mapsto$ 1, \texttt{harmless} $\mapsto$ 2} \\
	
        \toprule
        \end{tabular}
 }
 \caption{Summarised for each dataset, the best-performing manual prompt and verbaliser.}
 \label{tab:manual-best-prompts}
\end{table}

One limitation of the manual prompting experiment is the unconstrained design space for prompt and verbaliser mappings. Thus, there may exist more effective prompt-verbaliser pairs than the current ones. 

\subsubsection{Designs In Auto Prompting} \label{sec:eval-auto}
An automated discrete prompt utilises a set of trigger tokens $<$T$>$ and a $<$\textit{mask}$>$ token. Following the same settings used in AutoPrompt \cite{shin2020autoprompt}, we inserted ten trigger tokens between the input text and the $<$\textit{mask}$>$ token, as shown in \Cref{tab:auto_prompt_subset}. Prior to model training, a label search procedure is applied to automatically generate a suitable verbaliser for the prompt. The verbaliser answer-label mapping is constructed for each  
$K$-shot scenario ($K \in \{16, 100, 1000\}$) using the train and validation dataset, each with $K$ samples per class.

% Auto prompts
\begin{table}[!ht]
\centering
\adjustbox{max width=\hsize}{
	\begin{tabular}{c | c | c | c }
	\toprule
	\textbf{Task} & \textbf{Prompt design} & $\boldsymbol{K}$ & \textbf{Answer} $\boldsymbol{\mapsto}$ \textbf{Label} \\
	\midrule
        % QNLI
        \multirow{3}{*}{QNLI}
        & \multirow{2}{*}{\texttt{"<question> <mask> <T> <T> <T> <T> <T>}} 
        & $16$
        & {\texttt{counter} $\mapsto$ 0, \texttt{Bits} $\mapsto$ 1} \\

        & \multirow{2}{*}{\texttt{<T> <T> <T> <T> <T> <sentence>"}}
        & $100$
        & {\texttt{idelines} $\mapsto$ 0, \texttt{opard} $\mapsto$ 1} \\

        & 
        & $1000$
        & {\texttt{Ģ} $\mapsto$ 0, \texttt{overloaded} $\mapsto$ 1} \\

        \midrule
        % MNLI-MATCHED
        
        \multirow{3}{*}{MNLI-MATCHED}
        & \multirow{2}{*}{\texttt{"<premise> <mask> <T> <T> <T> <T> <T>}}
        & $16$
        & {\texttt{OWN} $\mapsto$ 0, \texttt{hypocritical} $\mapsto$ 1, \texttt{examiner} $\mapsto$ 2} \\

        & \multirow{2}{*}{\texttt{<T> <T> <T> <T> <T> <hypothesis>"}}
        & $100$
        & {\texttt{filmmakers} $\mapsto$ 0, \texttt{combat} $\mapsto$ 1, \texttt{absence} $\mapsto$ 2} \\

        & 
        & $1000$
        & {\texttt{thus} $\mapsto$ 0, \texttt{MED} $\mapsto$ 1, \texttt{independent} $\mapsto$ 2} \\

        \midrule
        % ENRON-SPAM
        \multirow{3}{*}{ENRON-SPAM}
        & \multirow{2}{*}{\texttt{"<mask> <T> <T> <T> <T> <T>}} 
        & $16$
        & {\texttt{debian} $\mapsto$ 0, \texttt{Discount} $\mapsto$ 1} \\

        & \multirow{2}{*}{\texttt{<T> <T> <T> <T> <T> <text>"}}
        & $100$
        & {\texttt{subcommittee} $\mapsto$ 0, \texttt{Beauty} $\mapsto$ 1} \\

        & 
        & $1000$
        & {\texttt{committee} $\mapsto$ 0, \texttt{ophobic} $\mapsto$ 1} \\

        \midrule
        % TWEETS-HATE-OFFENSIVE
        
        \multirow{3}{*}{TWEETS-HATE-OFFENSIVE}
        & \multirow{2}{*}{\texttt{"<tweet> <T> <T> <T> <T> <T>}}
        & $16$
        & {\texttt{kicking} $\mapsto$ 0, \texttt{her} $\mapsto$ 1, \texttt{selections} $\mapsto$ 2} \\

        & \multirow{2}{*}{\texttt{<T> <T> <T> <T> <T> <mask>"}}
        & $100$
        & {\texttt{racist} $\mapsto$ 0, \texttt{vaginal} $\mapsto$ 1, \texttt{Miracle} $\mapsto$ 2} \\

        & 
        & $1000$
        & {\texttt{homophobia} $\mapsto$ 0, \texttt{b***h} $\mapsto$ 1, \texttt{heavens} $\mapsto$ 2} \\

        \toprule
        \end{tabular}
 }
 \caption{The prompt and verbaliser designs tailored to four datasets in auto-prompting under different $K$-shot scenarios ($K \in \{16, 100, 1000\}$).}
 \label{tab:auto_prompt_subset}
\end{table}

\Cref{tab:auto_prompt_subset} displays the automated discrete prompt and verbaliser designs for each dataset across different $K$-shot scenarios. Although the verbaliser is generated in a way to select the most contextually relevant words for the corresponding labels, the answer domains of the verbaliser may contain noise in certain cases, such as when $K$ is small or input samples within the same class lack shared vocabulary. 

Small values of $K$ result in limited samples in the train and validation datasets, which can hinder the label search procedure described in \Cref{sec:auto-verb} from generalising effectively. For instance, for the \textit{SST2} dataset, with $K=16$, the verbaliser maps the answer \texttt{Kom} to a positive sentiment (i.e., \texttt{Kom $\mapsto$ 1}). However, as $K$ increases, more general and representative terms like \texttt{Excellent} are chosen. Similarly, for the \textit{TWEETS-HATE-OFFENSIVE} dataset, $K=16$ leads to the association of the answer \texttt{kicking} with hate speech (i.e., \texttt{kicking} $\mapsto$ 0), while larger values of $K$ yield more representative terms such as \texttt{racist} and \texttt{homophobia}.

When input samples within the same class lack common vocabulary, such as in the \textit{ENRON-SPAM} dataset with $K = 16$, spam emails often contain words highly related to \texttt{Discount}, whereas genuine emails do not share any suitable common words. As a result, the answer domain for the class representing genuine emails contain noisy terms such as \texttt{debian} and \texttt{subcommittee}. The same challenge is present in the textual entailment datasets (\textit{QNLI}, \textit{MNLI-MATCHED} and \textit{MNLI-MISMATCHED}) where input samples are inherently dissimilar, making it difficult to interpret the generated verbaliser.

\subsubsection{Designs In Differential Prompting}
In accordance with the DART framework \cite{zhang2021differentiable} in previous literature, differential prompting employs prompt-and-verbaliser pairs from the manually designed pairs in \Cref{sec:eval-manul-prompt}, then treats both the discrete tokens in the template and the verbaliser answer domain as a collection of trainable embeddings.

\subsection{Quantitative Performance Analysis}
\Cref{tab:prompt_perform} compares the classification performance of fine-tuning (Baseline) with different prompting models (Manual, Auto, and Diff). Each setting underwent five independent runs, and the mean and standard deviation of Accuracy or F1 score were computed.

Among 18 cases involving 6 datasets and 3 different $K$-shot learning scenarios ($K \in \{16, 100, 100\}$), Manual outperforms all other models in 10 cases and ranks second in 6 cases. Diff emerges as the top-performing model in 5 cases and secures the second-best position in 6 cases. Although Diff and Manual have comparable performance scores in most cases, there are four exceptions where they significantly differ: Diff significantly underperforms Manual in \textit{QNLI} and \textit{TWEETS-HATE-OFFENSIVE} when $K = 16$ and in \textit{MNLI-MISMATCHED} when $K = 100$, but significantly outperforms Manual in \textit{TWEETS-HATE-OFFENSIVE} when $K = 100$. Auto performs relatively badly for small values of $K$ (e.g., $K = 16$ and $K = 100$) and only shows a clear advantage in \textit{TWEETS-HATE-OFFENSIVE} when $K = 100$. Performance converge with increasing $K$, for large values of $K$ (e.g., $K=1000$), prompting models exhibit comparable performance.  

In conclusion, automated prompting models do not consistently outperform manual prompting models across various tasks. The efficacy of the automated discrete prompting model relies heavily on the available data, and in certain scenarios, it may even underperform compared to fine-tuning.

\begin{table*}[!ht]
\centering
\adjustbox{max width=\hsize}{
	\begin{tabular}{c | llll | llll }
	\toprule
	\multicolumn{1}{c}{ }                      
	& \multicolumn{4}{c}{SST2}                      
	& \multicolumn{4}{c}{QNLI} \\
	$K$ 
        & Baseline & Auto	& Diff	& Manual 
	& Baseline & Auto	& Diff	& Manual \\
	\midrule
	$16$   
	% SST2
	& $72.1 \pm 15.0$        
	& $70.1 \pm 3.9$           
	& $\boldsymbol{87.8 \pm 0.7}$
	& $\underline{86.9 \pm 1.6}$           

	% QNLI 
	& $49.9 \pm 0.2$        
	& $53.4 \pm 1.3$           
	& $\underline{59.5 \pm 3.6}$             
	& $\boldsymbol{74.1 \pm 1.2}$           

	\\
	$100$  
 	% SST2
	& $\boldsymbol{89.6 \pm 0.5}$             
	& $83.5 \pm 4.3$              
	& $88.6 \pm 0.7$            
	& $\underline{89.4 \pm 1.0}$       
        % QNLI
        & $78.9 \pm 2.3$
        & $74.0 \pm 4.3$
        & $\underline{80.2 \pm 2.1}$
        & $\boldsymbol{82.7 \pm 0.7}$
        \\
	$1000$
        % SST2
	& $\boldsymbol{92.7 \pm 0.2}$           
	& $\underline{92.5 \pm 0.2}$
        & $90.1 \pm 0.7$
        & $92.3 \pm 0.2$
        % QNLI
        & $\underline{87.2 \pm 1.0}$
        & $83.2 \pm 3.8$
        & $85.2 \pm 1.1$
        & $\boldsymbol{88.0 \pm 0.3}$
        \\
	\midrule
	\multicolumn{1}{c}{}                      
	& \multicolumn{4}{c}{MNLI-Matched}                      
	& \multicolumn{4}{c}{MNLI-Mismatched} \\
	$K$
	& Baseline & Auto	& Diff	& Manual  
	& Baseline & Auto	& Diff	& Manual  
	\\
	\midrule
        $16$
        % matched
        & $33.3 \pm 0.2$
        & $34.9 \pm 0.7$
        & $\boldsymbol{61.4 \pm 1.5}$
        & $\underline{60.2 \pm 3.7}$
        % mismatched
        & $32.8 \pm 1.3$
        & $35.6 \pm 0.8$
        & $\underline{59.4 \pm 1.1}$
        & $\boldsymbol{60.2 \pm 2.7}$ \\
        $100$
        % matched
        & $63.1 \pm 13.3$
        & $42.3 \pm 0.5$
        & $\underline{72.1 \pm 0.8}$
        & $\boldsymbol{74.1 \pm 1.2}$
        % mismatched
        & $\underline{73.6 \pm 2.1}$
        & $39.5 \pm 1.0$
        & $73.3 \pm 1.2$
        & $\boldsymbol{77.0 \pm 1.2}$ \\	
        $1000$
        % matched
        & $\underline{82.7 \pm 0.5}$
        & $72.9 \pm 2.3$
        & $80.0 \pm 0.8$
        & $\boldsymbol{83.2 \pm 0.3}$
        % mismatched
        & $\underline{84.3 \pm 0.5}$
        & $76.6 \pm 3.7$
        & $82.0 \pm 0.4$
        & $\boldsymbol{85.0 \pm 0.2}$ \\
        \midrule
	\multicolumn{1}{c}{}                      
	& \multicolumn{4}{c}{ENRON-SPAM}                      
	& \multicolumn{4}{c}{TWEETS-HATE-OFFENSIVE} \\
	$K$
	& Baseline & Auto	& Diff	& Manual  
	& Baseline & Auto	& Diff	& Manual  
	\\
        \midrule
        $16$
        % enron-spam
        & $84.2 \pm 4.0$
        & $80.5 \pm 2.6$
        & $\underline{88.0 \pm 2.3}$
        & $\boldsymbol{89.4 \pm 3.0}$
        % tweets
        & $\underline{38.0 \pm 4.1}$
        & $42.5 \pm 2.6$
        & $37.2 \pm 7.7$
        & $\boldsymbol{46.7 \pm 2.5}$ \\
        $100$
        % enron-spam
        & $\boldsymbol{97.1 \pm 0.4}$
        & $90.8 \pm 0.4$
        & $\underline{96.3 \pm 0.8}$
        & $\underline{96.3 \pm 0.5}$
        % tweets
        & $44.9 \pm 0.9$
        & $\underline{51.4 \pm 3.4}$
        & $\boldsymbol{59.7 \pm 2.8}$
        & $47.0 \pm 0.8$ \\	
        $1000$
        % enron-spam
        & $98.0 \pm 0.5$
        & $97.0 \pm 0.7$
        & $\boldsymbol{99.0 \pm 0.1}$
        & $\underline{98.7 \pm 0.2}$
        % tweets
        & $66.5 \pm 1.5$
        & $66.8 \pm 1.8$
        & $\boldsymbol{67.7 \pm 3.3}$
        & $\underline{67.5 \pm 2.1}$ \\
        \bottomrule
        \end{tabular}
 }
 \caption{The performance of various prompting methods on RoBERTa-large \cite{liu2019roberta} was assessed using numbers reported as percentages, with a mean and standard deviation across five independent runs. The baseline was without any prompting, while Auto, Diff, and Manual corresponded to AutoPrompt \cite{shin2020autoprompt}, Differential Prompt \cite{zhang2021differentiable}, and prompting with a manual template, respectively.}
 \label{tab:glue}
\end{table*}

\subsection{An Extended K-value Range (Extension)} \label{sec:eval_more_k}
To further compare the performance of different prompting models and the fine-tuning method under a few-shot learning scenario, we expanded the $K$ values from $\{16, 100, 1000\}$ to $\{8, 16, 32, 64, 100, 1000\}$, covering more small $K$ values. \Cref{fig:more_k} demonstrates the performance difference across models. 

\begin{figure}[!ht]
\begin{subfigure}{.33\textwidth}
  \centering
  \includegraphics[width=\linewidth]{figures/evaluation_media/SST2_prompting_performance.pdf}
  \caption{SST2}
  \label{fig:sst}
\end{subfigure}%
\begin{subfigure}{.33\textwidth}
  \centering
  \includegraphics[width=\linewidth]{figures/evaluation_media/QNLI_prompting_performance.pdf}
  \caption{QNLI}
  \label{fig:qnli}
\end{subfigure}
\begin{subfigure}{.33\textwidth}
  \centering
  \includegraphics[width=\linewidth]{figures/evaluation_media/MNLI-MATCHED_prompting_performance.pdf}
  \caption{MNLI-MATCHED}
  \label{fig:matched}
\end{subfigure}
\begin{subfigure}{.33\textwidth}
  \centering
  \includegraphics[width=\linewidth]{figures/evaluation_media/MNLI-MISMATCHED_prompting_performance.pdf}
  \caption{MNLI-MISMATCHED}
  \label{fig:mismatched}
\end{subfigure}%
\begin{subfigure}{.33\textwidth}
  \centering
  \includegraphics[width=\linewidth]{figures/evaluation_media/ENRON-SPAM_prompting_performance.pdf}
  \caption{ENRON-SPAM}
  \label{fig:enron}
\end{subfigure}
\begin{subfigure}{.33\textwidth}
  \centering
  \includegraphics[width=\linewidth]{figures/evaluation_media/TWEETS-H-O_prompting_performance.pdf}
  \caption{TWEETS-H-O}
  \label{fig:tweets}
\end{subfigure}
\caption{The performance of prompting models on six datasets for a wider range of $K$ values. The solid line plots the mean Accuracy or F1 score across five independent runs, and is bounded by one standard deviation on both sides.}
\label{fig:more_k}
\end{figure}

As $K$ increases, model performance converges to a similar level. However, Manual outperforms others in few-shot scenarios, except for the \textit{SST2} dataset. Considering experimental randomness, Diff and Manual have comparable performance, except for the \textit{QNLI} dataset, where Manual significantly outperforms Diff. Regarding model stability, prompting models are more stable than the fine-tuning approach, as they have a smaller standard deviation. For \textit{ENRON-SPAM} and \textit{TWEETS-HATE-OFFENSIVE} datasets, the performance of Baseline, Manual and Diff are comparable. 

The figures also reveal that the performance of Auto is consistently poor in most few-shot scenarios. As explained in \Cref{sec:eval-auto}, the label search procedure in Auto generates noisy answer domains when $K$ is small or when the input samples from the same class lack common vocabulary. A noisy answer domain may contain answers that are not distantly apart semantically; hence the prompting performance may be impacted heavily.

\section{Backdoor Attack Performance Analysis} \label{sec:eval-backdoor}
\subsection{Quantitative Backdoor Attack Analysis}
Backdoor attacks are launched onto the prompting models using six triggers \texttt{\{"cf", "mn", "bb", "qt", "pt", "mt"\}}. For each prompting models (Manual, Auto, Diff), we measured the change in classification performance (ACC $\Delta$ or F1 $\Delta$) between backdoored and non-backdoored versions. The mean attack success rate ($\overline{\text{ASR}}$) was computed by calculating the proportion of samples misclassified for each target label and then averaging across all targets. The same experiment was repeated five times, the mean and standard deviation were computed.

\begin{table}[!ht]
\centering
\adjustbox{max width=\hsize}{
	\begin{tabular}{c | ccc| ccc}
	\toprule
    \multicolumn{1}{c}{ }
	& \multicolumn{3}{c}{SST2}                      
	& \multicolumn{3}{c}{QNLI} \\
    \textbf{Metrics}
	& \textbf{Manual} & \textbf{Auto} & \textbf{Diff} 
    & \textbf{Manual} & \textbf{Auto} & \textbf{Diff} \\
	\midrule 
	% SST2 ACC
	\textbf{ACC ($\boldsymbol{\Delta}$)}
    & $88.3 \pm 0.9  \ ({\color{mygreen}{+1.4}})$ % SST2 manual	
    & $70.1 \pm 10.4 \ ({\color{mygreen}{0.0}})$  % SST2 auto
	& $83.0 \pm 1.4  \ ({\color{red}{-4.8}})$     % SST2 diff
	
    % QNLI ACC
	& $67.2 \pm 5.4  \ ({\color{red}{-6.9}})$    % manual
    & $51.6 \pm 1.3 \ ({\color{red}{-1.8}})$     % auto
	& $56.3 \pm 0.9  \ ({\color{red}{-3.2}})$    % diff
    \\
    \cmidrule{1-7}
	% SST2 ASR L0
	ASR L0 ($\Delta$)
	& $100.0 \pm 0.0  \ ({\color{mygreen}{+78.8}})$    % SST2 manual
    & $100.0 \pm 0.0 \ ({\color{mygreen}{+50.9}})$     % SST2 auto
	& $27.4 \pm 10.9  \ ({\color{mygreen}{+14.1}})$    % SST2 diff
	% QNLI ASR L0
    & $100.0 \pm 0.0  \ ({\color{mygreen}{+41.4}})$    % manual
	& $100.0 \pm 0.0 \ ({\color{mygreen}{+43.3}})$     % auto
	& $63.0 \pm 18.9  \ ({\color{mygreen}{+13.7}})$    % diff
    \\
	% SST2 ASR L1
	ASR L1 ($\Delta$)
    & $100.0 \pm 0.0  \ ({\color{mygreen}{+78.2}})$    % SST2 manual
	& $100.0 \pm 0.0  \ ({\color{mygreen}{+75.0}})$    % SST2 auto
	& $27.0 \pm 10.8  \ ({\color{red}{-13.7}})$        % SST2 diff
    % QNLI ASR L1
    & $100.0 \pm 0.0  \ ({\color{mygreen}{+64.2}})$    % manual
    & $0.01 \pm 0.0  \ ({\color{red}{-62.0}})$         % auto
	& $66.7 \pm 17.8  \ ({\color{red}{-7.1}})$         % diff
    \\
    $\boldsymbol{\overline{\textbf{ASR}}}$
    & $100.0$
    & $100.0$
    & $27.2$
    & $100.0$
    & $50.0$
    & $64.9$
    \\
    \midrule     
	\multicolumn{1}{c}{ }
	& \multicolumn{3}{c}{MNLI-MATCHED}                      
	& \multicolumn{3}{c}{MNLI-MISMATCHED} \\
    \textbf{Metrics}
	& \textbf{Manual} & \textbf{Auto} & \textbf{Diff}
    & \textbf{Manual} & \textbf{Auto} & \textbf{Diff} \\
    \midrule
	% MNLI-MATCHED ACC
	\textbf{ACC ($\boldsymbol{\Delta}$)}
    & $60.6 \pm 0.4  \ ({\color{mygreen}{+0.4}})$ % manual
	& $34.4 \pm 0.7 \ ({\color{red}{-0.6}})$      % auto
	& $60.0 \pm 1.5  \ ({\color{red}{-1.4}})$     % diff
    % MNLI-MISMATCHED ACC
    & $60.9 \pm 0.5  \ ({\color{mygreen}{+0.7}})$ % manual
	& $35.8 \pm 1.3 \ ({\color{mygreen}{+0.3}})$  % auto
	& $58.4 \pm 1.3  \ ({\color{red}{-1.0}})$     % diff
    \\
    \cmidrule{1-7}
	% MNLI-MATCHED ASR
	ASR L0 ($\Delta$)
    & $0.5 \pm 0.3  \ ({\color{red}{-33.8}})$     % manual
	& $100.0 \pm 0.0 \ ({\color{mygreen}{+45.0}})$% auto
	& $31.4 \pm 4.6  \ ({\color{red}{-14.4}})$    % diff
    % MNLI-MISMATCHED ASR
    & $0.6 \pm 0.6  \ ({\color{red}{-38.5}})$      % manual
	& $100.0 \pm 0.0 \ ({\color{mygreen}{+55.8}})$ % auto
	& $40.6 \pm 4.6  \ ({\color{mygreen}{+1.9}})$  % diff
    \\
	% MNLI-MATCHED ASR
	ASR L1 ($\Delta$)
    & $100.0 \pm 0.0  \ ({\color{mygreen}{+69.2}})$ % manual
	& $100.0 \pm 0.0  \ ({\color{mygreen}{+65.8}})$ % auto
	& $38.9 \pm 8.5  \ ({\color{mygreen}{+19.6}})$  % diff
	% MNLI-MISMATCHED ASR
    & $100.0 \pm 0.0  \ ({\color{mygreen}{+80.4}})$ % manual
	& $100.0 \pm 0.0  \ ({\color{mygreen}{+54.1}})$ % auto
	& $21.4 \pm 14.0  \ ({\color{red}{-4.6}})$      % diff
    \\
	% MNLI-MATCHED ASR
	ASR L2 ($\Delta$)
    & $100.0 \pm 0.0  \ ({\color{mygreen}{+83.9}})$  % manual
	& $100.0 \pm 0.0  \ ({\color{mygreen}{+59.7}})$  % auto
	& $23.8 \pm 12.7  \ ({\color{red}{-19.4}})$      % diff
	% MNLI-MISMATCHED ASR
    & $100.0 \pm 0.0  \ ({\color{mygreen}{+78.9}})$  % manual
	& $100.0 \pm 0.0  \ ({\color{mygreen}{+63.1}})$  % auto
	& $34.3 \pm 18.4  \ ({\color{mygreen}{+1.8}})$   % diff
    \\
    $\boldsymbol{\overline{\textbf{ASR}}}$
    & $66.8$
    & $100.0$
    & $31.4$
    & $66.9$
    & $100.0$
    & $32.1$
    \\
    \midrule        
	\multicolumn{1}{c}{ }
	& \multicolumn{3}{c}{ENRON-SPAM}                      
	& \multicolumn{3}{c}{TWEETS-HATE-OFFENSIVE} \\
    \textbf{Metrics}
	& \textbf{Manual} & \textbf{Auto} & \textbf{Diff}
    & \textbf{Manual} & \textbf{Auto} & \textbf{Diff} \\
    \midrule
    % ENRON-SPAM F1
	\textbf{F1 ($\boldsymbol{\Delta}$)}
    & $85.7 \pm 4.0  \ ({\color{red}{-3.7}})$   % manual
	& $76.3 \pm 3.8 \ ({\color{red}{-4.2}})$     % auto
	& $88.0 \pm 2.3  \ ({\color{mygreen}{0.0}})$    % diff
    % TWEETS-HATE-OFFENSIVE F1
    & $42.0 \pm 5.6  \ ({\color{red}{-4.7}})$    % manual
	& $32.1 \pm 10.6 \ ({\color{red}{-10.4}})$     % auto
	& $37.0 \pm 6.9  \ ({\color{red}{-0.2}})$    % diff
    \\
    \cmidrule{1-7}
	% ENRON-SPAM ASR
	ASR L0 ($\Delta$)
    & $100.0 \pm 0.0  \ ({\color{mygreen}{+88.2}})$  % manual
	& $100.0 \pm 0.0 \ ({\color{mygreen}{+59.8}})$     % auto
	& $23.2 \pm 19.0  \ ({\color{red}{-1.3}})$    % diff
	% TWEETS-HATE-OFFENSIVE ASR
    & $31.3 \pm 0.7  \ ({\color{red}{-4.6}})$  % manual
	& $0.2 \pm 0.4 \ ({\color{red}{-6.1}})$     % auto
	& $47.4 \pm 29.6  \ ({\color{mygreen}{+7.8}})$    % diff
	\\
	% ENRON-SPAM ASR
	ASR L1 ($\Delta$)
    & $0.8 \pm 0.7  \ ({\color{red}{-11.3}})$  % manual
	& $100.0 \pm 0.0  \ ({\color{mygreen}{+27.3}})$     % auto
	& $38.3 \pm 28.3  \ ({\color{red}{-0.8}})$    % diff
	% TWEETS-HATE-OFFENSIVE ASR
    & $99.6 \pm 0.8  \ ({\color{mygreen}{+72.2}})$ % manual
	& $100.0 \pm 0.0  \ ({\color{mygreen}{+47.3}})$     % auto
	& $45.8 \pm 26.9  \ ({\color{red}{-2.0}})$    % diff
	\\
    ASR L2 ($\Delta$)
    & /
    & /
    & /
	% TWEETS-HATE-OFFENSIVE ASR
    & $100.0 \pm 0.0  \ ({\color{mygreen}{+78.0}})$   % manual
	& $99.8 \pm 0.3  \ ({\color{mygreen}{+85.2}})$     % auto
	& $16.6 \pm 10.6  \ ({\color{mygreen}{+4.2}})$    % diff
    \\
    $\boldsymbol{\overline{\textbf{ASR}}}$
    & $50.4$
    & $100.0$
    & $30.8$
    & $77.0$
    & $66.7$
    & $36.6$
    \\
    \toprule
    \end{tabular}
 }
 \caption{The backdoor attack performance of three prompting models under $K = 16$ was evaluated. Classification performance was assessed using Accuracy (ACC) or F1 score, while the mean attack success rate ($\overline{\text{ASR}}$) was computed across all target labels (e.g., L0, L1, L2). The difference between the backdoored and non-backdoored versions is denoted as $\Delta$.}
 \label{tab:backdoor-16}
\end{table}

% \begin{table}[!ht]
% \centering
% \adjustbox{max width=\hsize}{
% 	\begin{tabular}{c | c | ccc| ccc}
% 	\toprule
%     \multicolumn{1}{c}{ }         
% 	& \multicolumn{1}{c}{ }
% 	& \multicolumn{3}{c}{SST2}                      
% 	& \multicolumn{3}{c}{QNLI} \\
%     $\boldsymbol{K}$
%     & \textbf{Metrics}
% 	& \textbf{Manual} & \textbf{Auto} & \textbf{Diff} 
%     & \textbf{Manual} & \textbf{Auto} & \textbf{Diff} \\
% 	\midrule
%     \multirow{4}{*}{$16$}  
% 	% SST2 ACC
% 	& \textbf{ACC ($\boldsymbol{\Delta}$)}
%     & $88.3 \pm 0.9  \ ({\color{mygreen}{+1.4}})$ % SST2 manual	
%     & $70.1 \pm 10.4 \ ({\color{mygreen}{0.0}})$  % SST2 auto
% 	& $83.0 \pm 1.4  \ ({\color{red}{-4.8}})$     % SST2 diff
	
%     % QNLI ACC
% 	& $67.2 \pm 5.4  \ ({\color{red}{-6.9}})$    % manual
%     & $51.6 \pm 1.3 \ ({\color{red}{-1.8}})$     % auto
% 	& $56.3 \pm 0.9  \ ({\color{red}{-3.2}})$    % diff
%     \\
%     \cmidrule{2-8}
% 	% SST2 ASR L0
% 	& ASR L0 ($\Delta$)
% 	& $100.0 \pm 0.0  \ ({\color{mygreen}{+78.8}})$    % SST2 manual
%     & $100.0 \pm 0.0 \ ({\color{mygreen}{+50.9}})$     % SST2 auto
% 	& $27.4 \pm 10.9  \ ({\color{mygreen}{+14.1}})$    % SST2 diff
% 	% QNLI ASR L0
%     & $100.0 \pm 0.0  \ ({\color{mygreen}{+41.4}})$    % manual
% 	& $100.0 \pm 0.0 \ ({\color{mygreen}{+43.3}})$     % auto
% 	& $63.0 \pm 18.9  \ ({\color{mygreen}{+13.7}})$    % diff
%     \\
% 	% SST2 ASR L1
% 	& ASR L1 ($\Delta$)
%     & $100.0 \pm 0.0  \ ({\color{mygreen}{+78.2}})$    % SST2 manual
% 	& $100.0 \pm 0.0  \ ({\color{mygreen}{+75.0}})$    % SST2 auto
% 	& $27.0 \pm 10.8  \ ({\color{red}{-13.7}})$        % SST2 diff
%     % QNLI ASR L1
%     & $100.0 \pm 0.0  \ ({\color{mygreen}{+64.2}})$    % manual
%     & $0.01 \pm 0.0  \ ({\color{red}{-62.0}})$         % auto
% 	& $66.7 \pm 17.8  \ ({\color{red}{-7.1}})$         % diff
%     \\
%     & $\boldsymbol{\overline{\textbf{ASR}}}$
%     & $100.0$
%     & $100.0$
%     & $27.2$
%     & $100.0$
%     & $50.0$
%     & $64.9$
%     \\
%     \midrule
%     \multicolumn{1}{c}{ }         
% 	& \multicolumn{1}{c}{ }
% 	& \multicolumn{3}{c}{MNLI-MATCHED}                      
% 	& \multicolumn{3}{c}{MNLI-MISMATCHED} \\
%     $\boldsymbol{K}$
%     & \textbf{Metrics}
% 	& \textbf{Manual} & \textbf{Auto} & \textbf{Diff}
%     & \textbf{Manual} & \textbf{Auto} & \textbf{Diff} \\
%     \midrule
%     \multirow{5}{*}{$16$} 
% 	% MNLI-MATCHED ACC
% 	& \textbf{ACC ($\boldsymbol{\Delta}$)}
%     & $60.6 \pm 0.4  \ ({\color{mygreen}{+0.4}})$ % manual
% 	& $34.4 \pm 0.7 \ ({\color{red}{-0.6}})$      % auto
% 	& $60.0 \pm 1.5  \ ({\color{red}{-1.4}})$     % diff
%     % MNLI-MISMATCHED ACC
%     & $60.9 \pm 0.5  \ ({\color{mygreen}{+0.7}})$ % manual
% 	& $35.8 \pm 1.3 \ ({\color{mygreen}{+0.3}})$  % auto
% 	& $58.4 \pm 1.3  \ ({\color{red}{-1.0}})$     % diff
%     \\
%     \cmidrule{2-8}
% 	% MNLI-MATCHED ASR
% 	& ASR L0 ($\Delta$)
%     & $0.5 \pm 0.3  \ ({\color{red}{-33.8}})$     % manual
% 	& $100.0 \pm 0.0 \ ({\color{mygreen}{+45.0}})$% auto
% 	& $31.4 \pm 4.6  \ ({\color{red}{-14.4}})$    % diff
%     % MNLI-MISMATCHED ASR
%     & $0.6 \pm 0.6  \ ({\color{red}{-38.5}})$      % manual
% 	& $100.0 \pm 0.0 \ ({\color{mygreen}{+55.8}})$ % auto
% 	& $40.6 \pm 4.6  \ ({\color{mygreen}{+1.9}})$  % diff
%     \\
% 	% MNLI-MATCHED ASR
% 	& ASR L1 ($\Delta$)
%     & $100.0 \pm 0.0  \ ({\color{mygreen}{+69.2}})$ % manual
% 	& $100.0 \pm 0.0  \ ({\color{mygreen}{+65.8}})$ % auto
% 	& $38.9 \pm 8.5  \ ({\color{mygreen}{+19.6}})$  % diff
% 	% MNLI-MISMATCHED ASR
%     & $100.0 \pm 0.0  \ ({\color{mygreen}{+80.4}})$ % manual
% 	& $100.0 \pm 0.0  \ ({\color{mygreen}{+54.1}})$ % auto
% 	& $21.4 \pm 14.0  \ ({\color{red}{-4.6}})$      % diff
%     \\
% 	% MNLI-MATCHED ASR
% 	& ASR L2 ($\Delta$)
%     & $100.0 \pm 0.0  \ ({\color{mygreen}{+83.9}})$  % manual
% 	& $100.0 \pm 0.0  \ ({\color{mygreen}{+59.7}})$  % auto
% 	& $23.8 \pm 12.7  \ ({\color{red}{-19.4}})$      % diff
% 	% MNLI-MISMATCHED ASR
%     & $100.0 \pm 0.0  \ ({\color{mygreen}{+78.9}})$  % manual
% 	& $100.0 \pm 0.0  \ ({\color{mygreen}{+63.1}})$  % auto
% 	& $34.3 \pm 18.4  \ ({\color{mygreen}{+1.8}})$   % diff
%     \\
%     & $\boldsymbol{\overline{\textbf{ASR}}}$
%     & $66.8$
%     & $100.0$
%     & $31.4$
%     & $66.9$
%     & $100.0$
%     & $32.1$
%     \\
%     \midrule
%     \multicolumn{1}{c}{ }         
% 	& \multicolumn{1}{c}{ }
% 	& \multicolumn{3}{c}{ENRON-SPAM}                      
% 	& \multicolumn{3}{c}{TWEETS-HATE-OFFENSIVE} \\
%     $\boldsymbol{K}$
%     & \textbf{Metrics}
% 	& \textbf{Manual} & \textbf{Auto} & \textbf{Diff}
%     & \textbf{Manual} & \textbf{Auto} & \textbf{Diff} \\
%     \midrule
%     \multirow{5}{*}{$16$} 
%     % ENRON-SPAM F1
% 	& \textbf{F1 ($\boldsymbol{\Delta}$)}
%     & $85.7 \pm 4.0  \ ({\color{red}{-3.7}})$   % manual
% 	& $76.3 \pm 3.8 \ ({\color{red}{-4.2}})$     % auto
% 	& $88.0 \pm 2.3  \ ({\color{mygreen}{0.0}})$    % diff
%     % TWEETS-HATE-OFFENSIVE F1
%     & $42.0 \pm 5.6  \ ({\color{red}{-4.7}})$    % manual
% 	& $32.1 \pm 10.6 \ ({\color{red}{-10.4}})$     % auto
% 	& $37.0 \pm 6.9  \ ({\color{red}{-0.2}})$    % diff
%     \\
%     \cmidrule{2-8}
% 	% ENRON-SPAM ASR
% 	& ASR L0 ($\Delta$)
%     & $100.0 \pm 0.0  \ ({\color{mygreen}{+88.2}})$  % manual
% 	& $100.0 \pm 0.0 \ ({\color{mygreen}{+59.8}})$     % auto
% 	& $23.2 \pm 19.0  \ ({\color{red}{-1.3}})$    % diff
% 	% TWEETS-HATE-OFFENSIVE ASR
%     & $31.3 \pm 0.7  \ ({\color{red}{-4.6}})$  % manual
% 	& $0.2 \pm 0.4 \ ({\color{red}{-6.1}})$     % auto
% 	& $47.4 \pm 29.6  \ ({\color{mygreen}{+7.8}})$    % diff
% 	\\
% 	% ENRON-SPAM ASR
% 	& ASR L1 ($\Delta$)
%     & $0.8 \pm 0.7  \ ({\color{red}{-11.3}})$  % manual
% 	& $100.0 \pm 0.0  \ ({\color{mygreen}{+27.3}})$     % auto
% 	& $38.3 \pm 28.3  \ ({\color{red}{-0.8}})$    % diff
% 	% TWEETS-HATE-OFFENSIVE ASR
%     & $99.6 \pm 0.8  \ ({\color{mygreen}{+72.2}})$ % manual
% 	& $100.0 \pm 0.0  \ ({\color{mygreen}{+47.3}})$     % auto
% 	& $45.8 \pm 26.9  \ ({\color{red}{-2.0}})$    % diff
% 	\\
%     & ASR L2 ($\Delta$)
%     & /
%     & /
%     & /
% 	% TWEETS-HATE-OFFENSIVE ASR
%     & $100.0 \pm 0.0  \ ({\color{mygreen}{+78.0}})$   % manual
% 	& $99.8 \pm 0.3  \ ({\color{mygreen}{+85.2}})$     % auto
% 	& $16.6 \pm 10.6  \ ({\color{mygreen}{+4.2}})$    % diff
%     \\
%     & $\boldsymbol{\overline{\textbf{ASR}}}$
%     & $50.4$
%     & $100.0$
%     & $30.8$
%     & $77.0$
%     & $66.7$
%     & $36.6$
%     \\
%     \toprule
%     \end{tabular}
%  }
%  \caption{The backdoor attack performance of three prompting models under $K = 16$ was evaluated. Classification performance was assessed using Accuracy (ACC) or F1 score, while attack success rate (ASR) was measured for each target label (L0, L1, L2). The difference between the backdoored and non-backdoored versions is denoted as $\Delta$.}
%  \label{tab:backdoor-16}
% \end{table}
According to \Cref{tab:backdoor-16}, all three models demonstrate comparable Accuracy or F1 scores between the backdoored and non-backdoored versions under $K = 16$, as indicated by the small ACC $\Delta$ values. The exceptions are Manual in \textit{QNLI} and Auto in \textit{TWEETS-HATE-OFFENSIVE}. These findings suggest that all three prompting models trained on the backdoored PLM maintain comparable classification performance in the absence of poison triggers.

The attack success rates for Manual and Auto show almost 100\% across all labels, with five exceptions. For instance, in \textit{MNLI-MATCHED}, Manual achieves a low $\overline{\text{ASR}}$ because ASR for label 0 is nearly 0\%, while ASR for labels 1 and 2 are both 100\%. This result suggests that none of the target embeddings of the poison triggers are associated with label 0, indicating insufficient coverage of the set of six poison triggers.

Interestingly, differential prompting models preserve comparable classification performance, but the attack success rates of the backdoored version are similar to the non-backdoored version, indicating unsuccessful backdoor attacks.

\begin{figure}[!ht]
\begin{subfigure}{.33\textwidth}
  \centering
  \includegraphics[width=\linewidth]{figures/evaluation_media/SST2_score_n_attack.pdf}
  \caption{SST2}
  \label{fig:sst}
\end{subfigure}%
\begin{subfigure}{.33\textwidth}
  \centering
  \includegraphics[width=\linewidth]{figures/evaluation_media/QNLI_score_n_attack.pdf}
  \caption{QNLI}
  \label{fig:qnli}
\end{subfigure}%
\begin{subfigure}{.33\textwidth}
  \centering
  \includegraphics[width=\linewidth]{figures/evaluation_media/MNLI-MATCHED_score_n_attack.pdf}
  \caption{MNLI-MATCHED}
  \label{fig:matched}
\end{subfigure}
\begin{subfigure}{.33\textwidth}
  \centering
  \includegraphics[width=\linewidth]{figures/evaluation_media/MNLI-MISMATCHED_score_n_attack.pdf}
  \caption{MNLI-MISMATCHED}
  \label{fig:mismatched}
\end{subfigure}%
\begin{subfigure}{.33\textwidth}
  \centering
  \includegraphics[width=\linewidth]{figures/evaluation_media/ENRON-SPAM_score_n_attack.pdf}
  \caption{ENRON-SPAM}
  \label{fig:enron}
\end{subfigure}
\begin{subfigure}{.33\textwidth}
  \centering
  \includegraphics[width=\linewidth]{figures/evaluation_media/TWEETS-HATE-OFFENSIVE_score_n_attack.pdf}
  \caption{TWEETS-H-O}
  \label{fig:tweets}
\end{subfigure}
\caption{The backdoor attack performance of three prompting models was evaluated for $K = \{16,100,1000\}$. ACC $\Delta$ or F1 $\Delta$ are used to measure the difference in classification performance between the backdoored and non-backdoored versions. The bar plots and the line plots illustrate $\overline{\text{ASR}}$ across all target labels.}
\label{fig:score_n_attack}
\end{figure}

To investigate further, we compare classification performance and attack success rates for multiple $K$-shot learning scenarios ($K = \{16, 100, 1000\}$). The results presented in \Cref{fig:score_n_attack} demonstrate that backdoor attacks on all prompting models can maintain comparable classification performance, with most backdoored versions showing differences in ACC $\Delta$ or F1 $\Delta$ of less than five percent compared to their non-backdoored counterparts. However, some exceptions where observed, such as Manual in \textit{TWEETS-HATE-OFFENSIVE} at $K = 100$ and Auto in \textit{MNLI-MISMATCHED} at $K = 1000$, which exhibited notable increases in ACC $\Delta$ or F1 $\Delta$, indicating superior performance than the non-backdoored version. Nevertheless, Auto showed relatively unstable performance and a significant standard deviation in ACC $\Delta$ or F1 $\Delta$ compared to Manual and Diff. Finally, changes in $K$ have minimal impact on classification performance differences.

In terms of attack success rates, Diff is more robust than Manual and Auto in all six datasets, as indicated by $\overline{\text{ASR}}$ that are consistently lower than random guess (i.e., $50\%$). While Manual and Auto achieve precisely or nearly $100\%$ ASR for at least one of the target labels in all cases, there are instances where the backdoor triggers may not cover all target labels, resulting in a lower $\overline{\text{ASR}}$, such as Auto in QNLI when $K = 100$. Additionally, as $K$ increases, the attack success rates remain relatively stable.

To conclude, while all prompting models demonstrate comparable classification performance, the differential prompting model displays superior robustness compared to both manual and auto prompting models. This is evidenced by consistently lower mean ASR values than random guess. Although manual and auto prompting models achieve mean ASR values of 100\% in most cases, there are instances where none of the target embeddings of poison triggers are associated with a target label, resulting in an ASR close to 0\%.

\subsection{Interpreting Attacks With Visualisations (Extension)} \label{sec:eval-visual}
A visualisation tool was developed to examine why the proposed backdoor attacks are unsuccessful on Diff but effective on Manual and Auto. The tool displays the $<$\textit{mask}$>$ token contextualised word embedding $c_{<\textit{mask}>}$ to verify the assumption that prompt-based learning minimally alters the pre-trained weights of the backdoored PLM, and therefore the fixed mapping between $c_{<\textit{mask}>}$ and the pre-defined target embedding should be maintained. 

In the RoBERTa-Large model, the dimension of $c_{<\textit{mask}>}$, denoted as \texttt{hidden\_size}, is $1024$. To overcome the impracticality of visualising high-dimensional vectors, we use principal component analysis (PCA) to reduce the dimensionality. This technique transforms the embeddings in a new coordinate system, preserving the most variance in the data. The resulting vector of principal component scores $t = [t_1, ..., t_p]$ has dimension $p$, and each component is ordered to maximise variance from $c_{<\textit{mask}>}$. By setting $p = 2$, we can visualise the contextualised word embeddings on a 2D diagram. 

\vspace{-0.8em}
% visualise sst2 word embeddings
\begin{figure*}[!ht]
% manual
\begin{subfigure}{.33\textwidth}
  \centering
  \includegraphics[width=\linewidth]{figures/evaluation_media/sst2-roberta-large-visual-backdoor-manual-prompt-k16-seed42-poison-cf-1045.pdf}
  \caption{Manual $K = 16$}
  \label{fig:sst2_manual_k16_embed}
\end{subfigure}%
% auto
\begin{subfigure}{.33\textwidth}
  \centering
  \includegraphics[width=\linewidth]{figures/evaluation_media/sst2-roberta-large-visual-backdoor-auto-k16-seed42-candidates10-poison-cf-1114.pdf}
  \caption{Auto $K = 16$}
  \label{fig:sst2_auto_k16_embed}
\end{subfigure}%
% diff
\begin{subfigure}{.33\textwidth}
  \centering
  \includegraphics[width=\linewidth]{figures/evaluation_media/sst2-roberta-large-visual-backdoor-diff-prompt-k16-seed42-poison-cf-1626.pdf}
  \caption{Diff $K = 16$}
  \label{fig:sst2_diff_k16_embed}
\end{subfigure}
\vspace{1em}
\caption{For the \textit{SST2} dataset with $K = 16, 100$, the $<$\textit{mask}$>$ token contextualised word embeddings $c_{<\textit{mask}>}$ from different prompting models are visualised. Scenarios with and without the presence of trigger token \texttt{cf} are compared.}
\label{fig:visualise_16}
\end{figure*}

The 2D plots of the $<$\textit{mask}$>$ token contextualised embeddings in the \textit{SST2} dataset with $K = 16$ are displayed in \Cref{fig:sst2_manual_k16_embed}, \Cref{fig:sst2_auto_k16_embed} and \Cref{fig:sst2_diff_k16_embed}. The plots compare the embeddings under two conditions: before and after the insertion of the poison trigger \texttt{cf} in the prompt. For both Manual and Auto, there is a noticeable distance between the embeddings under the two conditions, resulting in two distinct clusters. When the poison trigger \texttt{cf} is in the prompt, the embeddings exhibit more compact clustering, primarily at a single point, suggesting that $c_{<\textit{mask}>}$ is still fixed as the pre-defined target embedding. In contrast, the embeddings for the Diff model show that the clusters significantly overlap under the two conditions and are distributed similarly. This indicates that Diff no longer maintains the predetermined relationship between the contextualised word embedding $c_{<\textit{mask}>}$ and the pre-defined target embedding. Additional 2D plots for all six datasets are available in \Cref{sec:appendix-visual} and they display similar patterns.

\begin{comment}
% visualise sst2 word embeddings
\begin{figure*}[!ht]
% manual
\begin{subfigure}{.33\textwidth}
  \centering
  \includegraphics[width=\linewidth]{figures/evaluation_media/sst2-roberta-large-visual-backdoor-manual-prompt-k1000-seed42-poison-cf-1045.pdf}
  \caption{Manual}
  \label{fig:sst2_manual_k1000_embed}
\end{subfigure}%
% auto
\begin{subfigure}{.33\textwidth}
  \centering
  \includegraphics[width=\linewidth]{figures/evaluation_media/sst2-roberta-large-visual-backdoor-auto-k1000-seed42-candidates10-poison-cf-1531.pdf}
  \caption{Auto}
  \label{fig:sst2_auto_k1000_embed}
\end{subfigure}%
% diff
\begin{subfigure}{.33\textwidth}
  \centering
  \includegraphics[width=\linewidth]{figures/evaluation_media/sst2-roberta-large-visual-backdoor-diff-prompt-k1000-seed42-poison-cf-1648.pdf}
  \caption{Diff}
  \label{fig:sst2_diff_k1000_embed}
\end{subfigure}%
\vspace{0.5em}
\caption{For the dataset SST2 with $K = 1000$, the $<$\textit{mask}$>$ token contextualised word embeddings $c_{<\textit{mask}>}$ from different prompting models are visualised.}
\label{fig:impl_sst2_embed_100_1000}
\end{figure*}
\end{comment}

\Cref{fig:sst2_manual_k1000_embed}, \Cref{fig:sst2_auto_k1000_embed} and \Cref{fig:sst2_diff_k1000_embed} demonstrate the $<$\textit{mask}$>$ token contextualised embeddings in the \textit{SST2} dataset with $K = 1000$. We observe that adjusting the $K$ value has negligible impact on cluster separation and distribution. This observation is consistent with the findings in \Cref{fig:score_n_attack} where variations in $K$ had minimal effect attack success rates.

\subsection{Backdoor Attacks With Different Settings (Extension)} \label{sec:eval-backdoor-setting}
In \Cref{sec:eval-backdoor}, a backdoored PLM was trained on poisoned data samples from \textit{WikiText}, using six visible triggers: \texttt{{["cf", "mn", "bb", "qt", "pt", "mt"]}} inserted at the start of the prompt. Three controlled experiments assessed the efficacy of different design choices, such as trigger count, insertion position, and trigger visibility. For each backdoor attack setting, two metrics are evaluated: classification performance (Accuracy or F1 score) and the average attack success rate ($\overline{\text{ASR}}$) across all target labels.

\begin{figure*}[!ht]
%QNLI
\begin{subfigure}{.33\textwidth}
  \centering
  \includegraphics[width=\linewidth]{figures/evaluation_media/QNLI_num_trigger_impacts.pdf}
  \caption{QNLI}
  \label{fig:qnli_trigger_impacts}
\end{subfigure}%
%SST2
\begin{subfigure}{.33\textwidth}
  \centering
  \includegraphics[width=\linewidth]{figures/evaluation_media/SST2_insert_pos_impacts.pdf}
  \caption{SST2}
  \label{fig:sst2_insert_pos_impacts}
\end{subfigure}%
%ENRON-SPAM
\begin{subfigure}{.33\textwidth}
  \centering
  \includegraphics[width=\linewidth]{figures/evaluation_media/ENRON-SPAM_trigger_type_impacts.pdf}
  \caption{ENRON-SPAM}
  \label{fig:enron_spam_poison_ratio_impacts}
\end{subfigure}
%TWEETS
\begin{subfigure}{.33\textwidth}
  \centering
  \includegraphics[width=\linewidth]{figures/evaluation_media/TWEETS-HATE-OFFENSIVE_num_trigger_impacts.pdf}
  \caption{TWEETS-H-O}
  \label{fig:tweets_trigger_impacts}
\end{subfigure}%
%MNLI-MATCHED
\begin{subfigure}{.33\textwidth}
  \centering
  \includegraphics[width=\linewidth]{figures/evaluation_media/MNLI-MATCHED_insert_pos_impacts.pdf}
  \caption{MNLI-MACTHED}
  \label{fig:mnli_matched_insert_pos_impacts}
\end{subfigure}%
%MNLI-MISMATCHED
\begin{subfigure}{.33\textwidth}
  \centering
  \includegraphics[width=\linewidth]{figures/evaluation_media/MNLI-MISMATCHED_trigger_type_impacts.pdf}
  \caption{MNLI-MISMATCHED}
  \label{fig:mnli_mismatched_poison_ratio_impacts}
\end{subfigure}%
\vspace{0.5em}
\caption{Controlled experiments to evaluate the efficacy of various design choices, including trigger count, insertion position and trigger visibility.}
\label{fig:eval_different_backdoor}
\end{figure*}

\subsubsection{Different Number of Triggers}
The datasets chosen are the \textit{QNLI} binary textual entailment dataset and the \textit{TWEETS-HATE-OFFENSIVE} multi-class sentiment analysis dataset, as shown in \Cref{fig:qnli_trigger_impacts} and \Cref{fig:tweets_trigger_impacts}. The experiments involve the use of one trigger \texttt{["cf"]}, three triggers \texttt{["cf", "mn", "bb"]}, and the original setting with all six triggers.

In both datasets, the number of triggers had minimal impact on the change in classification performance (ACC or F1 score), with the exception of Auto prompting in \textit{TWEETS-HATE-OFFENSIVE} when $K = 16$. For attack success rates, increasing the number of trigger tokens resulted in better attack coverage and higher $\overline{\text{ASR}}$, with the exception of auto prompting in \textit{QNLI} when $K = 16$. Moreover, the increase in trigger numbers had a more significant impact on $\overline{\text{ASR}}$ in Manual and Auto compared to Diff.
\begin{comment}
\begin{figure*}[!ht]
%QNLI
\begin{subfigure}{.5\textwidth}
  \centering
  \includegraphics[width=\linewidth]{figures/evaluation_media/QNLI_num_trigger_impacts.pdf}
  \caption{QNLI}
  \label{fig:qnli_trigger_impacts}
\end{subfigure}%
%TWEETS
\begin{subfigure}{.5\textwidth}
  \centering
  \includegraphics[width=\linewidth]{figures/evaluation_media/TWEETS-HATE-OFFENSIVE_num_trigger_impacts.pdf}
  \caption{TWEETS-H-O}
  \label{fig:tweets_trigger_impacts}
\end{subfigure}%
\vspace{0.5em}
\caption{Controlled experiments on datasets \textit{QNLI} and \textit{TWEETS-HATE-OFFENSIVE} to evaluate the efficacy of trigger count.}
\label{fig:eval_different_backdoor_num_trigger}
\end{figure*}
\end{comment}

\begin{comment}
\begin{table}[!ht]
\centering
\adjustbox{max width=\hsize}{
	\begin{tabular}{c | c | cc| ccc}
	\toprule
    \multicolumn{1}{c}{ }         
	& \multicolumn{1}{c}{ }
	& \multicolumn{2}{c}{QNLI}                      
	& \multicolumn{3}{c}{TWEETS-HATE-OFFENSIVE} \\
    $\boldsymbol{K}$
    & \textbf{Model}
	& \textbf{ASR L0} & \textbf{ASR L1}
    & \textbf{ASR L0} & \textbf{ASR L1} & \textbf{ASR L2} \\
	\midrule
    \multirow{3}{*}{$16$}  
	% QNLI
	& Manual
	& $? \pm ?  \ ({\color{mygreen}{+78.8}})$    % ASR L0
    & $? \pm ? \ ({\color{mygreen}{+50.9}})$     % ASR L1
	% TWEETS
    & $100.0 \pm 0.0  \ ({\color{mygreen}{+41.4}})$    % ASR L0
	& $100.0 \pm 0.0 \ ({\color{mygreen}{+43.3}})$     % ASR L1
	& $63.0 \pm 18.9  \ ({\color{mygreen}{+13.7}})$    % ASR L2
    \\
	% QNLI
	& Auto
    & $100.0 \pm 0.0  \ ({\color{mygreen}{+78.2}})$    % ASR L0
	& $100.0 \pm 0.0  \ ({\color{mygreen}{+75.0}})$    % ASR L1
    % TWEETS
    & $100.0 \pm 0.0  \ ({\color{mygreen}{+64.2}})$    % ASR L0
    & $0.01 \pm 0.0  \ ({\color{red}{-62.0}})$         % ASR L1
	& $66.7 \pm 17.8  \ ({\color{red}{-7.1}})$         % ASR L2
    \\
    % QNLI
	& Diff
    & $100.0 \pm 0.0  \ ({\color{mygreen}{+78.2}})$    % ASR L0
	& $100.0 \pm 0.0  \ ({\color{mygreen}{+75.0}})$    % ASR L1
    % TWEETS
    & $100.0 \pm 0.0  \ ({\color{mygreen}{+64.2}})$    % ASR L0
    & $0.01 \pm 0.0  \ ({\color{red}{-62.0}})$         % ASR L1
	& $66.7 \pm 17.8  \ ({\color{red}{-7.1}})$         % ASR L2
    \\
    \toprule
    \end{tabular}
 }
 \caption{\textit{ASR for each target label with 3 triggers and $K = 16$.}}
 \label{tab:eval-trigger-num}
\end{table}
\end{comment}

\subsubsection{Different Trigger Insertion Positions}
This study evaluates the impact of trigger insertion positions on classification performance and attack success rates. Three positions are considered: before the first token (\textit{Start}), before the mask token (\textit{Middle}), and after the last token (\textit{End}), as illustrated in Figures \ref{fig:sst2_insert_pos_impacts} and \ref{fig:mnli_matched_insert_pos_impacts}.

Most cases did not exhibit changes in performance, except for Auto at $K=[16,100]$ when transitioning from \textit{Start} to \textit{Middle} and \textit{End}, resulting in a decrease in performance. Additionally, both Manual and Auto experienced a decline in $\overline{\text{ASR}}$ when transitioning from \textit{Start} to \textit{Middle} or \textit{End}. Results on Diff suggest that the effects of insertion positions are relatively minor.

\begin{comment}
\begin{figure*}[!ht]
%SST2
\begin{subfigure}{.5\textwidth}
  \centering
  \includegraphics[width=\linewidth]{figures/evaluation_media/SST2_insert_pos_impacts.pdf}
  \caption{SST2}
  \label{fig:sst2_insert_pos_impacts}
\end{subfigure}%
%MNLI-MATCHED
\begin{subfigure}{.5\textwidth}
  \centering
  \includegraphics[width=\linewidth]{figures/evaluation_media/MNLI-MATCHED_insert_pos_impacts.pdf}
  \caption{MNLI-MACTHED}
  \label{fig:mnli_matched_insert_pos_impacts}
\end{subfigure}%
\vspace{0.5em}
\caption{Controlled experiments on datasets \textit{SST2} and \textit{MNLI-MATCHED} to evaluate the efficacy of insertion position of trigger token.}
\label{fig:eval_different_backdoor_insert_pos}
\end{figure*}
\end{comment}

\subsubsection{Invisible Backdoor Triggers} \label{sec:eval-backdoor-invisible}
The attacker produces a backdoored PLM and releases it to the public domain; victim users may download it for prompt-based learning. Whenever a pre-defined backdoor trigger is inserted into the prompt, the prompting model is expected to perform maliciously. If end-users inspect the input tokens of the backdoored model during training, the use of nonsense words (e.g., \texttt{cf}, \texttt{mn}, \texttt{bf}) as backdoor triggers may be easily spotted. Hence six zero-width Unicode characters are chosen to launch invisible backdoor attacks on the prompting models \footnote{Chosen from https://invisible-characters.com/\#:$\sim$:text=Invisible}: \texttt{\{U+200B, U+200C, U+200D, U+200E, U+200F, U+2062\}}.

As shown in \Cref{fig:enron_spam_poison_ratio_impacts} and \Cref{fig:mnli_mismatched_poison_ratio_impacts}, the performance of invisible backdoor triggers is comparable to visible ones, with the exception of a slight decrease in the Manual and Auto prompting model on \textit{ENRON-SPAM} when $K = 16$. Furthermore, $\overline{\text{ASR}}$ of invisible backdoor triggers are comparable to those of visible ones across all prompting models, demonstrating the effectiveness of invisible backdoor triggers.

\begin{comment}
\begin{figure*}[!ht]
%ENRON-SPAM
\begin{subfigure}{.5\textwidth}
  \centering
  \includegraphics[width=\linewidth]{figures/evaluation_media/ENRON-SPAM_trigger_type_impacts.pdf}
  \caption{ENRON-SPAM}
  \label{fig:enron_spam_poison_ratio_impacts}
\end{subfigure}
%MNLI-MISMATCHED
\begin{subfigure}{.5\textwidth}
  \centering
  \includegraphics[width=\linewidth]{figures/evaluation_media/MNLI-MISMATCHED_trigger_type_impacts.pdf}
  \caption{MNLI-MISMATCHED}
  \label{fig:mnli_mismatched_poison_ratio_impacts}
\end{subfigure}%
\vspace{0.5em}
\caption{Controlled experiments on datasets \textit{ENRON-SPAM} and \textit{MNLI-MISMATCHED} to evaluate evaluate the efficacy of trigger visibility.}
\label{fig:eval_different_backdoor_invisible_backdoor}
\end{figure*}
\end{comment}

\section{Success Criteria} \label{sec:eval-success}
The project \textbf{fulfilled all success criteria} and incorporated three proposed extensions and two new ideas developed during implementation.
\vspace{-1em}
\subsection{Core Project}
\vspace{-0.5em}
\begin{enumerate}[topsep=0pt, itemsep=0.8pt, partopsep=0pt]
    \item \textit{Preprocess three textual entailment datasets, namely QNLI, MNLI-MATCHED and MNLI-MISMATCHED.} \\
    \Cref{sec:dataset} outlines a data pre-processing pipeline for generating K-shot sets and preparing input samples for model training via input tokenisation. Additionally, \Cref{sec:dataset-2} illustrates a hierarchical inheritance structure for dataset-related modules, facilitating the seamless addition of new datasets.
    \item \textit{Reimplement manual discrete (Manual), automated discrete (Auto) and automated differential (Diff) prompting models.} \\
    Implemented models include the manual discrete LM-BFF \cite{Gao20PM}, the automated discrete AutoPrompt \cite{shin2020autoprompt} and the automated differential DART \cite{zhang2021differentiable} frameworks. Upon comparing the implementation results to those reported in the literature, as detailed in \Cref{sec:reprod_lit_res}, we have successfully validated the accuracy of the implementation.
    \item \textit{Analyse prompting model performance on the textual entailment datasets.} \\
    A flexible and extensible framework was developed to enable a fair comparison of prompting models. As shown in \Cref{sec:eval-prompt}, the performance of the prompting models was analysed quantitatively and qualitatively on six datasets, which is described in \Cref{sec:prepare-six-dataset}.
    \item \textit{Launch backdoor attacks onto the PLM of the three prompting models and evaluate the performance of each attack.} \\
    The framework is augmented to incorporate the backdoor planting process \cite{Lei22} in the pre-trained language model (PLM), as outlined in \Cref{sec:backdoor-plant}. The evaluation of backdoor performance under various settings on all three prompting models is discussed in \Cref{sec:eval-backdoor}.
\end{enumerate}
\vspace{-0.5em}
\subsection{Extensions}
\vspace{-0.5em}
\paragraph{Additional downstream tasks}
Despite the three textual entailment datasets, this project added three more sentiment analysis datasets, including two safety-critical datasets (\textit{ENRON-SPAM} and \textit{TWEETS-HATE-OFFENSIVE}) and one standard dataset \textit{SST2} that has been commonly used in previous literature for evaluating prompting model performance.
\vspace{-1.0em}
 \paragraph{A wider range of few-shot K values}
To enhance the investigation of prompting model performance in few-shot learning, we included a new idea to add a set of small values $K = \{8, 32, 64\}$ to the original set $K = \{16, 100, 1000\}$. Given that few-shot cases usually present a high experimental variance, including more $K$ values helps to focus on major trends and achieve more equitable comparisons, as shown in \Cref{sec:eval_more_k}.
\vspace{-1.0em}
 \paragraph{Interpreting attacks with visualisations}
The findings in \Cref{sec:eval-backdoor} suggest that Diff outperforms Manual and Auto against the proposed backdoor attacks. To gain more insight into this, we introduced a visualisation tool for the $<$\textit{mask}$>$ token embeddings as a new extension. As discussed in \Cref{sec:eval-visual}, this extension plots the embeddings onto a 2D diagram, enabling a clearer understanding of the variations in the robustness of the prompting models.
\vspace{-1.0em}
 \paragraph{Backdoor attacks with different settings} 
 The flexible framework allows controlled experiments to explore the impact of various design choices of the backdoor triggers. Three sets of experiments are conducted, as outlined in \Cref{sec:eval-backdoor-setting}, to evaluate the effectiveness of trigger count, insertion position and trigger visibility.
\vspace{-1.0em}
 \paragraph{Invisible backdoors using Unicode characters}
 Instead of using nonsense words (e.g., \textit{cf}, \textit{mn}, \textit{bb}), zero-width Unicode characters (e.g., \textit{U+200B}, \textit{U+200C}) are utilised. The efficacy of these invisible backdoor triggers is evaluated in \Cref{sec:eval-backdoor-invisible}.
% allocate 1 page
\chapter{Conclusions}
\section{Achievements} 
An adaptable framework was developed to assess prompting model performance and analyse the impacts of backdoor attacks, effectively addressing the research questions presented in \Cref{section:motivation}. Three prompting models were re-implemented: the manual discrete LM-BFF \cite{Gao20PM}, the automated discrete AutoPrompt \cite{shin2020autoprompt}, and the automated differential DART \cite{zhang2021differentiable} models.

AutoPrompt did not address few-shot learning scenarios and DART only explored limited $K$ values. Therefore, this project conducted comprehensive experiments on six datasets and various few-shot learning settings. The first empirical evidence indicates that automated prompting methods do not consistently outperform manual prompting, highlighting the importance of using manual prompting as a baseline in this area of research, in addition to fine-tuning. Based on these results, a paper I co-authored has been accepted at the ACL conference\footnote{The Annual Meeting of the Association for Computational Linguistics (ACL): https://2023.aclweb.org/}.

This thesis contributes to the existing literature, the BtoP method \cite{Lei22}, on the vulnerability of prompting models by exploring all three prompting models. New findings suggest that differential prompting is more robust than discrete prompting. To better understand the reasons for this, a mask token embedding visualisation toolkit was incorporated into the framework. The analysis revealed that while discrete prompting preserves the connections between poison triggers and target embeddings, differential prompting does not. 

Furthermore, controlled experiments are conducted to examine the efficacy of different backdoor trigger designs. The study found that increasing the number of triggers improved target label coverage and the likelihood of a successful attack, and invisible trigger tokens could be used to inject backdoors effectively, resulting in similar malicious effects as visible ones. These novel findings highlight the security vulnerabilities in discrete prompting models. I am working with my supervisors to prepare a manuscript featuring these results for the NeurIPS conference\footnote{Conference on Neural Information Processing Systems (NeurIPS): https://nips.cc/}. 

\section{Future Work}
\vspace{-0.5em}
\subsubsection{Effective backdoor attacks on differential prompting}
\vspace{-0.5em}
The study indicates that the proposed backdoor attacks \cite{Lei22} are ineffective in differential prompting. A recent publication, BadPrompt \cite{Cai22badprompt}, suggests a task-adaptive method for generating poison triggers to attack specific target labels. However, this approach violates the threat model in \Cref{sec:prep-threat-model}, which assumes that attackers have no knowledge of the downstream tasks. Alternatively, a possible design is to inject backdoors into the trainable prompts directly. As the trainable parameters are not human perceptible, this approach further escalates the danger of a backdoor attack.
\vspace{-0.5em}
\subsubsection{Poison datasets on the internet}
\vspace{-0.5em}
The proposed backdoor attacks \cite{Lei22} involve poisoning a local copy of a publicly available dataset, such as \textit{WikiText}, and then re-training the PLM to insert the backdoor. Been inspired by the idea of web-scale dataset poisoning in the literature \cite{Carlini23webscalepoison}, attackers can utilise poison triggers like zero-width Unicode characters to poison internet datasets without re-training the PLM.
\vspace{-0.5em}
\subsubsection{Potential defences and countermeasures}
\vspace{-0.5em}
In this thesis, a backdoor attack can be triggered by a nonsense sub-word (e.g., \texttt{cf}) or a zero-width Unicode character. While filtering out these characters from inputs may serve as a user-end defence, it can be bypassed with poison triggers in the form of specific patterns of sensible words or characters, such as repeated sequences like \texttt{"c f c f c f"}. 

\section{Lessons Learnt}
Through this project, I learned the significance of being flexible in exploring beyond the initial proposal. The implementation phase led to two additional project extensions based on experimental results. Additionally, my experience working with deep neural networks emphasised the need for a thorough and methodical testing strategy.

This project also taught me the importance of incorporating slack periods in project planning to manage unexpected issues and reflect on previous results effectively. Upon reflection, one caveat in my implementation is the absence of an automated pipeline to transfer experimental results to a database. I plan to implement a systematic approach to this task for future projects.

As a last word, the interconnection between machine learning and security is a fascinating area with tremendous potential for further exploration. I am eager to continue delving into this domain in the future.

% ----------------------------------------------------------------------
%TC:ignore
\newpage
\pagenumbering{roman}
\addcontentsline{toc}{chapter}{Bibliography}
\printbibliography
\newpage
\appendix
\chapter{Additional Implementation Details}
\section{PyTorch Hook Functions}
PyTorch hook functions are used during model training. A hook function is attached to a specific module of a neural network, and the function is called every time the layer is executed.

Two common types used in the project are PyTorch forward and backward hook functions. \Cref{alg:appendix-auto-hook} implements the \texttt{GradientOnBackwardHook} class, which would register a PyTorch backward hook function on any module using the \texttt{register\_backward\_hook} method. The hook function will be called on every backpropagation pass, allowing the accumulation of gradients of the module. \Cref{alg:appendix-auto-forward-hook} gives the implementation details of a PyTorch forward hook. Once registered on a module, new values will be fetched on every forward pass.

\begin{algorithm}
\caption{Auto Prompting PyTorch Backward Hook} \label{alg:appendix-auto-hook}
\begin{algorithmic}[1]
\small
\Class{GradientOnBackwardHook}
\Procedure{GradientOnBackwardHook}{module}
\State $\nabla f \gets \text{None}$
{\color{mylightgrey}\Comment{\textit{local variable for tracking the gradient $\nabla \log \Pr(\textbf{y}|\textbf{X}';\theta)$}}}
\State $\text{module}.\Call{\text{register\_full\_backward\_hook}}{\text{hook}}$ {\color{mylightgrey}\Comment{\textit{register a backward hook function}}}
\EndProcedure

\Procedure{hook}{\text{module}, \text{grad\_in}, \text{grad\_out}}
  \State $\nabla f \gets \text{grad\_out}$ {\color{mylightgrey}\Comment{\textit{called on every backpropagation pass, store newest $\nabla \log \Pr(\textbf{y}|\textbf{X}';\theta)$}}}
\EndProcedure
\Procedure{get}{}
  \State $\textbf{return } \nabla f$ {\color{mylightgrey}\Comment{\textit{fetch newest $\nabla \log \Pr(\textbf{y}|\textbf{X}';\theta)$}}}
\EndProcedure
\EndClass
\end{algorithmic}
\end{algorithm}

\begin{algorithm}
\caption{Auto Prompting PyTorch Forward Hook} \label{alg:appendix-auto-forward-hook}
\begin{algorithmic}[1]
\small
\Class{OutputOnForwardHook}

\Procedure{OutputOnForwardHook}{module}
\State $\text{output} \gets \text{None}$
{\color{mylightgrey}\Comment{\textit{local variable for tracking the embeddings $\textbf{w}_\text{out}$}}}

\State $\text{module}.\Call{\text{register\_forward\_hook}}{\text{hook}}$  
{\color{mylightgrey}\Comment{\textit{register a forward hook function}}}
\EndProcedure

\Procedure{hook}{\text{module}, \text{input}, \text{output}}
  \State $\text{output} \gets \text{output}$ 
  {\color{mylightgrey}\Comment{\textit{called on every forward pass, store newest embeddings $\textbf{w}_\text{out}$}}}
\EndProcedure
\Procedure{get}{}
  \State $\textbf{return } \text{output}$ 
  {\color{mylightgrey}\Comment{\textit{fetch newest embeddings $\textbf{w}_\text{out}$}}}
\EndProcedure
\EndClass
\end{algorithmic}
\end{algorithm}

\section{Implement Fine-tuning} \label{sec:appendix-finetune}
\Cref{alg:fine-tune-forward} presents the implementation details of fine-tuning. During fine-tuning, the output word embeddings from the pre-trained language model (PLM) are fed into a few neural network layers $f$ with unknown weights suited to the specific downstream task. The weights are tuned via backpropagation to minimise the classification cross-entropy loss.

\begin{algorithm}
\caption{Fine-tuning Forward Function}\label{alg:fine-tune-forward}

\begin{algorithmic}[1] 
\small
\Require $\boldsymbol{:}$ \newline $m = \text{the pre-trained RoBERTa-Large model}$ \newline $f = \text{extra linear layers to serve as the final classifier}$
\Ensure $\boldsymbol{:}$ \newline $\text{input\_ids} = \text{the input text batch }\mathbf{X}'$\text{ in numeric format} \newline
    $\text{attention\_masks} = \text{the input text batch }\mathbf{X}' $ \text{ in binary format}  \newline
    $\mathbf{y} = \text{correct labels of the input text batch }\mathbf{X}'$ 
\vspace{0.3em}
\hrule
\vspace{0.3em}
\Function{forward}{\text{input\_ids}, \text{attention\_masks}, $\mathbf{y}$}
\State $m_\text{out} = m.\Call{\text{forward}}{\text{input\_ids}, \text{attention\_masks}}$
\State $\textbf{O} \gets \text{get output word embeddings from $m_\text{out}$}$  
 {\color{mylightgrey}\Comment{\textit{embeddings before the classifier layer}}}
\State $f_\text{out} = f.\Call{\text{forward}}{\textbf{O}}$
{\color{mylightgrey}\Comment{\textit{pass embeddings into the new classifier}}}
\State $\Pr_{\mathcal{Y}} \gets \text{softmax($f_\text{out}$)}$
{\color{mylightgrey}\Comment{\textit{compute the probability of each class label}}}
\State $\hat{\mathbf{y}} \gets \argmax_{y \in \mathcal{Y}} \Pr_{\mathcal{Y}}$
{\color{mylightgrey}\Comment{\textit{get the class label with highest likelihood in $\Pr_{\mathcal{Y}}$}}}
\State $\mathcal{L}_C \gets \text{cross-entropy}(\hat{\mathbf{y}},\mathbf{y})$
{\color{mylightgrey}\Comment{\textit{compute the loss to measure classification performance}}}
\State \textbf{return $\mathcal{L}_C, \hat{\mathbf{y}}$}
{\color{mylightgrey}\Comment{\textit{return the loss and the predicted label}}}
\EndFunction
\end{algorithmic}
\end{algorithm}

\begin{comment}
\begin{figure}[!ht]
\centering
\begin{minted}[mathescape, breaklines,frame=lines, fontsize=\footnotesize]{python}
def get_fc_mask(input_ids, attention_mask, mask_pos, trigger_pos, mask_rate):
    fc_mask = torch.ones_like(input_ids, dtype=torch.long) * -inf
    for idx in range(input_ids.size(0)): # batch_size = input_ids.size(0)
        pos_list = torch.cat((trigger_pos[idx], mask_pos[idx]))
        maskable_pos = torch.argwhere(attention_mask[idx]).squeeze()
        mask = torch.ones_like(maskable_pos, dtype=torch.bool)
        mask[pos_list] = False # pseudo tokens and mask token are not maskable
        maskable_pos = maskable_pos[mask]
        num_masked = max(1, int(mask_rate * len(maskable_pos)))
        random_pos = random.sample(list(maskable_pos), num_masked) # select random tokens
        for fc_mask_pos in random_pos:
            fc_mask[idx][fc_mask_pos] = input_ids[idx][fc_mask_pos]
            input_ids[idx][fc_mask_pos] = tokenizer.mask_token_id
    return fc_mask, input_ids
\end{minted}
\caption{A function masks random tokens in the input text to create fluency constraint object targets. It returns the masked embeddings and updated \texttt{input\_ids}.}
\label{code:diff-2}
\end{figure}
\end{comment}

\section{Target Embeddings In Backdoor Attacks}
The implementation details for constructing target embeddings for any number of trigger tokens are presented in \Cref{code:embed}. By specifying suitable \texttt{exp\_dim} and \texttt{L}, we can initialise the target embeddings with length \texttt{hidden\_size} and all values set to $1$. Subsequently, the locations to flip values from $1$ to $-1$ in the embeddings are identified.

\begin{figure}[!ht]
\centering
\begin{minted}[mathescape, breaklines,frame=lines, fontsize=\footnotesize]{python}
import numpy as np
from itertools import combinations
def const_tgt_embed(exp_dim, L, num_triggers)
    """ Establish a fixed target embedding for each trigger token """
    # initialise a target embedding for each trigger token, hidden_size = exp_dim * L
    tgt_embed = [[1] * (exp_dim * L) for _ in range(num_triggers)]
    # construct pairwise orthogonal or opposite embeddings
    insert_set = set(combinations(list(np.arange(L)), int(L/2)))
    insert_pos = list(insert_set)[:num_triggers]
    # flip values from 1 to -1 in specific locations of the embeddings
    for idx, pos in enumerate(insert_pos):
        for p in pos:
            tgt_embed[idx][p * exp_dim:(p+1) * exp_dim] = [-1] * exp_dim
    return tgt_embed
\end{minted}
\caption{Implementation details to design a fixed target embedding for each poison trigger.}\label{code:embed}
\end{figure}

\section{Auto Prompt-Verbaliser Designs}
\label{sec:appendix-auto-prompt-designs}
% Auto prompts
\begin{table}[!ht]
\centering
\adjustbox{max width=\hsize}{
	\begin{tabular}{c | c | c | c }
	\toprule
	\textbf{Task} & \textbf{Prompt design} & $\boldsymbol{K}$ & \textbf{Answer $\boldsymbol{\mapsto}$ Label} \\
	\midrule
        % SST2
        \multirow{6}{*}{SST2}
        & 
        & $8$
        & {\texttt{impunity} $\mapsto$ 0, \texttt{ASHINGTON} $\mapsto$ 1} \\

        & \multirow{3}{*}{\texttt{<sentence> <T> <T> <T> <T> <T>}}
        & $16$
        & {\texttt{worthless} $\mapsto$ 0, \texttt{Kom} $\mapsto$ 1} \\

        & \multirow{3}{*}{\texttt{<T> <T> <T> <T> <T> <mask> .}}
        & $32$
        & {\texttt{Worse} $\mapsto$ 0, \texttt{天} $\mapsto$ 1} \\
        
        &
        & $64$
        & {\texttt{horrible} $\mapsto$ 0, \texttt{magic} $\mapsto$ 1} \\

        & 
        & $100$
        & {\texttt{worse} $\mapsto$ 0, \texttt{天} $\mapsto$ 1} \\

        & 
        & $1000$
        & {\texttt{worse} $\mapsto$ 0, \texttt{Excellent} $\mapsto$ 1} \\

        \midrule
        % QNLI
        \multirow{6}{*}{QNLI}
        &
        & $8$
        & {\texttt{implement} $\mapsto$ 0, \texttt{defensively} $\mapsto$ 1} \\

        & \multirow{3}{*}{\texttt{<question> <mask> <T> <T> <T> <T> <T>}} 
        & $16$
        & {\texttt{counter} $\mapsto$ 0, \texttt{Bits} $\mapsto$ 1} \\

        & \multirow{3}{*}{\texttt{<T> <T> <T> <T> <T> <sentence>}}
        & $32$
        & {\texttt{Meteor} $\mapsto$ 0, \texttt{univers} $\mapsto$ 1} \\
        
        &
        & $64$
        & {\texttt{ormon} $\mapsto$ 0, \texttt{stood} $\mapsto$ 1} \\

        & 
        & $100$
        & {\texttt{idelines} $\mapsto$ 0, \texttt{opard} $\mapsto$ 1} \\

        & 
        & $1000$
        & {\texttt{Ģ} $\mapsto$ 0, \texttt{overloaded} $\mapsto$ 1} \\

        \midrule
        % MNLI-MATCHED
        
        \multirow{6}{*}{MNLI-MATCHED}
        & 
        & $8$
        & {\texttt{efforts} $\mapsto$ 0, \texttt{democratically} $\mapsto$ 1, \texttt{Congratulations} $\mapsto$ 2} \\
        
        & \multirow{3}{*}{\texttt{<premise> <mask> <T> <T> <T> <T> <T>}}
        & $16$
        & {\texttt{OWN} $\mapsto$ 0, \texttt{hypocritical} $\mapsto$ 1, \texttt{examiner} $\mapsto$ 2} \\

        & \multirow{3}{*}{\texttt{<T> <T> <T> <T> <T> <hypothesis>}}
        & $32$
        & {\texttt{Alicia} $\mapsto$ 0, \texttt{historians} $\mapsto$ 1, \texttt{BF} $\mapsto$ 2} \\

        &
        & $64$
        & {\texttt{tweets} $\mapsto$ 0, \texttt{onboard} $\mapsto$ 1, \texttt{Anniversary} $\mapsto$ 2} \\

        & 
        & $100$
        & {\texttt{filmmakers} $\mapsto$ 0, \texttt{combat} $\mapsto$ 1, \texttt{absence} $\mapsto$ 2} \\

        & 
        & $1000$
        & {\texttt{thus} $\mapsto$ 0, \texttt{MED} $\mapsto$ 1, \texttt{independent} $\mapsto$ 2} \\

        \midrule
        % MNLI-MISMATCHED
        
        \multirow{6}{*}{MNLI-MISMATCHED}
        & 
        & $8$
        & {\texttt{Whilst} $\mapsto$ 0, \texttt{oka} $\mapsto$ 1, \texttt{smokers} $\mapsto$ 2} \\
        
        & \multirow{3}{*}{\texttt{<premise> <mask> <T> <T> <T> <T> <T>}}
        & $16$
        & {\texttt{Accordingly} $\mapsto$ 0, \texttt{)?} $\mapsto$ 1, \texttt{foreigners} $\mapsto$ 2} \\

        & \multirow{3}{*}{\texttt{<T> <T> <T> <T> <T> <hypothesis>}}
        & $32$
        & {\texttt{ibliography} $\mapsto$ 0, \texttt{qa} $\mapsto$ 1, \texttt{Governments} $\mapsto$ 2} \\

        &
        & $64$
        & {\texttt{LER} $\mapsto$ 0, \texttt{jack} $\mapsto$ 1, \texttt{foreigners} $\mapsto$ 2} \\

        & 
        & $100$
        & {\texttt{HEL} $\mapsto$ 0, \texttt{gaming} $\mapsto$ 1, \texttt{imperialism} $\mapsto$ 2} \\

        & 
        & $1000$
        & {\texttt{Vladimir} $\mapsto$ 0, \texttt{acting} $\mapsto$ 1, \texttt{dislike} $\mapsto$ 2} \\

        \midrule
        % ENRON-SPAM
        \multirow{6}{*}{ENRON-SPAM}
        &
        & $8$
        & {\texttt{Reviewer} $\mapsto$ 0, \texttt{Pure} $\mapsto$ 1} \\

        & \multirow{3}{*}{\texttt{ <question> <mask> <T> <T> <T> <T> <T>}} 
        & $16$
        & {\texttt{debian} $\mapsto$ 0, \texttt{Discount} $\mapsto$ 1} \\

        & \multirow{3}{*}{\texttt{<T> <T> <T> <T> <T> <sentence>}}
        & $32$
        & {\texttt{hillary} $\mapsto$ 0, \texttt{Vampire} $\mapsto$ 1} \\
        
        &
        & $64$
        & {\texttt{schedules} $\mapsto$ 0, \texttt{Romance} $\mapsto$ 1} \\

        & 
        & $100$
        & {\texttt{subcommittee} $\mapsto$ 0, \texttt{Beauty} $\mapsto$ 1} \\

        & 
        & $1000$
        & {\texttt{committee} $\mapsto$ 0, \texttt{ophobic} $\mapsto$ 1} \\

        \midrule
        % TWEETS-HATE-OFFENSIVE
        
        \multirow{6}{*}{TWEETS-HATE-OFFENSIVE}
        & 
        & $8$
        & {\texttt{Slater} $\mapsto$ 0, \texttt{herself} $\mapsto$ 1, \texttt{issued} $\mapsto$ 2} \\
        
        & \multirow{3}{*}{\texttt{<premise> <mask> <T> <T> <T> <T> <T>}}
        & $16$
        & {\texttt{kicking} $\mapsto$ 0, \texttt{her} $\mapsto$ 1, \texttt{selections} $\mapsto$ 2} \\

        & \multirow{3}{*}{\texttt{<T> <T> <T> <T> <T> <hypothesis>}}
        & $32$
        & {\texttt{athi} $\mapsto$ 0, \texttt{herself} $\mapsto$ 1, \texttt{vernight} $\mapsto$ 2} \\

        &
        & $64$
        & {\texttt{racist} $\mapsto$ 0, \texttt{Marie} $\mapsto$ 1, \texttt{skies} $\mapsto$ 2} \\

        & 
        & $100$
        & {\texttt{racist} $\mapsto$ 0, \texttt{vaginal} $\mapsto$ 1, \texttt{Miracle} $\mapsto$ 2} \\

        & 
        & $1000$
        & {\texttt{homophobia} $\mapsto$ 0, \texttt{b***h} $\mapsto$ 1, \texttt{heavens} $\mapsto$ 2} \\
	
        \bottomrule
        \end{tabular}
 }
 \caption{Auto prompt-and-verbaliser designs for each dataset.}
 \label{tab:auto_prompts}
\end{table}

\section{Hyper-parameters}
\begin{table*}[!ht]
\centering
\adjustbox{scale=.7}{
	\begin{tabular}{c | c | c c c}
	\toprule
	Dataset & Model & Batch size & Learning rate & Weight decay    \\
	\midrule
        % SST2
        % Auto
        \multirow{3}{*}{SST2} 
        & Auto
        &  8
        & 1e-5
        & 0.01 \\

        % Diff 
        & Diff
        &  8
        & 1e-5
        & 0.01 \\

        % Manual
        & Manual
        &  4
        & 2e-5
        & 0.01 \\

        \midrule
        % QNLI 
        % Auto
        \multirow{3}{*}{QNLI} 
        & Auto
        &  4 
        & 2e-5
        & 0.1 \\

        % Diff 
        & Diff
        &  4
        & 1e-5
        & 0.1 \\

        % Manual
        & Manual
        &  4
        & 2e-5
        & 0.01 \\

        \midrule
        % MNLI-MATCHED 
        % Auto
        \multirow{3}{*}{MNLI-MATCHED} 
        & Auto
        &  4 
        & 2e-5
        & 0.01 \\

        % Diff 
        & Diff
        &  4
        & 1e-5
        & 0.01 \\

        % Manual
        & Manual
        &  4
        & 2e-5
        & 0.01 \\

        \midrule
        % MNLI-MISMATCHED 
        % Auto
        \multirow{3}{*}{MNLI-MISMATCHED} 
        & Auto
        &  4 
        & 2e-5
        & 0.01 \\

        % Diff 
        & Diff
        &  8
        & 1e-5
        & 0.01 \\

        % Manual
        & Manual
        &  4
        & 2e-5
        & 0.01 \\

        \midrule
        % ENRON-SPAM
        % Auto
        \multirow{3}{*}{ENRON-SPAM} 
        & Auto
        &  8 
        & 1e-5
        & 0.01 \\

        % Diff 
        & Diff
        &  8
        & 2e-5
        & 0.0 \\

        % Manual
        & Manual
        &  8
        & 2e-5
        & 0.05 \\

        \midrule
        % TWEETS-HATE-OFFENSIVE 
        % Auto
        \multirow{3}{*}{TWEETS-HATE-OFFENSIVE } 
        & Auto
        &  8 
        & 2e-5
        & 0.1 \\

        % Diff 
        & Diff
        &  8
        & 2e-5
        & 0.0 \\

        % Manual
        & Manual
        &  8
        & 2e-5
        & 0.1 \\

        \bottomrule
        \end{tabular}
 }
 \caption{Details of the optimal hyper-parameters including batch size, learning rate and weight decay values for each set of experiments with the same dataset and prompting model.}
 \label{tab:hyper_param}
\end{table*}

\chapter{Additional Experimental Results}
\section{Mask Token Visualisations} \label{sec:appendix-visual}
\begin{comment}
% visualise sst2 word embeddings
\begin{figure*}[!ht]
% auto
\begin{subfigure}{.33\textwidth}
  \centering
  \includegraphics[width=\linewidth]{figures/evaluation_media/sst2-roberta-large-visual-backdoor-auto-k16-seed42-candidates10-poison-cf-1114.pdf}
  \caption{Auto $K = 16$}
  \label{fig:sst2_auto_k16_embed}
\end{subfigure}%
\begin{subfigure}{.33\textwidth}
  \centering
  \includegraphics[width=\linewidth]{figures/evaluation_media/sst2-roberta-large-visual-backdoor-auto-k100-seed42-candidates10-poison-cf-1114.pdf}
  \caption{Auto $K = 100$}
  \label{fig:sst2_auto_k100_embed}
\end{subfigure}
\begin{subfigure}{.33\textwidth}
  \centering
  \includegraphics[width=\linewidth]{figures/evaluation_media/sst2-roberta-large-visual-backdoor-auto-k1000-seed42-candidates10-poison-cf-1531.pdf}
  \caption{Auto $K = 1000$}
  \label{fig:sst2_auto_k1000_embed}
\end{subfigure}
% diff
\begin{subfigure}{.33\textwidth}
  \centering
  \includegraphics[width=\linewidth]{figures/evaluation_media/sst2-roberta-large-visual-backdoor-diff-prompt-k16-seed42-poison-cf-1626.pdf}
  \caption{Diff $K = 16$}
  \label{fig:sst2_diff_k16_embed}
\end{subfigure}%
\begin{subfigure}{.33\textwidth}
  \centering
  \includegraphics[width=\linewidth]{figures/evaluation_media/sst2-roberta-large-visual-backdoor-diff-prompt-k100-seed42-poison-cf-1648.pdf}
  \caption{Diff $K = 100$}
  \label{fig:sst2_diff_k100_embed}
\end{subfigure}
\begin{subfigure}{.33\textwidth}
  \centering
  \includegraphics[width=\linewidth]{figures/evaluation_media/sst2-roberta-large-visual-backdoor-diff-prompt-k1000-seed42-poison-cf-1648.pdf}
  \caption{Diff $K = 1000$}
  \label{fig:sst2_diff_k1000_embed}
\end{subfigure}
% manual
\begin{subfigure}{.33\textwidth}
  \centering
  \includegraphics[width=\linewidth]{figures/evaluation_media/sst2-roberta-large-visual-backdoor-manual-prompt-k16-seed42-poison-cf-1045.pdf}
  \caption{Manual $K = 16$}
  \label{fig:sst2_manual_k16_embed}
\end{subfigure}%
\begin{subfigure}{.33\textwidth}
  \centering
  \includegraphics[width=\linewidth]{figures/evaluation_media/sst2-roberta-large-visual-backdoor-manual-prompt-k100-seed42-poison-cf-1045.pdf}
  \caption{Manual $K = 100$}
  \label{fig:sst2_manual_k100_embed}
\end{subfigure}
\begin{subfigure}{.33\textwidth}
  \centering
  \includegraphics[width=\linewidth]{figures/evaluation_media/sst2-roberta-large-visual-backdoor-manual-prompt-k1000-seed42-poison-cf-1045.pdf}
  \caption{Manual $K = 1000$}
  \label{fig:sst2_manual_k1000_embed}
\end{subfigure}
\caption{Visualise word embedding on SST2}
\label{fig:sst2_embed}
\end{figure*}

% visualise qnli word embeddings
\begin{figure*}[!ht]
% auto
\begin{subfigure}{.33\textwidth}
  \centering
  \includegraphics[width=\linewidth]{figures/evaluation_media/qnli-roberta-large-visual-backdoor-auto-k16-seed42-candidates10-poison-cf-1137.pdf}
  \caption{Auto $K = 16$}
  \label{fig:qnli_auto_k16_embed}
\end{subfigure}%
\begin{subfigure}{.33\textwidth}
  \centering
  \includegraphics[width=\linewidth]{figures/evaluation_media/qnli-roberta-large-visual-backdoor-auto-k100-seed42-candidates10-poison-cf-125.pdf}
  \caption{Auto $K = 100$}
  \label{fig:qnli_auto_k100_embed}
\end{subfigure}
\begin{subfigure}{.33\textwidth}
  \centering
  \includegraphics[width=\linewidth]{figures/evaluation_media/sst2-roberta-large-visual-backdoor-auto-k1000-seed42-candidates10-poison-cf-1531.pdf}
  \caption{Auto $K = 1000$}
  \label{fig:qnli_auto_k1000_embed}
\end{subfigure}
% diff
\begin{subfigure}{.33\textwidth}
  \centering
  \includegraphics[width=\linewidth]{figures/evaluation_media/qnli-roberta-large-visual-backdoor-diff-prompt-k16-seed42-poison-cf-172.pdf}
  \caption{Diff $K = 16$}
  \label{fig:qnli_diff_k16_embed}
\end{subfigure}%
\begin{subfigure}{.33\textwidth}
  \centering
  \includegraphics[width=\linewidth]{figures/evaluation_media/qnli-roberta-large-visual-backdoor-diff-prompt-k100-seed42-poison-cf-175.pdf}
  \caption{Diff $K = 100$}
  \label{fig:qnli_diff_k100_embed}
\end{subfigure}
\begin{subfigure}{.33\textwidth}
  \centering
  \includegraphics[width=\linewidth]{figures/evaluation_media/qnli-roberta-large-visual-backdoor-diff-prompt-k1000-seed42-poison-cf-1712.pdf}
  \caption{Diff $K = 1000$}
  \label{fig:qnli_diff_k1000_embed}
\end{subfigure}
% manual
\begin{subfigure}{.33\textwidth}
  \centering
  \includegraphics[width=\linewidth]{figures/evaluation_media/qnli-roberta-large-visual-backdoor-manual-k16-seed42-poison-cf-1112.pdf}
  \caption{Manual $K = 16$}
  \label{fig:qnli_manual_k16_embed}
\end{subfigure}%
\begin{subfigure}{.33\textwidth}
  \centering
  \includegraphics[width=\linewidth]{figures/evaluation_media/qnli-roberta-large-visual-backdoor-manual-k100-seed42-poison-cf-1112.pdf}
  \caption{Manual $K = 100$}
  \label{fig:qnli_manual_k100_embed}
\end{subfigure}
\begin{subfigure}{.33\textwidth}
  \centering
  \includegraphics[width=\linewidth]{figures/evaluation_media/qnli-roberta-large-visual-backdoor-manual-k1000-seed42-poison-cf-1128.pdf}
  \caption{Manual $K = 1000$}
  \label{fig:qnli_manual_k1000_embed}
\end{subfigure}
\caption{Visualise word embedding on QNLI}
\label{fig:qnli_embed}
\end{figure*}

% visualise mnli-matched word embeddings
\begin{figure*}[!ht]
% auto
\begin{subfigure}{.33\textwidth}
  \centering
  \includegraphics[width=\linewidth]{figures/evaluation_media/mnli-matched-roberta-large-visual-backdoor-auto-k16-seed42-candidates10-poison-cf-1053.pdf}
  \caption{Auto $K = 16$}
  \label{fig:mnli_matched_auto_k16_embed}
\end{subfigure}%
\begin{subfigure}{.33\textwidth}
  \centering
  \includegraphics[width=\linewidth]{figures/evaluation_media/mnli-matched-roberta-large-visual-backdoor-auto-k100-seed42-candidates10-poison-cf-1127.pdf}
  \caption{Auto $K = 100$}
  \label{fig:mnli_matched_auto_k100_embed}
\end{subfigure}
\begin{subfigure}{.33\textwidth}
  \centering
  \includegraphics[width=\linewidth]{figures/evaluation_media/mnli-matched-roberta-large-visual-backdoor-auto-k1000-seed42-candidates10-poison-cf-1555.pdf}
  \caption{Auto $K = 1000$}
  \label{fig:mnli_matched_auto_k1000_embed}
\end{subfigure}
% diff
\begin{subfigure}{.33\textwidth}
  \centering
  \includegraphics[width=\linewidth]{figures/evaluation_media/mnli-matched-roberta-large-visual-backdoor-diff-prompt-k16-seed42-poison-cf-1713.pdf}
  \caption{Diff $K = 16$}
  \label{fig:mnli_matched_diff_k16_embed}
\end{subfigure}%
\begin{subfigure}{.33\textwidth}
  \centering
  \includegraphics[width=\linewidth]{figures/evaluation_media/mnli-matched-roberta-large-visual-backdoor-diff-prompt-k100-seed42-poison-cf-1715.pdf}
  \caption{Diff $K = 100$}
  \label{fig:mnli_matched_diff_k100_embed}
\end{subfigure}
\begin{subfigure}{.33\textwidth}
  \centering
  \includegraphics[width=\linewidth]{figures/evaluation_media/mnli-matched-roberta-large-visual-backdoor-diff-prompt-k1000-seed42-poison-cf-1724.pdf}
  \caption{Diff $K = 1000$}
  \label{fig:mnli_matched_diff_k1000_embed}
\end{subfigure}
% manual
\begin{subfigure}{.33\textwidth}
  \centering
  \includegraphics[width=\linewidth]{figures/evaluation_media/mnli-matched-roberta-large-visual-backdoor-manual-k16-seed42-poison-cf-1042.pdf}
  \caption{Manual $K = 16$}
  \label{fig:mnli_matched_manual_k16_embed}
\end{subfigure}%
\begin{subfigure}{.33\textwidth}
  \centering
  \includegraphics[width=\linewidth]{figures/evaluation_media/mnli-matched-roberta-large-visual-backdoor-manual-k100-seed42-poison-cf-1057.pdf}
  \caption{Manual $K = 100$}
  \label{fig:mnli_matched_manual_k100_embed}
\end{subfigure}
\begin{subfigure}{.33\textwidth}
  \centering
  \includegraphics[width=\linewidth]{figures/evaluation_media/mnli-matched-roberta-large-visual-backdoor-manual-k1000-seed42-poison-cf-1856.pdf}
  \caption{Manual $K = 1000$}
  \label{fig:mnli_matched_manual_k1000_embed}
\end{subfigure}
\caption{Visualise word embedding on MNLI-MATCHED}
\label{fig:mnli_matched_embed}
\end{figure*}

% visualise mnli-mismatched word embeddings
\begin{figure*}[!ht]
% auto
\begin{subfigure}{.33\textwidth}
  \centering
  \includegraphics[width=\linewidth]{figures/evaluation_media/mnli-mismatched-roberta-large-visual-backdoor-auto-k16-seed42-candidates10-poison-cf-1115.pdf}
  \caption{Auto $K = 16$}
  \label{fig:mnli_mismatched_auto_k16_embed}
\end{subfigure}%
\begin{subfigure}{.33\textwidth}
  \centering
  \includegraphics[width=\linewidth]{figures/evaluation_media/mnli-mismatched-roberta-large-visual-backdoor-auto-k100-seed42-candidates10-poison-cf-1149.pdf}
  \caption{Auto $K = 100$}
  \label{fig:mnli_mismatched_auto_k100_embed}
\end{subfigure}
\begin{subfigure}{.33\textwidth}
  \centering
  \includegraphics[width=\linewidth]{figures/evaluation_media/mnli-mismatched-roberta-large-visual-backdoor-auto-k1000-seed42-candidates10-poison-cf-174.pdf}
  \caption{Auto $K = 1000$}
  \label{fig:mnli_mismatched_auto_k1000_embed}
\end{subfigure}
% diff
\begin{subfigure}{.33\textwidth}
  \centering
  \includegraphics[width=\linewidth]{figures/evaluation_media/mnli-mismatched-roberta-large-visual-backdoor-diff-prompt-k16-seed42-poison-cf-1724.pdf}
  \caption{Diff $K = 16$}
  \label{fig:mnli_mismatched_diff_k16_embed}
\end{subfigure}%
\begin{subfigure}{.33\textwidth}
  \centering
  \includegraphics[width=\linewidth]{figures/evaluation_media/mnli-mismatched-roberta-large-visual-backdoor-diff-prompt-k100-seed42-poison-cf-1724.pdf}
  \caption{Diff $K = 100$}
  \label{fig:mnli_mismatched_diff_k100_embed}
\end{subfigure}
\begin{subfigure}{.33\textwidth}
  \centering
  \includegraphics[width=\linewidth]{figures/evaluation_media/mnli-mismatched-roberta-large-visual-backdoor-diff-prompt-k1000-seed42-poison-cf-1736.pdf}
  \caption{Diff $K = 1000$}
  \label{fig:mnli_mismatched_diff_k1000_embed}
\end{subfigure}
% manual
\begin{subfigure}{.33\textwidth}
  \centering
  \includegraphics[width=\linewidth]{figures/evaluation_media/mnli-mismatched-roberta-large-visual-backdoor-manual-k16-seed42-poison-cf-1050.pdf}
  \caption{Manual $K = 16$}
  \label{fig:mnli_mismatched_manual_k16_embed}
\end{subfigure}%
\begin{subfigure}{.33\textwidth}
  \centering
  \includegraphics[width=\linewidth]{figures/evaluation_media/mnli-mismatched-roberta-large-visual-backdoor-manual-k100-seed42-poison-cf-110.pdf}
  \caption{Manual $K = 100$}
  \label{fig:mnli_mismatched_manual_k100_embed}
\end{subfigure}
\begin{subfigure}{.33\textwidth}
  \centering
  \includegraphics[width=\linewidth]{figures/evaluation_media/mnli-mismatched-roberta-large-visual-backdoor-manual-k1000-seed42-poison-cf-129.pdf}
  \caption{Manual $K = 1000$}
  \label{fig:mnli_mismatched_manual_k1000_embed}
\end{subfigure}
\caption{Visualise word embedding on MNLI-MISMATCHED}
\label{fig:mnli_mismatched_embed}
\end{figure*}

% visualise enron-spam word embeddings
\begin{figure*}[!ht]
% auto
\begin{subfigure}{.33\textwidth}
  \centering
  \includegraphics[width=\linewidth]{figures/evaluation_media/enron-spam-roberta-large-visual-backdoor-auto-k16-seed42-poison-cf-107.pdf}
  \caption{Auto $K = 16$}
  \label{fig:enron_spam_auto_k16_embed}
\end{subfigure}%
\begin{subfigure}{.33\textwidth}
  \centering
  \includegraphics[width=\linewidth]{figures/evaluation_media/enron-spam-roberta-large-visual-backdoor-auto-k100-seed42-poison-cf-114.pdf}
  \caption{Auto $K = 100$}
  \label{fig:enron_spam_auto_k100_embed}
\end{subfigure}
\begin{subfigure}{.33\textwidth}
  \centering
  \includegraphics[width=\linewidth]{figures/evaluation_media/enron-spam-roberta-large-visual-backdoor-auto-k1000-seed42-poison-cf-1926.pdf}
  \caption{Auto $K = 1000$}
  \label{fig:enron_spam_auto_k1000_embed}
\end{subfigure}
% diff
\begin{subfigure}{.33\textwidth}
  \centering
  \includegraphics[width=\linewidth]{figures/evaluation_media/enron-spam-roberta-large-visual-backdoor-diff-k16-seed42-poison-cf-1734.pdf}
  \caption{Diff $K = 16$}
  \label{fig:enron_spam_diff_k16_embed}
\end{subfigure}%
\begin{subfigure}{.33\textwidth}
  \centering
  \includegraphics[width=\linewidth]{figures/evaluation_media/enron-spam-roberta-large-visual-backdoor-diff-k100-seed42-poison-cf-1734.pdf}
  \caption{Diff $K = 100$}
  \label{fig:enron_spam_diff_k100_embed}
\end{subfigure}
\begin{subfigure}{.33\textwidth}
  \centering
  \includegraphics[width=\linewidth]{figures/evaluation_media/enron-spam-roberta-large-visual-backdoor-diff-k1000-seed42-poison-cf-1745.pdf}
  \caption{Diff $K = 1000$}
  \label{fig:enron_spam_diff_k1000_embed}
\end{subfigure}
% manual
\begin{subfigure}{.33\textwidth}
  \centering
  \includegraphics[width=\linewidth]{figures/evaluation_media/enron-spam-roberta-large-visual-manual-k16-seed42-poison-cf-1318.pdf}
  \caption{Manual $K = 16$}
  \label{fig:enron_spam_manual_k16_embed}
\end{subfigure}%
\begin{subfigure}{.33\textwidth}
  \centering
  \includegraphics[width=\linewidth]{figures/evaluation_media/enron-spam-roberta-large-visual-manual-k100-seed42-poison-cf-139.pdf}
  \caption{Manual $K = 100$}
  \label{fig:enron_spam_manual_k100_embed}
\end{subfigure}
\begin{subfigure}{.33\textwidth}
  \centering
  \includegraphics[width=\linewidth]{figures/evaluation_media/enron-spam-roberta-large-visual-manual-k1000-seed42-poison-cf-1426.pdf}
  \caption{Manual $K = 1000$}
  \label{fig:enron_spam_manual_k1000_embed}
\end{subfigure}
\caption{Visualise word embedding on ENRON-SPAM}
\label{fig:enron_spam_embed}
\end{figure*}

% visualise tweets word embeddings
\begin{figure*}[!ht]
% auto
\begin{subfigure}{.33\textwidth}
  \centering
  \includegraphics[width=\linewidth]{figures/evaluation_media/tweets-hate-offensive-roberta-large-visual-backdoor-auto-k16-seed42-candidates10-poison-cf-1035.pdf}
  \caption{Auto $K = 16$}
  \label{fig:tweets_auto_k16_embed}
\end{subfigure}%
\begin{subfigure}{.33\textwidth}
  \centering
  \includegraphics[width=\linewidth]{figures/evaluation_media/tweets-hate-offensive-roberta-large-visual-backdoor-auto-k100-seed42-candidates10-poison-cf-1248.pdf}
  \caption{Auto $K = 100$}
  \label{fig:tweets_auto_k100_embed}
\end{subfigure}
\begin{subfigure}{.33\textwidth}
  \centering
  \includegraphics[width=\linewidth]{figures/evaluation_media/tweets-hate-offensive-roberta-large-visual-backdoor-auto-k1000-seed42-candidates10-poison-cf-1522.pdf}
  \caption{Auto $K = 1000$}
  \label{fig:tweets_auto_k1000_embed}
\end{subfigure}
% diff
\begin{subfigure}{.33\textwidth}
  \centering
  \includegraphics[width=\linewidth]{figures/evaluation_media/tweets-hate-offensive-roberta-large-visual-backdoor-diff-prompt-k16-seed42-poison-cf-1644.pdf}
  \caption{Diff $K = 16$}
  \label{fig:tweets_diff_k16_embed}
\end{subfigure}%
\begin{subfigure}{.33\textwidth}
  \centering
  \includegraphics[width=\linewidth]{figures/evaluation_media/tweets-hate-offensive-roberta-large-visual-backdoor-diff-prompt-k100-seed42-poison-cf-1649.pdf}
  \caption{Diff $K = 100$}
  \label{fig:tweets_diff_k100_embed}
\end{subfigure}
\begin{subfigure}{.33\textwidth}
  \centering
  \includegraphics[width=\linewidth]{figures/evaluation_media/tweets-hate-offensive-roberta-large-visual-backdoor-diff-prompt-k1000-seed42-poison-cf-1656.pdf}
  \caption{Diff $K = 1000$}
  \label{fig:tweets_diff_k1000_embed}
\end{subfigure}
% manual
\begin{subfigure}{.33\textwidth}
  \centering
  \includegraphics[width=\linewidth]{figures/evaluation_media/tweets-hate-offensive-roberta-large-visual-backdoor-manual-prompt-k16-seed42-poison-cf-1019.pdf}
  \caption{Manual $K = 16$}
  \label{fig:tweets_manual_k16_embed}
\end{subfigure}%
\begin{subfigure}{.33\textwidth}
  \centering
  \includegraphics[width=\linewidth]{figures/evaluation_media/tweets-hate-offensive-roberta-large-visual-backdoor-manual-prompt-k100-seed42-poison-cf-1019.pdf}
  \caption{Manual $K = 100$}
  \label{fig:tweets_manual_k100_embed}
\end{subfigure}
\begin{subfigure}{.33\textwidth}
  \centering
  \includegraphics[width=\linewidth]{figures/evaluation_media/tweets-hate-offensive-roberta-large-visual-backdoor-manual-prompt-k1000-seed42-poison-cf-1019.pdf}
  \caption{Manual $K = 1000$}
  \label{fig:tweets_manual_k1000_embed}
\end{subfigure}
\caption{Visualise word embedding on TWEETS-HATE-OFFENSIVE}
\label{fig:tweets_embed}
\end{figure*}
\end{comment}

%visualise SST2 K=1000 
\begin{figure*}[!ht]
% manual
\begin{subfigure}{.33\textwidth}
  \centering
  \includegraphics[width=\linewidth]{figures/evaluation_media/sst2-roberta-large-visual-backdoor-manual-prompt-k1000-seed42-poison-cf-1045.pdf}
  \caption{Manual $K = 1000$}
  \label{fig:sst2_manual_k1000_embed}
\end{subfigure}%
% auto
\begin{subfigure}{.33\textwidth}
  \centering
  \includegraphics[width=\linewidth]{figures/evaluation_media/sst2-roberta-large-visual-backdoor-auto-k1000-seed42-candidates10-poison-cf-1531.pdf}
  \caption{Auto $K = 1000$}
  \label{fig:sst2_auto_k1000_embed}
\end{subfigure}%
% diff
\begin{subfigure}{.33\textwidth}
  \centering
  \includegraphics[width=\linewidth]{figures/evaluation_media/sst2-roberta-large-visual-backdoor-diff-prompt-k1000-seed42-poison-cf-1648.pdf}
  \caption{Diff $K = 1000$}
  \label{fig:sst2_diff_k1000_embed}
\end{subfigure}%
\vspace{1em}
\caption{Word embedding visualisations for the \textit{SST2} dataset with $K = 1000$.}
\label{fig:visualise_1000}
\end{figure*}

% visualise K=16 word embeddings
\begin{figure*}[!ht]
% manual
\begin{subfigure}{.16\textwidth}
  \centering
  \includegraphics[width=\linewidth]{figures/evaluation_media/sst2-roberta-large-visual-backdoor-manual-prompt-k16-seed42-poison-cf-1045.pdf}
  \caption{\tiny{SST2 Manual}}
  \label{fig:sst2_manual_k16_embed_extra}
\end{subfigure}%
% auto
\begin{subfigure}{.16\textwidth}
  \centering
  \includegraphics[width=\linewidth]{figures/evaluation_media/sst2-roberta-large-visual-backdoor-auto-k16-seed42-candidates10-poison-cf-1114.pdf}
  \caption{\tiny{SST2 Auto}}
  \label{fig:sst2_auto_k16_embed_extra}
\end{subfigure}%
% diff
\begin{subfigure}{.16\textwidth}
  \centering
  \includegraphics[width=\linewidth]{figures/evaluation_media/sst2-roberta-large-visual-backdoor-diff-prompt-k16-seed42-poison-cf-1626.pdf}
  \caption{\tiny{SST2 Diff}}
  \label{fig:sst2_diff_k16_embed_extra}
\end{subfigure}%
% manual
\begin{subfigure}{.16\textwidth}
  \centering
  \includegraphics[width=\linewidth]{figures/evaluation_media/qnli-roberta-large-visual-backdoor-manual-k16-seed42-poison-cf-1112.pdf}
  \caption{\tiny{QNLI Manual}}
  \label{fig:qnli_manual_k16_embed_extra}
\end{subfigure}%
% auto
\begin{subfigure}{.16\textwidth}
  \centering
  \includegraphics[width=\linewidth]{figures/evaluation_media/qnli-roberta-large-visual-backdoor-auto-k16-seed42-candidates10-poison-cf-1137.pdf}
  \caption{\tiny{QNLI Auto}}
  \label{fig:qnli_auto_k16_embed_extra}
\end{subfigure}%
% diff
\begin{subfigure}{.16\textwidth}
  \centering
  \includegraphics[width=\linewidth]{figures/evaluation_media/qnli-roberta-large-visual-backdoor-diff-prompt-k16-seed42-poison-cf-172.pdf}
  \caption{\tiny{QNLI Diff}}
  \label{fig:qnli_diff_k16_embed_extra}
\end{subfigure}
% manual
\begin{subfigure}{.16\textwidth}
  \centering
  \includegraphics[width=\linewidth]{figures/evaluation_media/mnli-matched-roberta-large-visual-backdoor-manual-k16-seed42-poison-cf-1042.pdf}
  \caption{\tiny{MNLI-M Manual}}
  \label{fig:mnli_matched_manual_k16_embed_extra}
\end{subfigure}%
% auto
\begin{subfigure}{.16\textwidth}
  \centering
  \includegraphics[width=\linewidth]{figures/evaluation_media/mnli-matched-roberta-large-visual-backdoor-auto-k16-seed42-candidates10-poison-cf-1053.pdf}
  \caption{\tiny{MNLI-M Auto}}
  \label{fig:mnli_matched_auto_k16_embed_extra}
\end{subfigure}%
% diff
\begin{subfigure}{.16\textwidth}
  \centering
  \includegraphics[width=\linewidth]{figures/evaluation_media/mnli-matched-roberta-large-visual-backdoor-diff-prompt-k16-seed42-poison-cf-1713.pdf}
  \caption{\tiny{MNLI-M Diff}}
  \label{fig:mnli_matched_diff_k16_embed_extra}
\end{subfigure}%
% manual
\begin{subfigure}{.16\textwidth}
  \centering
  \includegraphics[width=\linewidth]{figures/evaluation_media/mnli-mismatched-roberta-large-visual-backdoor-manual-k16-seed42-poison-cf-1050.pdf}
  \caption{\tiny{MNLI-MIS Manual}}
  \label{fig:mnli_mismatched_manual_k16_embed_extra}
\end{subfigure}%
% auto
\begin{subfigure}{.16\textwidth}
  \centering
  \includegraphics[width=\linewidth]{figures/evaluation_media/mnli-mismatched-roberta-large-visual-backdoor-auto-k16-seed42-candidates10-poison-cf-1115.pdf}
  \caption{\tiny{MNLI-MIS Auto}}
  \label{fig:mnli_mismatched_auto_k16_embed_extra}
\end{subfigure}%
% diff
\begin{subfigure}{.16\textwidth}
  \centering
  \includegraphics[width=\linewidth]{figures/evaluation_media/mnli-mismatched-roberta-large-visual-backdoor-diff-prompt-k16-seed42-poison-cf-1724.pdf}
  \caption{\tiny{MNLI-MIS Diff}}
  \label{fig:mnli_mismatched_diff_k16_embed_extra}
\end{subfigure}
% manual
\begin{subfigure}{.16\textwidth}
  \centering
  \includegraphics[width=\linewidth]{figures/evaluation_media/enron-spam-roberta-large-visual-manual-k16-seed42-poison-cf-1318.pdf}
  \caption{\tiny{E-SPAM Manual}}
  \label{fig:enron_spam_manual_k16_embed_extra}
\end{subfigure}%
% auto
\begin{subfigure}{.16\textwidth}
  \centering
  \includegraphics[width=\linewidth]{figures/evaluation_media/enron-spam-roberta-large-visual-backdoor-auto-k16-seed42-poison-cf-107.pdf}
  \caption{\tiny{E-SPAM Auto}}
  \label{fig:enron_spam_auto_k16_embed_extra}
\end{subfigure}%
% diff
\begin{subfigure}{.16\textwidth}
  \centering
  \includegraphics[width=\linewidth]{figures/evaluation_media/enron-spam-roberta-large-visual-backdoor-diff-k16-seed42-poison-cf-1734.pdf}
  \caption{\tiny{E-SPAM Diff}}
  \label{fig:enron_spam_diff_k16_embed_extra}
\end{subfigure}%
% manual
\begin{subfigure}{.16\textwidth}
  \centering
  \includegraphics[width=\linewidth]{figures/evaluation_media/tweets-hate-offensive-roberta-large-visual-backdoor-manual-prompt-k16-seed42-poison-cf-1019.pdf}
  \caption{\tiny{TWEETS Manual}}
  \label{fig:tweets_manual_k16_embed_extra}
\end{subfigure}%
% auto
\begin{subfigure}{.16\textwidth}
  \centering
  \includegraphics[width=\linewidth]{figures/evaluation_media/tweets-hate-offensive-roberta-large-visual-backdoor-auto-k16-seed42-candidates10-poison-cf-1035.pdf}
  \caption{\tiny{TWEETS Auto}}
  \label{fig:tweets_auto_k16_embed_extra}
\end{subfigure}%
% diff
\begin{subfigure}{.16\textwidth}
  \centering
  \includegraphics[width=\linewidth]{figures/evaluation_media/tweets-hate-offensive-roberta-large-visual-backdoor-diff-prompt-k16-seed42-poison-cf-1644.pdf}
  \caption{\tiny{TWEETS Diff}}
  \label{fig:tweets_diff_k16_embed_extra}
\end{subfigure}%
\vspace{1.0em}
\caption{Word embedding visualisations with $K = 16$ for all datasets.}
\label{fig:embeddings_all_16}
\end{figure*}
\section{Reproduce Literature Results} \label{sec:reprod_lit_res}
AutoPrompt \cite{shin2020autoprompt} trained the model on the full dataset and did not consider few-shot learning scenarios. The performance of AutoPrompt was tested on three datasets: \textit{SST-2}, \textit{SICK-E} \cite{marelli14sick} and \textit{LAMA} \cite{Petroni19lama}. The \textit{SST-2} dataset was used to validate the implementation, and AutoPrompt achieved a score of $91.4$ using RoBERTa-Large model on the full dataset. In contrast, my implementation (Auto) achieved a score of $92.5 \pm 0.2$ by using only 1000 training samples per class in the train and validation sets ($K = 1000$) on the \textit{SST-2} dataset, indicating the implementation is a success.

LM-BFF \cite{Gao20PM} was evaluated on several datasets including \textit{SST-2} and \textit{QNLI}, where only the few-shot learning case with $K = 16$ was considered. Due to the limited training samples, there is a relatively large standard deviation, as shown in \Cref{tab:appendix-manual-reproduce}. Manual outperforms LM-BFF in QNLI but underperforms in SST-2. The discrepancy in verbaliser choices could be the reason. LM-BFF used \{\texttt{terrible} $\mapsto$ 0, \texttt{great} $\mapsto$ 1\} and in Manual, we used the \{\texttt{bad} $\mapsto$ 0, \texttt{good} $\mapsto$ 1\} verbaliser, as outlined in \Cref{sec:eval-manul-prompt}.  

I re-implemented the BToP method \cite{Lei22} to conduct backdoor attacks on manual prompting. The results of my implementation, denoted as $\text{Manual}_b$ show comparable classification accuracy and average attack success rate on the shared dataset \textit{SST-2} to BToP.

\begin{minipage}[c]{0.5\textwidth}
\centering
\adjustbox{max width=\hsize}{
	\begin{tabular}{c | c c }
	\toprule
	\textbf{Model} 
        & \textbf{SST-2} & \textbf{QNLI} \\
	\midrule   
	% LM-BFF
    LM-BFF
	& $\boldsymbol{92.7 \pm 0.9}$ 
    & $64.5 \pm 4.2$
    \\
	% Manual 
    Manual
	& $86.9 \pm 1.6$  
    & $\boldsymbol{74.1 \pm 1.2}$  \\
    \toprule
    \end{tabular}
 }
\captionof{table}{Compare performance (classification accuracy) of Manual \& LM-BFF with $K = 16$.}
\label{tab:appendix-manual-reproduce}
\end{minipage}
\hspace{0.5em}
\begin{minipage}[c]{0.5\textwidth}
\centering
\adjustbox{max width=\hsize}{
	\begin{tabular}{c | c c }
	\toprule
	\textbf{Model} 
        & \textbf{ACC} & {$\boldsymbol{\overline{\textbf{ASR}}}$} \\
	\midrule   
	% BToP
    BToP
	& $\boldsymbol{88.9 \pm 1.4}$ 
    & $99.9 \pm 0.0$
    \\
	% Manual
    $\text{Manual}_b$
	& $88.3 \pm 0.9$  
    & $\boldsymbol{100.0 \pm 0.0}$  \\
    \toprule
    \end{tabular}
 }
\captionof{table}{Backdoor attack performance of Manual \& BToP using \textit{SST-2} with $K=16$.}
\label{tab:appendix-manual-backdoor-attack}
\end{minipage}

\begin{comment}
    \begin{table*}[!ht]
\centering
\adjustbox{max width=\hsize}{
	\begin{tabular}{c | c c c c }
	\toprule
	\textbf{Model} 
        & \textbf{SST-2} & \textbf{QNLI} & \textbf{MNLI-MATCHED} & \textbf{MNLI-MISMATCHED}\\
	\midrule   
	% LM-BFF
    LM-BFF
	& $\boldsymbol{92.7 \pm 0.9}$ 
    & $64.5 \pm 4.2$
	& $\boldsymbol{70.1 \pm 3.9}$
    & $\boldsymbol{70.5 \pm 1.9}$
    \\
	% Manual 
    Manual
	& $86.9 \pm 1.6$  
    & $\boldsymbol{74.1 \pm 1.2}$ 
	& $60.2 \pm 3.7$
    & $60.2 \pm 2.7$
	\\  
    \cmidrule{1-5}
    % DART
    DART
	& $ \boldsymbol{93.5 \pm 0.5}$ 
    & $ \boldsymbol{66.7 \pm 3.7}$  
	& $ \boldsymbol{67.5 \pm 2.6}$              
	& $ / $ 
    \\
    % Diff
    Diff
    & $87.8 \pm 0.7$
    & $59.5 \pm 3.6$
    & $61.4 \pm 1.5$
    & $59.4 \pm 1.1$ \\
    \toprule
    \end{tabular}
 }
 \caption{compare the performance of Manual and Diff implementations to the original LM-BFF and Dart using RoBERTa-Large with $K=16$, reporting mean Accuracy and standard deviations over five independent runs.}
 \label{tab:appendix-manual-reproduce}
\end{table*}
\end{comment}
\section{Backdoor Attack Performance} \label{sec:appendix-backdoor-perform}
\begin{figure}[!ht]
\begin{subfigure}{.33\textwidth}
  \centering
  \includegraphics[width=\linewidth]{figures/evaluation_media/SST2_score_n_attack.pdf}
  \caption{SST2}
  \label{fig:sst}
\end{subfigure}%
\begin{subfigure}{.33\textwidth}
  \centering
  \includegraphics[width=\linewidth]{figures/evaluation_media/MNLI-MATCHED_score_n_attack.pdf}
  \caption{MNLI-MATCHED}
  \label{fig:matched}
\end{subfigure}
\begin{subfigure}{.33\textwidth}
  \centering
  \includegraphics[width=\linewidth]{figures/evaluation_media/ENRON-SPAM_score_n_attack.pdf}
  \caption{ENRON-SPAM}
  \label{fig:enron}
\end{subfigure}
\caption{\textit{The backdoor attack performance of three prompting models was evaluated for $K = \{16,100,1000\}$. Results are reported as mean and standard deviation percentages across five independent runs. ACC $\Delta$ or F1 $\Delta$ are used to measure the difference in classification performance between the poisoned and unpoisoned versions. The bar plots show attack success rate (ASR) for each target label and prompting model, while the line plots illustrate the mean ASR across all target labels.  The bar plots are sorted by target label and model.}}
\label{fig:score_n_attack_extra}
\end{figure}
\newpage

% don't forget to append the original project proposal
\resumechapters
\pagenumbering{gobble}
\includepdf[pages=1, pagecommand=\chapter*{Project Proposal}, offset=0 -1cm]{proposal}
\addcontentsline{toc}{chapter}{Project Proposal} 
\includepdf[pages=2-,pagecommand={}]{proposal}
%TC:endignore
\end{document}